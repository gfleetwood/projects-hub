\documentclass[]{book}
\usepackage{lmodern}
\usepackage{amssymb,amsmath}
\usepackage{ifxetex,ifluatex}
\usepackage{fixltx2e} % provides \textsubscript
\ifnum 0\ifxetex 1\fi\ifluatex 1\fi=0 % if pdftex
  \usepackage[T1]{fontenc}
  \usepackage[utf8]{inputenc}
\else % if luatex or xelatex
  \ifxetex
    \usepackage{mathspec}
  \else
    \usepackage{fontspec}
  \fi
  \defaultfontfeatures{Ligatures=TeX,Scale=MatchLowercase}
\fi
% use upquote if available, for straight quotes in verbatim environments
\IfFileExists{upquote.sty}{\usepackage{upquote}}{}
% use microtype if available
\IfFileExists{microtype.sty}{%
\usepackage{microtype}
\UseMicrotypeSet[protrusion]{basicmath} % disable protrusion for tt fonts
}{}
\usepackage{hyperref}
\hypersetup{unicode=true,
            pdftitle={A History Of Eaerwa},
            pdfborder={0 0 0},
            breaklinks=true}
\urlstyle{same}  % don't use monospace font for urls
\usepackage{natbib}
\bibliographystyle{apalike}
\usepackage{longtable,booktabs}
\usepackage{graphicx,grffile}
\makeatletter
\def\maxwidth{\ifdim\Gin@nat@width>\linewidth\linewidth\else\Gin@nat@width\fi}
\def\maxheight{\ifdim\Gin@nat@height>\textheight\textheight\else\Gin@nat@height\fi}
\makeatother
% Scale images if necessary, so that they will not overflow the page
% margins by default, and it is still possible to overwrite the defaults
% using explicit options in \includegraphics[width, height, ...]{}
\setkeys{Gin}{width=\maxwidth,height=\maxheight,keepaspectratio}
\IfFileExists{parskip.sty}{%
\usepackage{parskip}
}{% else
\setlength{\parindent}{0pt}
\setlength{\parskip}{6pt plus 2pt minus 1pt}
}
\setlength{\emergencystretch}{3em}  % prevent overfull lines
\providecommand{\tightlist}{%
  \setlength{\itemsep}{0pt}\setlength{\parskip}{0pt}}
\setcounter{secnumdepth}{5}
% Redefines (sub)paragraphs to behave more like sections
\ifx\paragraph\undefined\else
\let\oldparagraph\paragraph
\renewcommand{\paragraph}[1]{\oldparagraph{#1}\mbox{}}
\fi
\ifx\subparagraph\undefined\else
\let\oldsubparagraph\subparagraph
\renewcommand{\subparagraph}[1]{\oldsubparagraph{#1}\mbox{}}
\fi

%%% Use protect on footnotes to avoid problems with footnotes in titles
\let\rmarkdownfootnote\footnote%
\def\footnote{\protect\rmarkdownfootnote}

%%% Change title format to be more compact
\usepackage{titling}

% Create subtitle command for use in maketitle
\providecommand{\subtitle}[1]{
  \posttitle{
    \begin{center}\large#1\end{center}
    }
}

\setlength{\droptitle}{-2em}

  \title{A History Of Eaerwa}
    \pretitle{\vspace{\droptitle}\centering\huge}
  \posttitle{\par}
    \author{}
    \preauthor{}\postauthor{}
      \predate{\centering\large\emph}
  \postdate{\par}
    \date{2020-01-25}

\usepackage{booktabs}
\usepackage{amsthm}
\makeatletter
\def\thm@space@setup{%
  \thm@preskip=8pt plus 2pt minus 4pt
  \thm@postskip=\thm@preskip
}
\makeatother

\begin{document}
\maketitle

{
\setcounter{tocdepth}{1}
\tableofcontents
}
\hypertarget{a-history-of-eaerwa}{%
\chapter{A History Of Eaerwa}\label{a-history-of-eaerwa}}

Being a guide to the history and backstory of THE SECOND APOCALYPSE series by R. Scott Bakker, comprising THE PRINCE OF NOTHING trilogy and THE ASPECT EMPEROR quartet.

Words by Adam Whitehead, artwork by Jason Deem, based on the work of R. Scott Bakker

\hypertarget{part-1-the-fall-of-the-ark-of-the-heavens}{%
\chapter{Part 1: The Fall of the Ark of the Heavens}\label{part-1-the-fall-of-the-ark-of-the-heavens}}

Before the dawn of antiquity, well over five thousand years ago, the lands of Eärwa were the domain of a beautiful, ancient and long-lived race. They called themselves the ji'cûnû roi (or, more familiarly, Cûnuroi), the People of the Dawn. Men, not known for their flights of imagination, would later call them Oserukki, ``Not Us'', or in everyday parlance, the Nonmen. The origins of the Cûnuroi are lost to speculation and debate, but in their mythology their civilisation was established by the Ur-King Imimorûl (who may or may not have been a god himself), who fled from the Heavens to seek sanctuary far below the ground, away from the watchful eyes of jealous gods and the hungry sky. He removed his shield-arm and used its bones to create the first Nonmen.

The Cûnuroi in those days were not immortal, but their lifespans were measured at over four centuries. They raised great citadels - mansions - within the mountains of Eärwa and forged mighty artifacts and items of glorious beauty. They were the first to master the art of sorcery through their Qûya mages. Their noble-warrior caste, the Ishroi, became renowned for their absolute mastery of combat. The Cûnuroi did not give up their long lives lightly in battle, but from time to time strife was incurred and the mansions would go to war. But for the most part they lived in peace.

At first, they paid little heed to the savage race of fur-clad primitives who spread across
Eärwa, dwelling in the forests and between the mountain peaks. They called this race
j'ala roi (more familiarly, Halaroi), the People of Summer, for their lives were fleeting
but flamed so hot and passionate. But as the Halaroi spread in number, breeding at a
far faster rate than the slow-burning Cûnuroi, the Nonmen saw the wisdom in
subduing the interlopers. The Nonmen Mansions broke the spirit of these men,
reducing them to becoming the Emwama, a tribute race who existed as slaves of the
Cûnuroi. They delved the deep mines for their long-lived masters and tilled the fields
for them. The sole exception was in the uttermost north, where the Cûnuroi of Viri
instead employed men and treated them with more respect, to the derision of their
peers.

The number of Nonmen Mansions is unknown, although it is known that the
mightiest and oldest, the High Mansions, numbered nine. The greatest, most powerful
and most populous (at least originally) was Siöl, located deep under the northern Great
Kayarsus Mountains, defending a series of fortified passes leading east into Eänna, a
land of primitive Halaroi tribes which the Nonmen harvested for slaves. The proudest
and most fiercely independent was Nihrimsûl, located under the south-eastern Yimaleti
Mountains. Cil-Aujas lay in the east, under the peak of Aenaratiol at the south tip of
the Osthwai Mountains. Ishoriöl, the Exalted Hall, lay in the far west, beyond the
Demua Mountains near the shores of the Great Ocean. Viri lay underneath the peak of
Antareg in the Urokkas, a small range of mountains overlooking the Sea of Neleöst, the
Misty Sea. Illisserû, the Lighthouse, was located in the Betmulla Mountains overlooking
the Sea of Meneanor, with Curunq under the Araxes Mountains of the east and Cil-
Aumûl under the Hinayati Mountains of the far south. Western-most of the Nine was
Incissal, located under hills near the Great Ocean.

Siöl claimed to be the oldest Mansion, the House Primordial, founded directly by
Imimorûl himself at the dawn of days. This claim was disputed by Nihrimsûl, who
instead claimed to be the true House Primordial. In the Nihrimsûli tradition it is said
that Imimorûl discovered his son Tsonos and daughter Olissis engaged in a passionate
relationship and was angered. Fearful of his wrath, the children murdered their father
and fled to find safety in the Deepest Deeps beneath the Kayarsus, where they
established Siöl. In the Siölan version, Imimorûl founded Siöl himself and died
peacefully, with Tsonos succeeding him. Nihrimsûl was founded by a lesser child of
Imimorûl, with inferior blood. Theological disputes arose from these disagreements
which in turn led to war: at one stage Nihrimsûl endured the so-called Thousand Year
Siege before an uneasy peace was made.

Siöl founded Ishoriöl, Viri, Illisserû and Cil-Aujas as colonies, and they in turned
foundedOf the Nine, much could be said: Ishoriöl, the Exalted Hall (which in more recent
times men have called Ishterebinth) waxed quickly, its cultural and economic power
impressing even Siöl. Illisserû was the one Mansion to be built overlooking the sea and
whose people built vessels to sail on the water, and its people soon became regarded as
strange for this obsession. But of the Nine, few could rival Viri in tragedy and
misfortune.

Of the Nine, much could be said: Ishoriöl, the Exalted Hall (which in more recent
times men have called Ishterebinth) waxed quickly, its cultural and economic power
impressing even Siöl. Illisserû was the one Mansion to be built overlooking the sea and
whose people built vessels to sail on the water, and its people soon became regarded as
strange for this obsession. But of the Nine, few could rival Viri in tragedy and
misfortune.

Viri controlled a great swath of territory. Its dominion extended across either side of
the River Sursa, reaching north and westwards through thickly-forested lands to the
frigid and massive Yimaleti Mountains and eastwards around the curving shoreline of
the Misty Sea. Many Halaroi dwelt in these lands, but the Cûnuroi of Viri treated with
them and bartered for their service, to the amusement of their southern kin. The
Nonmen of Viri nevertheless lived peacefully and prosperously. Until the waxing of
Imburil.

Imburil, the Newborn, was the name given by the Cûnuroi to the pole star, the
brightest star in the sky. Men would later call it the Nail of Heaven, the star around
which all others turned. One night the star suddenly blazed with a strange, sudden
intensity. This waxing lasted a long time and then abated. The Cûnuroi could not
explain it, but then dismissed it as a curiosity and moved on with their lives.

Three years after this curious event, death came swirling down.
A colossal crack sounded around the world, briefly shaking even the foundations of Cil-
Aujas some two thousand miles to the south-east. The skies turned red as far away as
Siöl, and it was clear that something momentous had happened in the far north-
western corner of Eärwa, in the southern reaches of the Yimaleti Mountains.

The Nonmen of Viri had a far closer view.

There, the shaking came as terrific, terrible waves of destruction. Some of the lower
halls of the mansion collapsed. Passages caved in and the great mines were laid waste.
Tens of thousands of Cûnuroi were killed. Some, in desperation, attempted to flee the
mansion altogether. Those who did beheld - briefly - a firestorm sweeping out of the
west, destroying all before it. The great forests of western Viri were vapourised, the
farmlands obliterated and the outer walls of the mansion scorched.

The devastation was total and the intensity of its fury was terrifying: Viri was still
hundreds of miles from where the cataclysm had taken place. Almost equidistant was
Ishoriöl, which lay to the south-west of the Yimaleti Mountains beyond the Leash, the
long straits linking the Great Ocean to the Misty Sea. Ishoriöl was luckier, however.
The south-western Yimaleti Mountains and the hills around Ishoriöl helped deflect the worst of the damage away. Furthermore, the fertile hinterlands of Ishoriöl - Injor-Niyas -
were located further to the south and were not affected by the cataclysm. Although
damaged, Ishoriöl was able to recover quickly without outside help.

The same was not true of Viri. Its rich tributary lands had been utterly destroyed, its
population reduced catastrophically. Its few surviving Halaroi client-tribes were reduced
to begging at the gates of Viri for aid. Foreseeing disaster, King Nin'janjin sent word to
Siöl, greatest of the mansions and requested succour from King Cû'jara Cinmoi, the
greatest Cûnuroi ruler of the age.

The Sky has cracked into potter's shards,
Fire sweeps the compass of Heaven,
The beasts flee, their hearts maddened,
The trees fall, their backs broken.
Ash has shrouded all sun, choked all seed,
The Halaroi howl piteously at the Gates,
Dread Famine stalks my Mansion.
Brother Siöl, Viri begs your pardon.

Cû'jara Cinmoi read the message and realised he had no choice but to act: his armies
swept across the borders of Viri and invaded the territory of Nin'janjin. Incredulous but
unable to resist, Nin'janjin chose subjugation. He allowed the forces of Siöl to occupy
his kingdom without giving battle. It was humiliating, but it also saved his mansion.
The forces of Siöl prevented its extinction, for the price of its sovereignty.

His bloodless victory assured, Cû'jara Cinmoi turned his eye west to where the
cataclysm had taken place. The land was still scarred and blackened, but the immediate
firestorm had burned out and the way was passable, for someone with the will. Cû'jara
Cinmoi chose Ingalira, a great hero of Siöl, and sent him into the heart of the storm to
learn what had transpired.

Ingalira returned to Viri three months later with his report, which was hard to fathom.
According to him, a great golden vessel had been responsible for the devastation. It had
fallen from the sky with great speed and crashed into the mountains with tremendous
force. A vast circular depression had been created by the impact, with a new range of
peaks, the Occlusion or Ring Mountains, thrown up around its edges. Most of the
vessel was buried underground, with only two titanic golden horns -- one upright, the other canted -- reaching miles skywards from the site of impact, now a boiling cauldron
of molten rock. This vessel was dubbed the Incû-Holoinas , the Ark-of-the-Skies. Ingalira
attempted an exploration of the vessel, but its inhabitants -- with their clam-shaped,
skull-like heads and hooked wings -- were unpleasant to look upon and made noises
devoid of meaning. For this reason, and the fact that they came from the empty skies,
they were dubbed the Inchoroi , or People of Emptiness. Ingalira brought two of these
creatures back with him, but Cû'jara Cinmoi was so revolted by their aspect that he had
them slain on the spot.

History may have been better served had the Cûnuroi marched on the Ark
immediately, but Cû'jara Cinmoi had already seen a greater opportunity afforded by
the Arkfall. His forces bolstered by the survivors of Viri, he chose instead to make war.
His armies marched on Cil-Aujas and Nihrimsûl and subdued both in battle.
Sin'niroiha, King of Nihrimsûl, the ancient rival of Siöl, was forced to wash Cû'jara
Cinmoi's sword as a sign of supplication. Cû'jara Cinmoi became the High King of
Four Mansions, his Reach extending from the Yimaleti Mountains to the Sea of
Meneanor and from the Great Kayarsus to the Demua range. The might of Siöl was
uncontested.

The skies above the Incû-Holoinas cleared and the lands cooled. Western Viri had been
reduced to a wasteland, Agongorea , the Field Appalling, which stretched from the edges
of the Ring Mountains to the River Sursa, under the very walls of Viri. Nothing would
grow there and nothing could live there. Not even footsteps would make a mark on the
dead ground. Viri's power and might had been exhausted, even before its conquest by
Siöl. Nin'janjin brooded on Cû'jara Cinmoi's betrayal and the reduction of his
mansion, and dreamt of vengeance.

A Watch had been placed on the Incû-Holoinas. The Ark was ringed by sentries and
forts, but somehow a delegation of Inchoroi slipped through their lines. They came to
Nin'janjin in secrecy and spoke to him, but this time in the Ihrimsû tongue of the
Cûnuroi. The Inchoroi could now speak through faces they had somehow grafted into
their mouths. They claimed that the descent of their vessel was uncontrolled and the devastation suffered by Viri unplanned. They regretted the cataclysm and offered to
make amends. The Inchoroi would make alliance with Viri and stand with them
against Siöl. They would give Nin'janjin the power to avenge Cû'jara Cinmoi's
treachery. Against the advice of many of his Ishroi and Qûya advisors and despite his
own misgivings, Nin'janjin agreed.

Viri revolted. The Siölan occupation forces were slaughtered or enslaved. The Inchoroi
swarmed from the Ark under the command of their king, Sil, and overthrew the Watch
in the First Battle of the Ark. Only the two great twin heroes of Siöl, Oirinas and
Oirûnas, survived to relate news of the peril to Cû'jara Cinmoi. The High King
gathered his armies and marched north-west to meet the threat on the field of Pir-
Pahal, south-east of the Sea of Neleöst. Nin'janjin assembled the might of Viri there to
await them. However, when the Inchoroi host arrived the Ishroi of Viri became
disturbed, for the Inchoroi wore festering bodies as garments of war. Their obscene
appearance offended the Cûnuroi of Viri as it had Cû'jara Cinmoi. The warrior
Gin'gûrima confronted Nin'janjin and realised that the king's desire for vengeance and
redress had overthrown his reason. ``Hate has blinded him!'' he cried, and soon most of
the host of Viri had taken up the cry. They refused to fight alongside the Inchoroi. The
Inchoroi, fearing that the Nonmen planned to join Cû'jara Cinmoi and turn on them,
attacked first, hoping to destroy them ere the arrival of the might of Siöl.

The battle was hard-fought, the Cûnuroi valour and skill at arms and sorcery proving a
match for the Inchoroi's weapons of light, which scythed through their ranks with
abandon. The Inchoroi would have likely won regardless, but the Nonmen of Viri only
had to hold back the threat until the armies of Siöl arrived. Finding his once-vassals
beleaguered, Cû'jara Cinmoi threw himself into the fray.

The battle lasted a day and a night. The armies of Siöl were tested by the Inchoroi
weapons but triumphed. Cû'jara Cinmoi himself faced and defeated Sil, slaying him
where he stood and seizing his weapon, Sûrgoil, ``Shining Death'', which in a later age
men would call the Heron Spear. The Inchoroi broke and ran, fleeing back to the Incû-
Holoinas. The Cûnuroi followed, planning to destroy them once and for all, but word
came of disasters in distant corners of the Siölan empire: Cil-Aujas and Nihrimsûl had
revolted and broken free of the yoke of Siöl. Believing the Inchoroi broken and
finished, Cû'jara Cinmoi ordered Oirinas and Oirûnas to return to the Ark and set a
renewed Watch on it. Cû'jara Cinmoi then took the bulk of his forces back east to
retake the rebellious mansions. He won back Cil-Aujas in a hard campaign, but
Nihrimsûl and its king, Sin'niroiha, refused to concede. Battle after battle was fought,
to no avail.

Dozens of bloody battles resulted to no end, but proud Cû'jara Cinmoi refused to treat
until Sin'niroiha became King of Ishoriöl through marriage to the sorceress Tsinirû,
having won the right to wield the fabled Diurnal or Day Lantern from Emilidis, the
Artisan. Hearing the news, Cû'jara Cinmoi relented and sent a message to the High
King of Nihrimsûl and Ishoriöl: ``A King of Three Mansions may be Brother to a King
of Two.''

During this time Cû'jara Cinmoi had been forced to denude the Second Watch of
Cûnuroi warriors. To replace them, Oirinas and Oirûnas recruited from the primitive
tribes of men. Among them was Sirwatta, a man who had seduced the wife of a high-
ranking Ishroi and gotten her with child, a daughter named Cimoira. Cimoira was
raised as a Cûnuroi but Sirwatta was banished to the Watch. During his exile, he chose
to enter the Ark. He disappeared for a month and was assumed dead, but then
emerged, deranged and speaking stories so worrying that he was borne directly to
Cû'jara Cinmoi. What news was related was unknown and Sirwatta was ordered to be
put to death. By other accounts, this order was rescinded and Sirwatta merely had his
tongue removed.

More years passed and Cû'jara Cinmoi grew old and infirm. His eyesight dimmed and
the end seemed to approach. At this time Nin'janjin returned, begging Mercy and
Penance as per the ancient codes. Cû'jara Cinmoi granted him audience, but was
amazed to see that Nin'janjin had not aged a day since he last saw him on the Field of
Pir-Pahal, many decades earlier. Nin'janjin confirmed that the science of the Inchoroi
preserved him. He told Cû'jara Cinmoi that the Inchoroi lived in stark terror of the
might of Siöl, so remained in the Ark in misery. They begged to sue for peace.
Nin'janjin asked what tribute they could pay to temper the High King's fury.

The High King said, fatefully, ``I would be young of heart, face and limb. I would banish
Death from the halls of my people''. His counsellors urged him otherwise, but Cû'jara
Cinmoi had seen Nin'janjin's vigour and it awoke in him a greed for the return of his
own youth and strength. The Second Watch was disbanded and the Inchoroi allowed
to minister to the Nonmen of Siöl as their physicians. This began the time of the
Inoculation.

The Inchoroi gave the treatments and ministrations to the Cûnuroi that would both
bless them with immortality and doom them. Soon their effectiveness became clear, as
the Cûnuroi of Siöl grew in strength and skill, their youth restored to them. The other
Mansions abhorred the Inchoroi, but the fear of death gripped them all one. One by one, they gave in and allowed the Inchoroi to practice their arts on them as well. Only
the High King Sin'niroiha refused.

After a century, all seemed well and the power of the Cûnuroi waxed again, until
Hanalinqû, the legendary wife of Cû'jara Cinmoi, died of an affliction. The Inchoroi
strove to save her, to no avail and Cû'jara Cinmoi praised their diligence. But soon
other Cûnuroi women started to die, first a few and then scores. The Inchoroi fled en
masse , abandoning the mansions to return to the Ark. Cû'jara Cinmoi realised with
horror that he and his entire race had been deceived, and poisoned. The Womb-Plague,
as it was called, consumed the entire race and killed every single woman it touched.
Within a few scant years fully half of the Cûnuroi species had been murdered, and the
ability of it to reproduce removed, forever.

Cû'jara Cinmoi called for a muster of arms like nothing before seen in Eärwa. Not just
the mansions under his control, but every stronghold of the Nonmen between the
Yimaleti Mountains and the shores of the Three Seas in the uttermost south
responded. The might of the Nine High Mansions assembled. Cû'jara Cinmoi led this
army through the Occlusion and onto the Inniür-Shigogli, the Black Furnace Plain that
lay about the Golden Horns of the Incû-Holoinas. There he laid down the body of his
slain wife and demanded that the Inchoroi answer for their crimes.

But the Inchoroi had prepared for this day. For many long years they had practiced foul
skills, melding technology and flesh to create hordes of horrific servants: Sranc, a
piteous abomination of Cûnuroi, given to obscene hungers of the flesh; powerful
Bashrags, tall, fierce and hideous warriors of tremendous size and strength; and Wracu,
winged beasts whom men would later call dragons. These boiled forth to assail the
Cûnuroi on the day later known as Pir Minginnial.

The Cûnuroi may have yet carried the day, for their numbers were immense (thirty
thousand at the very least), their shields tall and their sorcery strong. But the Inchoroi
had seduced the Qûya practitioners of the Aporos, the form of sorcery focused on
negation . These sorcerers had created for the Inchoroi devices they called Chorae ,
trinkets, later called by men ``Tears of God''. Each Chorae was a small iron sphere,
banded in runes inscribed in the Qûya language and one inch in diameter. Anyone
wearing a Chorae was rendered immune to sorcery. If a Chorae came into contact with
a sorcerer, it killed them instantly, transforming their bodies into pillars of salt. The
Chorae turned the tide of battle, slaughtering the Qûya by the dozens and reducing the
struggle to one of swords, teeth and talons.

The heroes of Eärwa struck back. Ciögli the Mountain, strongest of the Ishroi, broke
the neck of Wutteät the Black, the Father of Dragons (although he survived). Oirinas
and Oirûnas fought side-by-side, slaughtering Sranc and Bashrags by the score. Ingalira
strangled Vshikcrû, one of the mightiest of the Inchoroi, and cast his burning body
down. The Cûnuroi would not relent and would not yield.

The battle only turned when Nin'janjin, his hatred not dimmed by the passage of
generations, found and battled Cû'jara Cinmoi. He slew the High King of Siöl and
sundered his head from his body. The mighty Gin'gûrima fell, gored to death by a
Wracu. Oirinas was slain by an Inchoroi spear of light. Sin'niroiha, the High King of
Nihrimsûl and Ishoriöl, rallied the surviving Nonmen (retrieving the Heron Spear
along the way) and they began a fighting retreat, withdrawing to the Ring Mountains.

The Inchoroi, despite their hordes of slave-servants and their Chorae, were reluctant to
pursue. They had suffered grievous losses. The Black Furnace Plain was covered in the
bodies of Sranc, Bashrags, Wracu and Inchoroi themselves. The Inchoroi chose not to
pursue their foe but to regroup.

This proved to be a mistake, although at first it did not seem so. The Cûnuroi retreated
to their mansions to raise fresh troops, but they could not replenish their losses. The
Inchoroi bred countless more Sranc and Bashrags to throw at their foe, and the Isûphiryas , the record of Cûnuroi history, recorded nothing but defeat after defeat for
decades. Most famed of these was the Battle of Imogirion, when the Cûnuroi of
Illisserû launched a daring raid on the Incû-Holoinas by sea. Despite the bravery of the
move and taking the Inchoroi by surprise, the attacking force was defeated in a
nocturnal slaughter on Agongorea. Less than a hundred warriors survived to return to
their ships and only one survived the storms of the voyage to see home again.

But the Inchoroi were also a dying race: they could also not replenish their losses, and
every Inchoroi that fell was a major victory for the Cûnuroi. And the Cûnuroi were,
even in their reduced state, far more numerous.

The Inchoroi were also overly reliant on their weapons of light and their technology,
but these were reliant on the Ark and the Ark seemed to be failing. One-by-one, the
Inchoroi spears of light ceased working. Their other weapons likewise failed, and their
ability to create countless Wracu, Sranc and Bashrags became reduced. They were
forced to let the creatures breed instead, and this was a slower process.

Politics also worked in the Inchoroi's favour, at least at first. Sin'niroiha was not of the
Blood of Tsonos, the line of Nonman descent from Imimorûl's eldest son, and even in
this time of great need the other mansions were reluctant to follow his lead. Even his
marriage to Tsinirû of Ishoriöl did not fully win over the other Cûnuroi. It was only
after Sin'niroiha's death of old age (as he had never endured the Inoculation), during
the Siege of the Second Delve, that the Blood of Tsonos was reunited in his son
Nil'giccas, who was able to unite the mansions once again under his leadership.

Finally, nigh on five centuries after the defeat at the Black Furnace Plain, the Inchoroi
were driven back into the Ark at the Battle of Isal'imial. No longer did the Cûnuroi call
it the Incû-Holoinas, the Ark-of-the-Skies. Now they called it Min-Uroikas , the Pit of
Obscenities, Golgotterath in the tongues of men. The Cûnuroi set about a methodical
eradication of the Ark, scouring it hall by hall. It took twenty years to explore and
secure every last hold and every last corner of the vessel but finally it was done. The
Inchoroi were pronounced eradicated, destroyed and defeated.

But a problem emerged, a secret so profound it was struck from the Isûphiryas itself.
During the exploration of the Ark a room had been discovered, the Golden Court of
Sil, from where the Inchoroi had prosecuted their war. In this room was located an
item or force known only as the ``Inverse Fire''. Every Cûnuroi who had ever beheld
this artifact had been driven insane on the instant, proclaiming that the Inchoroi were
right and all the people of the World were doomed to damnation. It was this item
which had turned Nin'janjin to the enemy's cause.

Nil'giccas ordered three of his greatest servants -- the Ishroi warriors Misariccas and
Runidil, and the Qûya mage Cet'ingira -- into the Golden Court to investigate further.
They returned, Misariccas and Runidil raving that the Inchoroi had indeed been
correct and that all the Cûnuroi were doomed to damnation and hellfire. Nil'giccas
asked for Cet'ingira's opinion and he replied that his companions had been subverted
by the Inverse Fire and had gone over to the foe. He advised that Nil'giccas kill them
on the instant. He agreed. Cet'ingira, the lone Qûya who had beheld the Inverse Fire
and apparently resisted it, was spared.

Unable to actually destroy the vessel itself, Nil'giccas, King of Ishoriöl, ordered the
Qûya under Emilidis, the Artisan, to seal the vessel from the outside world. The
Artisan created his greatest achievement, the Barricades, it to hide it away from the rest
of the world and prevent entry. The Cûnuroi were forbidden from speaking of the
accursed place, or telling others where it lay.

The Cûnuroi had achieved their victory, but at catastrophic cost. Millions of their race
had been slaughtered. Every last Cûnuroi woman had been killed. There was no way to
restore their race, or save it. They were ageless, but not invulnerable. They could die in
battle. The very passage of time exacted a toll on their souls, their memories fading
until only the most horrific and scarring remained, and soon they began to descend
into madness.

Worse still, although this would not be known to the Nonmen for millennia, the
Cûnuroi victory was incomplete. Two Inchoroi yet lived: Aurang, the Warlord, and his
brother Aurax. Two where there had once teemed multitudes, surviving the Arkfall to
deliver war upon the populace of the World. But two could not hope to succeed where
tens of thousands had failed, not without allies. Looking beyond the eastern horizon,
they realised a weapon existed, a people who could become an army , who could do the
job for them, if they could manipulate and control them to fulfil their desire.

To this end, they went amongst those people, learned their sacred stories from them,
and created a fantastic artifact with those stories and scriptures inscribed upon it\ldots{}with
one alteration. Included on this artifact of bone and wisdom was a command from the
Gods, to journey into the west and kill the ``False Men'' wherever they could be found.

The Age of the Cûnuroi waned even as, in the lands of Eänna beyond the great eastern
mountains, the Age of Man began.

\hypertarget{part-2-the-age-of-man}{%
\chapter{Part 2: The Age Of Man}\label{part-2-the-age-of-man}}

Eärwa is seen as the cradle of civilisation, the home of the Cûnuroi and the greatest
nations in the history of the world. But it is not the only continent in the world. To the
south, beyond the Three Seas, lies the desolate desert land of Kutnarmu, dominated by
vast deserts and untraversable wastes. To the east, beyond the vast Kayarsus Mountains,
lies another land: Eänna, the Land of the Uplifted Sun.

Little is known of Eänna, even today. Explorations of that continent have revealed
mountains, deserts and plains, populated sparsely. Such explorations have not
progressed far before turning back for lack of supplies, or have not returned at all. But
we know that the earliest tribes of independent men dwelt in Eänna. Even as their
western brethren were enslaved by the Cûnuroi, becoming the Emwama, the men of
Eänna were building the rudiments of civilisation. Over time they became divided into
five distinct tribes: the Ketyai, the Norsirai, the Satyothi, the Scylvendi and the
Xiuhianni. The Tribes warred against one another but were also united by religion. The
Tribes came to believe in the Hundred Gods, a hundred distinct, individual spiritual
20entities, divine aspects of the God-of-Gods, who responded to their prayers and
intervened in the affairs of men.

Originally, sorcerers were respected amongst the tribes. They were seen as wizards and
prophets both, Shamans , and for centuries they dominated the spiritual discourse of
humanity. But others became jealous of their power. The Old Prophets, non-sorcerous
servants of the Hundred, rose to power, propelled by divine miracles. They proclaimed
that all sorcerers were damned to an eternity of pain and suffering for their challenge of
the gods' power. In time the Shamans were defeated, and sorcery outlawed by the word
of the Gods. The sorcerers were reduced to the Few, a tiny number of warlocks and
witches practising at the very fringes of society, whilst the philosophical notion of the
God-of-Gods lost importance, becoming a ``placeholder'' in theology.

The new religion, the Kiünnat tradition, was given form and structure through the
Tusk. A colossal bone-artifact, the Tusk was inscribed with the holy words and stories of
the Five Tribes, accumulated over centuries. It was gifted to the Tribes by strangers who
journeyed out of the west. The age of the Tusk is unknown, save it far predates the
Breaking of the Gates, which took place (according to tradition) 4,132 years before
Anasûrimbor Kellhus's Great Ordeal marched onto the Istyuli Plains.

The Tribes of Men dwelt in the wilds of Eänna for centuries. Their forays into Eärwa
were met with enslavement or death at the hands of the Cûnuroi, whom the Tribes
soon came to curse as the Oserukki, ``Not Us'', the Nonmen. It was a hard existence in a
hand land.

Despairing of the lot of men, Angeshraël, a priest or holy man of the Tusk, climbed to
the peak of Mount Eshki, fasting and praying to the Hundred for guidance. At length,
he descended from the mountain and found a hare to skin and eat. Once he had his
fill, he was joined at his fire by a man, at first glance a traveller of the wastes. But
Angeshraël recognised the man as the god Husyelt, the Holy Stalker made manifest,
and fell to his knees. Husyelt asked why he did not throw his face into the earth as
homage demanded, so Angeshraël did as he bid, even though this meant bowing his
head into his fire. Angeshraël burned his face, but the god acknowledged his piety.
They talked for a time before Husyelt left him. The experience ended the time of
Angeshraël the man and began the time of Angeshraël, the Burned Prophet.

He went amongst the Five Tribes, declaring that beyond the western mountains lay a
land of bounty and gift which was the rightful birthright of the Tribes. It was held by an
accursed race of False Men whose extermination was called for by the Tusk itself. The
False Men wielded great powers, but when the Tusk was delivered unto the Tribes
certain ``gifts'' had come with it, metal spheres which would render these powers useless.
Angeshraël's words spread amongst the Tribes and soon found great favour. He urged
those who would follow him to gather on the slopes of Mount Kinsureah.

There Angeshraël made his final case, arguing for the Five Tribes to cross the Great
Kayarsus and claim the Land of the Felled Sun, Eärwa, the promised land. There was
tremendous doubt and discussion. One of the Five Tribes, the Xiuhianni, rejected his
words and left, scattering back into Eänna. But Angeshraël convinced the rest by
performing a great sacrifice, slaying his son Oresh as a sign of his conviction.

The four remaining tribes agreed to follow the Burned Prophet. In their multitudes,
they swarmed through the mountain passes of the Great Kayarsus and found their way
barred by the Gate of Thayant, the Gates of Eärwa, which the Nonmen had fortified in
ages long past. Great assaults were made, but even reduced by their wars with the
Inchoroi, the Cûnuroi were able to throw back every assault through their Qûya mages.
The Tribes had the power of the Chorae to aid them, but even this was checked by the
Qûya, who had experience of resisting the weapons.

Reluctantly, an accord and truce was made amongst the Four Tribes. It was agreed that
sorcerers could again practice their magic in the service of the Gods. Thus, the first
sorcerous schools were founded, organisations of wizards and warlocks (but not
witches; the Kiünnat teachings insisted on the inferiority of women in matters both
temporal and spiritual, a stipulation that would not be thrown down until the time of
Kellhus four thousand years later). Although practicing Anagogic sorcery (rooted in
tradition and superstition) far inferior to the Gnosis of the Qûya (rooted in
mathematics and logic), the sheer number of the Halaroi sorcerers soon overcame the
Nonmen.

The Gate of Thayant was broken and shattered. Beyond, according to many historians,
lay the great Cûnuroi mansion of Siöl itself. The Four Tribes ranged through its halls,
putting the Cûnuroi to death and casting down the far gates, allowing the Halaroi in
their tens and hundreds of thousands to swarm through and out onto the plains of
north-eastern Eärwa.

The Tribes threw down the gates in ruin, an act immortalised as the ``Breaking of the
Gates'', the beginning of recorded history and also the beginning of the Second Age,
Far Antiquity and the Age of Bronze.

The four tribes swept across Eärwa from the north and east, throwing down the great
High Mansions one-by-one. Some fought bitterly, but others, traumatised by millennia
of tragedy, opened their gates and bared their throats to the inevitable. After the fall of
Siöl, Nihrimsûl followed, and then the remnants of Viri and far Illisserû. Only Ishoriöl
and Cil-Aujas survived of the great mansions. The Cûno-Halaroi Wars were fought over
generations but ended in the defeat of most of the Cûnuroi in Eärwa.

Once secure in Eärwa, the tribes found new homes. The hardy Norsirai settled the
north, particularly the lands to the south of the Sea of Neleöst along the fertile River
Aumris. The Scylvendi settled the lands further south, between the Atkondras
Mountains and the Hethanta Mountains, on the Jiünati Steppe and the lands south as
far as the unhospitable Great Carathay Desert. The Ketyai, the most numerous tribe,
made their home on the rich Kyranae Plains and the lands extending north and east
around the Meneanor Sea and Sea of Nyranisas, as far east as the Southern Kayarsus.
The Satyothi went to the far south-west of Eärwa, beyond the Atkondras range and
Carathay Desert, settling the lands to the west of the Hinayati Mountains as far as the
Great Ocean itself.

The first human nations arose soon after. The Satyothi, isolated from the rest of Eärwa
by geography and distance, established a kingdom known as Angka, a forerunner of
modern Zeüm. The Ketyai established the kingdom of Shigek, the first nation of the
Three Seas, around the broad delta of the River Sempis. Another Ketyai kingdom was
established at Nilnamesh in the far south, beyond the Carathay Desert. The Scylvendi
disdained the trappings of civilisation, preferring to remain pastoralists dwelling on the
steppes and plains.

But it was in the Ancient North that human civilisation first truly took hold in Eärwa.
The River Aumris and the surrounding region became the focus of such settlements,
with the great Norsirai cities of Trysë, Sauglish, Etrith, Lokor and Ûmerau founded in
relatively short order. Controversially, these city-states disdained the command of the
Tusk to exterminate the Nonmen and began trading with the Cûnuroi of Ishoriöl to
the north-east, to their mutual enrichment. The power of the Aumris River cities grew
quickly. Somewhere in the 4th Century after the Breaking of the Gates, Cûnwerishau,
the God-King of Trysë, made a pact with Nil'giccas, the King of Ishoriöl. He received a
copy of the Isûphiryas , the chronicle of the history of the Nonmen prior to the
25Breaking of the Gates and the oldest extant work of literature in the world. The Cûno-
Halaroi Wars ended with an accord of peace and trade.

By 430 Year-of-the-Tusk, the God-Kings of Trysë had been overthrown and Ûmerau
had became the primary power of the Aumris River Valley. By 500 the Ûmeri Empire
had formed, the first truly great empire of men, extending along the full length of the
Aumris River and extending across the lands to either side. This also coincided with the
growth in power of the Ketyai to the south, with the Seto and Annaria tribes colonising
the length of the River Sayut and the Secharib Plains.

In 555 the Nonman Tutelage began. The Cûnuroi, mostly of Ishoriöl, formed an
alliance with the Norsirai of the Ûmeri Empire and began teaching them in arts both
mundane and sorcerous. Most notably, the Nonmen Qûya imparted to the Norsirai
Anagogic sorcerers the secret of the Gnosis, the most powerful form of sorcery known
to exist. It was also around this time that the subtle Cûnuroi game of benjuka was also
taught to men. Those Nonmen who went to live amongst humans and serve them as
teachers were called Siqû.

The next three centuries saw the Ûmeri Empire flourish thanks to this alliance. In 560
the Great Library of Sauglish was founded by Carû-Ongonean, the third Ûmeri God-
King. Ten years later he founded the fortress of Ara-Etrith, ``New Etrith'', which would
later be called Atrithau.

Emilidis, the Artisan, the creator of the Sublime Contrivances (the greatest works of
Nonman sorcery, including the Day Lantern, the Immaculate Rim and the Barricades
themselves), deigned to teach some of the sorcerers of man. About 661 he founded the
Gnostic School of Mihtrûlic, the Contrivers. Gin'yursis, a Cûnuroi of Cil-Aujas exiled
from his home mansion, travelled to the Ancient North and undertook tutoring of
men in the arts of sorcery. In 668 he founded the Gnostic School of Sohonc. His
student Sos-Praniura would then found the Gnostic School of Mangaecca in 684,
under the tutelage of Cet'ingira. The power of the Ancient North increased thanks to
these schools of learning and sorcery.

In 750 the Heron Spear, Suörgil (``Shining Death''), seized from the Inchoroi King Sil
by Cû'jara Cinmoi himself, vanished from its place of safekeeping, deep in the heart of
Ishoriöl. Unbeknown to the rulers of that mansion, Cet'ingira (later ``Mekeritrig'',
``Traitor of Men'') had arranged the theft. After the end of the Cûno-Inchoroi Wars,
Cet'ingira had been sent into the Golden Court of the Incû-Holoinas on the orders of
Nil'giccas. He returned sane and whole, but his companions who had accompanied him
had been driven mad and were put to death. However, it now appeared that Cet'ingira
had subtly surrendered his allegiance to the Inchoroi, two of whom had - somehow -
survived the twenty-year purge of the Ark. Cet'ingira delivered the Heron Spear to the
Ark and arranged for it to be hidden in its environs (the interior remained barred from
entry). In 777 Cet'ingira set about the corruption of the School of Mangaecca, revealing
to them the existence of the Golden Ark and the Inchoroi. Over the following
centuries the Mangaecca raised great fortifications about the Golden Ark as they
attempted to penetrate Emilidis's Barricades, to no avail.

In 809 the great river town of Cenei was founded on the Kyranae Plains, soon
establishing itself as the greatest Ketyai power north of Shigek. Just two years later the
great kingdom of Akksersia was founded on the northern shores of the Sea of Cerish,
with its capital at Myclai.

The nations and city-states of Eärwa circa 1110 Year-of-the-Tusk.

In 825 the Nonman Tutelage ended with a crime most foul, a rape committed by the
Siqû Jiricet against Anasûrimbor Omindalea, the daughter of Sanna-Neorjë, a ruling
noble of the Ûmeri Empire. When Ishoriöl refused to hand over Jiricet for trial, the
Empire expelled all Cûnuroi from within its borders and ended the alliance.
Omindalea would die bearing Jiricet's son, Anasûrimbor Sanna-Jephera, known as
``Twoheart''. Holding the child blameless for the sins of his father, Sanna-Neorjë made
Sanna-Jephera his heir, resulting in the extreme longevity of many of his line.

By 850 Akksersia had sent colonists across the Sea of Cerish, founding the city of
Kelmeöl on the southern shores of the sea. The people of this region soon became
known as the Meöri. By 1104 the single city-state had expanded into the Meöri (or
Meörn) Empire, extending south to the River Wernma.

In 917 the Ûmeri Empire collapsed, overrun by the Cond tribesmen of Aulyanau the
Conqueror. This led to a second period of domination over the Aumris Valley by
Trysë. In 927 the Cond conquered Atrithau and settled several tribes in the region.

Shaeönanra, Grandmaster of the Mangaecca and the reviled leader of the Unholy Consult, is confronted
by Titirga, Grandmaster of the Sohonc, in 1119. Titirga wields the fabled Diurnal, the mighty Day
Lantern crafted by the Nonman Emilidis thousands of years earlier.

In the second half of the 11 th Century (some say 1086 but this would make him far
younger than he is oft-depicted) Shaeönanra was born in Ûmerau. He was the son of a
treasurer and showed tremendous aptitude for sorcery. He was taken in by the
Mangaecca and became the school's most promising student. By 1111 he had already
become the Grandvizier of the Mangaecca and had learned forbidden knowledge about
the Incû-Holoinas. Aided by the Cûnuroi traitor Cet'ingira, Shaeönanra set about
tearing down the glamour surrounding the Golden Ark, finally succeeding in
destroying it and making contact with the last two surviving Inchoroi, Aurax and
Aurang. In 1119 Shaeönanra and Aurang defeated Titirga, the Grandmaster of the
Sohonc and the most powerful sorcerer in history, after luring him into a trap in the
old Nonman mansion of Viri.

Shaeönanra, Cet'ingira and Aurang now formed an alliance, an Unholy Consult, with
the goal of finishing the job the Inchoroi had begun thousands of years earlier:
reducing the population of the World to 144,000 so it might be sealed shut against the
29Outside and ending the judgement of the heavens (and hells) on mortals, thus saving
their souls from eternal damnation.

In 1123 Shaeönanra announced to the world that he had discovered a means of saving
the souls of those damned by sorcery, utilising a secret hidden deep within the Ark, but
was promptly denounced for impiety. The Mangaecca were outlawed, fleeing Sauglish
for Golgotterath. For the next one thousand years they would attempt to bring about
the salvation they had promised.

By the end of the 13th Century Akksersia had become the most powerful Norsirai
nation, extending north from the Sea of Cerish onto the Plains of Gâl. At the same
time the city-state of Shir on the River Maurat had conquered the tribes of Set-Annaria
and founded new empire, Shiradi, trading with the Meöri to the north. However, the
Aumris Valley and the area around Atrithau had fallen under the yoke of the Scintya, a
new migratory tribe of Norsirai tribes. By 1381 Atrithau had liberated itself from the
Scintya and founded a new nation, Eämnor, which rapidly became one of the pre-
eminent powers of the Ancient North.

In 1408 Anasûrimbor Nanor-Ukkerja I, the Hammer of Heaven, defeated the Scintya
once and for all, driving them from the Aumris Valley in abject defeat. He then seized
the Ur-Throne in Trysë and declared himself the first High King of Kûniüri at the age
of just thirty. Kûniüri rapidly becomes the largest and most powerful empire of men in
Eärwa, extending north to the Yimaleti Mountains, east to the shores of the Cerish Sea,
south to Sakarpus and west to the Demua Mountains. Kûniüri was careful to maintain
good relations with Eämnor to the west, the Meörn Empire to the south-east and
Akksersia to the north-east. These four powers soon became immensely rich on trade
and cooperation, particularly against the Sranc who had begun to trouble the North in
worrying numbers (the Great Sranc Wars of the 13 th Century are now largely believed
to have been the work of the Consult, seeking to control the former servants of the
Inchoroi with mixed results).

Anasûrimbor Nanor-Ukkerja I died in 1556 at the age of 178, his long life the result of
Nonman blood in his veins. Upon his death, he divided the empire between his sons,
creating Aörsi (in the north, between the Neleöst Sea and the Yimaleti Mountains) and
Sheneor (in the east, between the seas of Neleöst and Cerish) in addition to Kûniüri
itself.

In the 15th Century, the Shiradi Empire was conquered by Xiuhianni invaders from
Eänna, who had crossed the southern Kayarsus in the vicinity of Jekhia. The city of Shir
was destroyed, but the imperial dynasty was able to relocate to Aöknyssus and, after
twenty years of warfare, managed to defeat the Eännan invaders. By 1800 the Shiradi
Empire had been firmly re-established and bolstered by the presence of the Surartu, an
Anagogic school of sorcerers (and forerunners of the modern Scarlet Spires) based at
the river fortress of Kiz in the city of Carythusal.

In 1591 the long, uneasy period of dominance by Shigek over the Kyranae Plains came
to an end. The Shigeki Empire had spent centuries gradually being sapped by internal
rebellions and clashes with Nilnamesh far to the south, particularly by Nilnamesh's
attempts to colonise the Middle-Lands of Amoteu on the Three Seas between their
empires. The native Kyranae plainsfolk managed to overthrow and defeat Shigek at the
Battle of Narakit. This was a precursor to the rise of the kingdom of Kyraneas itself,
with its capital originally at Parninas but later at Mehtsonc. Kyraneas defeated and
conquered both Shigek and Amoteu, forming a large empire stretching south along the
western coast of the Three Seas.

In 1896 Ajencis, who would soon be famed as the father of syllogistic logic and algebra,
as well as a philosopher of repute, was born in Mehtsonc. He would die in 2000, at the
age of 103, having written Theophysics , The First Analytic of Men and The Third
Analytic of Men, three of the greatest works of human knowledge and philosophy.

This was the age of great men, of warring cities and clashing empires. The lives of
humans were brief but passionate, the rapid rise and fall of empires likely bewildering
to the long-lived and slower-burning Cûnuroi. But there were also signs of growing
31maturity, with men like Ajencis (and his philosophical Kûniüri sparring-partner,
Ingoswitu) seeking true wisdom and larger, more stable nations forming such as
Kûniüri, based more around trade than warfare. What would have become of the great,
vast civilisation of the Ancient North and its neighbours in the Three Seas is fascinating
to speculate, but events meant that this was not to be.

In 2089 Anasûrimbor Celmomas II, the High King of Kûniüri, was born. In the exact
same year was born Seswatha, the son of Trysëan bronzesmith. These two men would
stand history upon its end, for they were fated to live in the time of the Apocalypse.

The nations and city-states of Eärwa circa 2089 Year-of-the-Tusk, on the eve of the Apocalypse.

\hypertarget{part-3-the-apocalypse}{%
\chapter{Part 3: The Apocalypse}\label{part-3-the-apocalypse}}

The man known to history as Seswatha and to the Sranc as ``Chigra'', ``Slaying Light'',
was born in Year-of-the-Tusk 2089 in Trysë, the son of a caste-menial bronzesmith.
Whilst still a child, he was identified as one of the Few, those that carry the Mark of
sorcery. He was taken to Sauglish to study with the Gnostic School of Sohonc, at the
time the largest and most powerful of the dozen or so sorcerous schools of the Ancient
North. Seswatha was a prodigy, his grasp of the Gnosis subtle and strong. Circa 2104,
at the age of fifteen, Seswatha would be proclaimed a sorcerer-of-rank, the youngest in
the School's history.

During this period Seswatha befriended Anasûrimbor Celmomas, the heir to the
imperial throne of Kûniüri who was studying with the Sohonc. The same age as
Seswatha and both intrigued by history, they became fast friends and allies. As Seswatha
grew in power and might through the ranks of the Sohonc, so Celmomas became
33famed as a warrior, general and scholar. Their great friendship was tested, however,
when Celmomas's most beloved wife Suriala (variously translated as Suiyela) gave birth
to their son Nau-Cayûti. Celmomas knew that Seswatha and Suriala shared a mutual
affection and became concerned that Nau-Cayûti was not of his blood. But such was his
love for his friend - and his inability to conclusively prove the truth of the matter - that
he did not have him publicly rebuked, merely withdrawing his friendship for a time.
Seswatha knew the truth, that Nau-Cayûti was his son, the result of a tryst with the
Queen while her husband lay in a drunken stupor in the King-Temple of Trysë.

Seswatha was a master sorcerer but also a keen politician. He befriended Anaxophus, a
young prince of Kyraneas, and treated with Nil'giccas, the Nonman King of
Ishterebinth (the ``Exalted Stronghold''), as Ishoriöl was now more frequently called.
Seswatha's insights were keen, his mind sharp, his sorcery formidable and his manner
one of ease, all attributes that saw him rise to become Grandmaster of the Sohonc in
his early thirties.

What happened next remains a matter of great debate. According to legend and The
Sagas , Seswatha received a delegation of Nonmen Siqû in Sauglish. Although the
Nonmen Tutelage was not reinstated, Seswatha had nevertheless forged closer ties with
Ishterebinth than had been seen since those times. According to some accounts,
Nil'giccas rewarded Seswatha's friendship with intelligence which was not so much
disquieting as alarming.

It had been long known that the School of Mangaecca had fled Sauglish to seek refuge
in Golgotterath. Its dark leader, Shaeönanra, survived thanks to Inchoroi knowledge
and his own sorcerous research. By the 14th Century, he had even been given a new
name: Shauriatas, ``Cheater of Gods''. The Mangaecca had not been seen since, but
their hand, and that of their Inchoroi overlords, was suspected in the Great Sranc
Wars, a series of strikes by hordes of Sranc out of Agongorea against Aörsi to the east
which had sorely tested that nation and led to the construction of a major stronghold,
Dagliash, on the Urokkas (in fact, atop the very ruins of ancient Viri). But in those days
Sranc were not a numerous, constant threat blanketing the North. They were mostly
34confined to Agongorea and the Yimaleti Mountains, and although their numbers were
concerning, they were not as inexhaustible as in later centuries. Or so it was supposed.

The Siqû warning was stark: the Mangaecca yet lived within the golden halls of the Ark
and they had formed a forsaken alliance - an Unholy Consult - with the surviving
Inchoroi princes. Worse still, their delvings and explorations of the Ark had uncovered
ancient secrets and disturbing ways of using the Tekne, the ancient art of science and
engineering that the Inchoroi had once employed to create weapons such as their staffs
of light and creatures such as the Wracu and Sranc, but had seemed to lose more and
more knowledge of with every passing year.

The Siqû warning convinced Seswatha that a threat was building in the pits of
Golgotterath and that, left unchecked, it would eventually destroy the world. This
threat was given a name by the Nonmen, one that Seswatha held close and only told
those closest to him: No-God.

The origin of this intelligence is unknown: some Mandate commentators suggest that
the Consult deliberately informed the Sohonc of the threat to trigger the very war that
now resulted, whilst others suggested that Consult traitors let the knowledge slip. But
given that all collaborators of the Consult were taken before the Inverse Fire and none
beheld its flames without breaking, this seems unlikely.
Whatever its origin, Seswatha took this knowledge to his old friend, who now ruled as
Anasûrimbor Celmomas II, High King of Kûniüri, the greatest nation in all Eärwa.
Celmomas may have been inclined to distrust his old friend for the alleged betrayal
with his wife, but he also respected his judgement. In the end, Celmomas was
convinced that Golgotterath remained a threat to the world and that threat needed to
be destroyed before it could unleash a horror that would bring about the end of
everything.

35The opening battles of the Apocalypse: 1. Sursa (2125). 2. The Great Investiture (2125-32). 3. Dagliash
(2133). 4. The Burning of the White Ships in Aesorea (2134). 5. Shiarau (2136).

In Year-of-the-Tusk 2123, Anasûrimbor Celmomas II called for the Great Ordeal, the
assembling of a vast host of armed and sorcerous might to be cast at Golgotterath, to
bring down and destroy the threat of the Consult and the Inchoroi once and for all.
Aörsi, which lay in the shadow of the Golden Ark and had suffered most from their
depredations, rallied to the call almost immediately, King Anasûrimbor Nimeric
contributing many tens of thousands of warriors already hardened in battle against
Sranc and Bashrags and the use of his fleet for transport and resupply. Nil'giccas sent
Qûya mages and Ishroi warriors from Ishterebinth, and Kyraneas sent a detachment of
troops, reflecting Seswatha's friendship with Prince Anaxophus (the prince himself was
still only fourteen, and it is unclear if he took part in the Ordeal at such an early age or
had returned to Kyraneas).

In 2124, the Great Ordeal crossed onto the plains of Agongorea but was engaged by a
host of Sranc and Bashrags. The resulting battle was indecisive and the Ordeal
withdrew across the Sursa to winter in Dagliash. Celmomas renewed the offensive in
the early spring, fording the Sursa before the Consult could prepare a defence. They
were forced to retreat to Golgotterath and allow the Ordeal to encircle it. The Great
Investiture lasted for six years but failed to starve the Consult into surrender.

36This period was marked by squabbling and petty jealousies erupting between the
commanders of the Ordeal, along with military disagreements on how to proceed. The
Investiture was complete, but the Consult seemed able to resupply at will. The Ark was
too well-defended for any conventional assault to succeed, and the Consult mages were
capable of resisting even the combined might of the Qûya and Sohonc. At one stage, in
the Great Chorae Hail, the Sohonc lost a third of their number in an ill-advised assault
ordered by Celmomas. Seswatha's reputation and leadership were tested but he
remained in command of the sorcerers in the Ordeal. In 2131, a more serious dispute
erupted between Celmomas and Nimeric, resulting in Celmomas withdrawing the
Kûniüri contingent of the Ordeal, to the disbelief of Seswatha.

A Gnostic sorcerer battles a Wracu of Golgotterath.

A year later the Consult went on the offensive. Employing passages reaching under the
Black Furnace Plain and into the Ring Mountains, the Consult launched devastating
assaults into the Ordeal's rear and flanks. Much-reduced by the absence of the Kûniüri
forces, the Ordeal's army almost collapsed. Qûya and Sohonc sorcery allowed at least a
small part of the army to escape, but Nil'giccas was so enraged to learn of the deaths of
at least two of his sons that he recalled the Cûnuroi contingent of the Ordeal
altogether, leaving Aörsi to fight on alone.

In 2133 Dagliash was taken by the Consult, allowing their armies to cross the Sursa in
force. Western Aörsi was overrun and Nimeric withdrew his forces to his capital,
37Shiarau. Celmomas realised his folly and rallied Kûniüri to rejoin the war in 2134, but
it was too late. The Aörsi fleet fled across the Neleöst to seek shelter in the Kûniüri port
of Aesorea, where it was promptly destroyed by enemy agents in the event known as the
Burning of the White Ships.

In 2135 Nimeric took a mortal wound during the Battle of Hamuir, dying soon
afterwards. In the spring of 2136 Shiarau fell, and with it Aörsi itself. Kûniüri stood
alone.

The latter course of the Apocalypse: 6. Ossirish (2137). 7. Shiarau (2137). 8. Dagliash (2139). 9. The
Second Investiture, ending in Initiation (2142-43). 10. The Fields of Eleneöt (2146). 11. Trysë (2147). 12.
Sauglish (2147). 13. Eämnor (2148). 14. The Fords of Tywanrae (2149). 15. Kelmeöl (2150). 16. Inweära
(2151). 17. Kathol Pass (2151). 18. Cil-Aujas (2152). 19. Shir (2153). 20. Sumna (2154). 21. Mehtsonc
(2154). 22. Mengedda (2155).

The situation seemed bleak, but in 2137 Anasûrimbor Nau-Cayûti, Prince of Kûniüri,
won a stunning victory over the Consult at the Battle of Ossirish. The armies of
Kûniüri had been hard-pressed by a Consult offensive, but Nau-Cayûti rallied his men
38by facing and slaughtering the Wracu Tanhafut the Red in direct combat, a feat
undreamt of since the Cûno-Inchoroi Wars. Nau-Cayûti then led the victorious army to
rout the Consult at the ruins of Shiarau, driving the remnants back across the Sursa by
the end of 2138. In 2139 he recaptured Dagliash before launching several major raids
across Agongorea, designed not to reinvest Golgotterath but simply slaughter Sranc and
Bashrags.

In 2140, the Consult abruptly switched tacks and kidnapped Aulisi, the beloved
concubine of Nau-Cayûti, bearing her to Golgotterath. Infuriated, Nau-Cayûti may
have decided on a rash assault (possibly the rationale for the act) but was talked down
by Seswatha. Seswatha proposed something else instead: a raid on the Incû-Holoinas,
such as that undertaken by some of the Nonman heroes of old. Seswatha had studied
among the Nonmen of Ishoriöl, donning the famed (and feared) cauldron-helm
Amiolas to gain knowledge that would permit such a raid to succeed. Many historians
consider the story of the raid that followed as being apocryphal due to sheer
unbelievability, but Seswatha's descendants in the School of Mandate have confirmed
(thanks to their sorcerous ability to relive Seswatha's life) that it is true.

Nau-Cayûti and Seswatha entered the Golden Ark, descending through chambers and
passageways that had supposedly been desolate and empty for thousands of years, since
the Cûnuroi had sacked the vessel from top to bottom. But, deep in the vessel's
cavernous hold, they did find a city of horrors, guarded by Sranc and Bashrags. They
failed to find any trace of Aulisi but they did find something that abruptly changed the
fortunes of the war: Suörgil, the Shining Death, the Heron Spear itself.

``I lied. Because I couldn't succeed, not alone. Because what we do here is more
important than truth or love. We search. We search for the Heron Spear.''
- Seswatha ( The Thousandfold Thought )

They bore the weapon back to Sauglish in great triumph, but this turned sour when
Nau-Cayûti died soon after, allegedly poisoned by his wife Iëva (some say out of jealousy
over Nau-Cayûti's infatuation with Aulisi, and the fear the other woman would
39supplant her). Iëva insisted on Nau-Cayûti being buried rather than burned, as this had
been his wish during life, although many of his family and comrades did not remember
him mentioning such a desire. But out of respect for his widow, they complied. Some
garbled reports from the time state that Consult agents later defiled the grave of Nau-
Cayûti and destroyed or stole his corpse, which seems a dangerous risk to take for petty
vengeance.

The Consult resumed the offensive in 2141, perhaps hoping for a loss of Kûniüri
morale following Nau-Cayûti's death. This hope proved false. General En-Kaujalau
destroyed a Sranc horde at the Battle of Skothera. In 2142, General Sag-Marmau
inflicted a very serious and debilitating defeat on the Consult (according to some
legends, Aurang himself took the field but was forced to withdraw) and again drove
them back to the Ark itself. Anasûrimbor Celmomas II began the Second Investiture in
the fall of that year, unaware that the Consult themselves were now playing for time as,
deep within the halls of the Ark, their millennial plan now reached its culmination.

An artistic interpretation of the Carapace in the depths of the Ark.

Something happened, an event second only to the original Arkfall in importance and
dread.

To this day no-one knows exactly when transpired, save that in the pits of Golgotterath
the Consult finally achieved what they had been attempting to do for a thousand years,
sparking the very warnings that had led to the Ordeal in the first place. They completed
the construction of the Carapace, a sarcophagus of Tekne origin, fused with eleven
Chorae to render it immune to sorcery. Inside the Carapace they created - or unleashed
- an entity of supreme and terrible power. This entity went by many names: Tsurumah
(``Hated One'' in Kyranean), Lokung (``Dead-God'', by the Scylvendi), Mursiris (``Wicked
North'', by the Shiradi) and Cara-Sincurimoi (``Angel of Endless Hunger'', by the
41Nonmen), as well as the Great Ruiner and World-Breaker. But his most famous title
was the one first bestowed upon him: Mog-Pharau in Ancient Kûniüric, ``No-God''.

The No-God first drew breath in the spring of the Year-of-the-Tusk 2143 in the event
known to the Consult as ``Initiation''. The Carapace emerged from the dread Ark,
floating above the ground. Gusts of wind started to form around the Object, quickly
becoming first a gale and then a roaring whirlwind. Upon the instant of Initiation,
every unborn child in the world was stillborn, beginning the horror known as the Years
of the Crib. A ``feeling of dread'' fell across all of humanity, drawing their eyes to the
northern horizon. Sranc, Bashrag and Wracu, including some who had escaped taking
part in the wars so far, were compelled to answer his call and descend on the Black
Furnace Plain and Golgotterath in numbers beyond counting, a horde which blanketed
the horizon.

The host of General Sag-Marmau was destroyed utterly. But the Horde of the No-God
did not march immediately, instead waiting as vast hosts of Sranc gathered and bred.
This gave Kûniüri a very brief space in which to cry for aid. Eärwa answered, the armies
of Ishterebinth marching under Nil'giccas and Kyraneas sending a significant army to
lend aid. Other nations, more distant, began to muster but the distances were too great
and time ran out.

Anasûrimbor Celmomas II led the so-called Second Ordeal into battle against the
Horde of the No-God on the Fields of Eleneöt, which in earlier millennia had been
called Pir Pahal, in 2146. The Horde engulfed the Kûniüri army. Celmomas knew the
only hope was to use the Heron Spear against the No-God. However, although the vast
Whirlwind that symbolised the No-God's presence gathered on the far horizon, the
entity itself refused to give battle, letting its vast army of minions do the work for it.
Celmomas is said to have thrown himself into battle with a rare fury and slain dozens
of enemies, only to be mortally wounded. Seswatha led a rallying force to retrieve the
High King, who lived long enough to impart a prophecy: that an Anasûrimbor would
return at the end of the world. Then he died.

``Did I ever tell you that my son once stole into the deepest pits of Golgotterath? How I
miss him, Seswatha! How I yearn to stand at his side once again. I see him so clearly.
He's taken the sun as his charger, and he rides among us. I see him! Galloping through
the hearts of my people, stirring them to wonder and fury! He says such sweet things to
give me comfort. He says that one of my seed will return, Seswatha. An Anasûrimbor
will return at the end of the world!''

\begin{itemize}
\tightlist
\item
  Anasûrimbor Celmomas II ( The Darkness That Comes Before )
\end{itemize}

Elsewhere on the battlefield, his heir Anasûrimbor Ganrelka outlived him, becoming
the High King of Kûniüri. According to popular legend, Ganrelka also died on the
Eleneöt Fields, but in reality, he survived thanks to four brave Knights of Trysë.
Ganrelka escaped home, gathered his household, and marched west into the Demua
Mountains. In the remotest peaks, protected by both geography and utter secrecy, the
Kûniüri High Kings had built a stronghold and a shelter, Ishuäl. Ganrelka took up
residence of there, but disease followed and wiped out most of the family\ldots{}save for
Ganrelka's bastard son, the last living son of House Anasûrimbor. He and his line fell
out of history for two thousand years.

43The Sarcophagus of the No-God, protected by the ever-present Whirlwind.

By the end of 2147 all of Kûniüri was overrun. The great river-cities of the Aumris
Valley were obliterated: Trysë, Ûmerau and Sauglish itself, with its famed library. The
Nonmen of Ishterebinth retreated over the Demua Mountains to their Mansion. The
Horde of the No-God pursued, laying siege to the Exalted Mansion for two years. The
Mansion shut its famed inner gate, reinforced from material harvested from the Ark in
millennia past, and neither the No-God nor even the fabled Sun Lance of Aurang (the
last functioning Inchoroi spear of light, save the Heron Spear) could gain entry.
Repulsed, the No-God turned south and destroyed Eämnor (although sparing its
capital, Atrithau, due to the complications of attacking a city raised on anarcane
ground and thus immune to sorcery) in 2148. Abandoning the siege of Ishterebinth,
the remainder of the Horde destroyed Akksersia in 2149 (after the Battle of Tywanrae
Fords), and the Meörn Empire collapsed in 2150, despite a hardy defence. Inweära was
cast down in 2151, although the Horde chose to avoid Sakarpus and its vast Chorae
44Hoard to instead rush the Kathol Pass - the gateway to the entire Three Seas - before it
could be fortified.

The Battle of Kathol Pass, fought in the autumn of 2151, was an unexpected victory for
the forces of men. A retreating army of Meöri warriors led by Nostol ran into an
advancing force of Nonmen out of Cil-Aujas, led by King Gin'yursis, a powerful wielder
of the Gnosis. They made common cause and successfully repulsed several waves of
attacks from the Horde on the pass. Shockingly, the Meöri turned on and betrayed the
Nonmen, slaughtering their army and then sacking Cil-Aujas. The reasons for this are
unclear, but may be related to the rising levels of religious fervour amongst the Norsirai
refugees (perhaps hoping that the Hundred Gods would intercede and destroy the No-
God for them), who hoped that by staying true to the teachings of the Tusk -- including
the commandment to destroy the False Men -- they could invoke the protection of the
Gods. It is also possible that the Meöri believed they could use Cil-Aujas as a refuge
should the No-God advance further south. Gin'yursis's death saw him curse the Meöri
for their betrayal, a curse sometimes used to explain the famous fractiousness of the
men of Galeoth (founded by the Meöri descendants), although Gin'yursis's curse had
in fact been directed at all mankind.

During this period, the populous and packed cities of the south cried out for succour
and divine intervention. They prayed to the Hundred Gods, but received no reply. The
people begged their priests to explain why the Hundred had not interceded and the
priests could not answer. Many years later, confused records of this time suggest that
the priests had in fact petitioned for help and gotten only bizarre responses: the Gods
could not see or feel the presence of the No-God, only the destruction that followed in
his wake, which they blamed solely on humanity itself. The Hundred could not
intervene because they could not even perceive the problem in the first place, and it
may be that that this nullification of divine perception is one of the reasons the No-
God was named as such. Humanity stood alone.

In 2153, the Horde of the No-God destroyed the Shiradi Empire at the Battle of
Nurubal, plunging the empire into chaos and collapse. It then turned west to invade
45Kyraneas. Anaxophus, Seswatha's old friend now ruling as King Anaxophus V, led his
nation with skill and cunning. The Scylvendi, the long-established pastoralists living
beyond the mountains to the north-west, had unexpectedly declared for the No-God
and invaded Kyraneas's flank, threatening to trap the kingdom in a vice at the Battle of
Mehsarunath. Anaxophus evaded the trap and escaped to the south. He chose not to
defend either the royal capital at Mehtsonc or the holy city of Sumna (from where the
Holy Tusk was evacuated by sea to Nilnamesh) instead choosing to fight a war of
irritation and attrition, testing the flanks of the No-God's horde and withdrawing when
the enemy attempted to respond.

Kyraneas was effectively overrun and destroyed by the end of 2154. But Anaxophus V
and his army, now aided by Seswatha and the remnants of the Gnostic Schools,
survived. They withdrew through the mountains to the ruined, ancient city of
Mengedda. The city had once been a trading post between Shigek and the cities of the
Kyraneas Plains when the age of man was young, but innumerable battles had been
fought then over the past two thousand years. The blasted landscape and ruins
provided Anaxophus and his army with cover and defences. More importantly, the
long, attritional warfare favoured by Anaxophus had helped reduce the size of the
Horde to one where victory by sheer weight of numbers was no longer certain.

Anaxophus's gamble worked: to ensure victory and the destruction of the last enemy
who might be any threat, the No-God took the field directly, the terrible Whirlwind
moving towards the Kyranean lines and declaring, as it had done all along, ``WHAT
DO YOU SEE?'' This allowed Anaxophus to do what he had been planning ever since
his knights had seized the Heron Spear from the Fields of Eleneöt eleven years
previously: he used the weapon directly against the No-God.

Aided by Seswatha, King Anaxophus V of Kyraneas uses the Heron Spear against the No-God.

As the Apocalypse began in doubt and uncertainty, so it ended with a clear victory. The
Whirlwind burst asunder, the No-God was destroyed and his armies were routed.
According to some reports, the Carapace itself was reduced to ashes, ashes which were
carried by the winds to all the corners of the Three Seas where they caused the Indigo
Plague. However, Mandate scholars insist that the No-God's body (if it could be called
that) and the damaged Carapace were saved by Consult sorcerers and borne back to
Golgotterath.

The end of the war was draped in controversy, for the knowledge that Anaxophus had
stolen the Heron Spear and kept it secret for a decade as the Ancient North and the
Shiradi Empire (Kyraneas's great rival to the east) were overrun and destroyed did not
endear him as the saviour of mankind, as perhaps should have been the case. However,
Anaxophus claimed that the disaster of the Eleneöt Fields had happened because the
Heron Spear had been deployed prematurely before the No-God had engaged, and that
he had no choice but to wait - no matter the cost - for the No-God to show himself
before he could risk using the weapon. This tactical claim has been supported -
although not altogether wholeheartedly - by the Mandate.

Another, more minor debate has also taken place in the centuries since the defeat of
the enemy: due to the No-God's very presence in the World inhibiting procreation,
47more than a few commentators have pointed out that the Consult had no need to go
on the offensive. Instead, the Consult could have simply had the Whirlwind raging in
the shadow of Golgotterath (or, to avoid counter-attacks, in even more remote corners
of the World) and, after a century or so, the human population of Eärwa would have
been extinguished. Mandate commentators have countered by suggesting that either the
No-God could not be entirely controlled once unleashed, or the weapon had a finite
lifespan. No conclusive answer has been provided for this quandary.

The end of the war resulted in the infamous Indigo Plague, which caused great misery
and suffering around the Three Seas, but also in a regrouping of civilisation. Seswatha
gathered together the few surviving Gnostic sorcerers and founded the School of
Mandate, based at the fortress of Atyersus on an island in the middle of the Three Seas.
Seswatha knew that the No-God had been destroyed and the Consult defeated, but the
Inchoroi Princes yet lived, the Consult sorcerers yet survived and the hordes of Sranc
and Bashrags (and even a few surviving Wracu) had only dispersed. But most damning
of all was the prophecy given to Seswatha by his friend and ally Celmomas at the
moment of his death:

``An Anasûrimbor will return at the end of the world!'' - The last words and prophecy of
Anasûrimbor Celmomas II ( The Darkness That Comes Before )

The First Apocalypse was over. Now the Mandate had to prepare humanity for the Second.

\hypertarget{part-4-the-age-of-warring-states}{%
\chapter{Part 4: The Age of Warring States}\label{part-4-the-age-of-warring-states}}

The Apocalypse destroyed the civilisation of the Ancient North. Two great cities,
Atrithau and Sakarpus, had survived but otherwise all of the glories of the Norsirai had
been lost and the surviving remnants of that once-great people pushed south into the
Three Seas. Attempts to found new cities and settlements in the north foundered
under vast numbers of Sranc. Leaderless and without direction, they continued
breeding, raiding and rampaging. With a truly vast amount of terrain to range freely
across, almost the entire northern half of the continent, their breeding could not be
controlled and within a few centuries their numbers were beyond calculation.
Fortunately, they showed no appetite for a concerted push into the Three Seas.

The borders of the Ceneian Empire after each major conquest: Gielgath (2349), Cepalor (c. 2390),
Shigek (2397), Xerash and Amoteu (2414), Nilnamesh (2483), Cingulat (2484), Amarah (2485), Cironj
(2508), Nron (2511), Ainon (2518), Cengemis (2519) and Annand (2525). The latter conquests were
carried out by Triamis I, the Great, the first Aspect-Emperor of the Three Seas (2456-2577).

The fall of the No-God at the battlefield of Mengedda spelt the end of the Consult's
plan to destroy the world, but not the Consult themselves. They retreated - according to
some, taking the No-God's Carapace with them - and sought refuge in Golgotterath.

49With initially thousands and later millions of Sranc infesting all the lands between the
Three Seas and the Yimaleti Mountains, the victorious armies of Kyraneas and their
sorcerous allies were unable to pursue. The ravages of the Indigo Plague of 2157 soon
exhausted what was left of Ketyai strength, already pushed to breaking point by decades
of warfare and accompanied by the death of Anaxophus V shortly after the end of the
war, led to the collapse of Kyraneas.

Seswatha survived the Apocalypse, in fact living until 2168 when he died at the age of
79. Shortly after the end of the Apocalypse, with the School of Sehonc effectively
destroyed, he founded the Gnostic School of Mandate, based in the fortress of Atyersus
on the island of Nron. A year later he founded Attrempus on the mainland to the
north-east. Fearing that his successors would forget the lessons of the Apocalypse,
Seswatha underwent a sorcerous ritual upon his death. His heart was extracted from his
body and placed in a chamber in Atyersus. Every Mandate schoolman, upon joining the
organisation, would undergo a ritual known as the Grasping. This ritual would transfer
Seswatha's memories to him. Every night he would dream the details of Seswatha's life,
the great battles, the descent into Golgotterath, the preparations for the Apocalypse
and the final battle with the No-God. In this way, the knowledge and fear of the
Consult would live on. The Mandate scoured the Three Seas searching for Consult
agents, occasionally exposing and destroying them. But for the most part the Consult
seemed willing to remain in hiding in far Golgotterath.

Although the Ancient North and the northern Three Seas had been ravaged by the No-
God, the southern nations remained untouched by the war. Amoteu, Shigek and
Nilnamesh soon proved resurgent and the refounded city-states of the Kyranae Plains
fell into internecine warfare, beginning the Age of Warring Cities (lasting
approximately from 2158 to 2477 Year-of-the-Tusk). This period may well have seen a
relapse into barbarism had not humanity found a new saviour.

Inri Sejenus, known to history as the Latter Prophet, was born in 2159. At a young age,
he claimed to be the pure incarnation of the Absolute Spirit (``the very proportion of
the God'') and to have been sent to amend the teachings of the Tusk. He argued for a
50fairer world and a willingness to embrace God in His singular aspect as well as that of
the Hundred. The extant Kiünnat sects at first dismissed Sejenus as a fringe
philosopher, but as he got older he attracted vast followings. His teachings were widely
disseminated and his popularity boomed. In 2198 Sejenus was arrested and sentenced
to death by King Shikol of Xerash. In 2202 the execution was carried out and Sejenus
was put to death outside the city of Shimeh, in what had been Amoteu (at that point a
subservient nation to Xerash). However, the King himself then died and it was said by
the faithful the Sejenus returned to life and ascended to the Nail of Heaven from the
temple at Kyudea, outside Shimeh (alternate stories say it was from Shimeh itself).

Sejenus's movement, Inrithism, slowly spread throughout the Three Seas. It was fought
against by the Kiünnat cults, but soon become irresistible. A framework was set up that
disseminated the teachings of Sejenus, The Tractate , through sub-temples worshipping
the Hundred. This became the Thousand Temples, with a single leader, the Shriah, at
its head. Although Shimeh was the holiest city in Inrithism for the martyrdom of
Sejenus, the presence of the Tusk led to the religion basing itself in Sumna (to where
the Tusk had been returned following the No-God's defeat), which capitulated in 2469.
In 2505, the religion gained official recognition as the state religion of the Ceneian
Empire, which by that point had become the pre-eminent power of Eärwa.

Cenei had been founded over a thousand years earlier, but had spent most of its
existence as a modest river town on the Phayus, the greatest river of the fertile Kyranae
Plains. The destruction of Mehtsonc during the Apocalypse had been carried out with
such thoroughness that the ruins were deemed uninhabitable, and over successive
generations downriver Cenei instead absorbed a lot of the returning refugees. The city
grew in size and power, and when the Age of Warring Cities began it was well-placed to
fight both defensively and offensively. In 2349, it captured Gielgath, at the mouth of
the Shaul, effectively giving it control over the intervening southern Kyranae Plains.
Xercallas II completed the reconquest of Kyraneas and his successors conquered
Cepalor in the north (inhabited by the descendants of Norsirai refugees) and then
Shigek to the south by 2397.

The root of Ceneian success was the Imperial Army, which was thoroughly well-trained
and formidably equipped. The organisation of the army, its ability to absorb recruits
from newly-conquered provinces and its willingness to change tactics resulted in a
military force arguably unmatched before or since in Eärwa. The Imperial Navy was
likewise impressively-organised. Between 2397 and 2414 the two institutions would
combine to outflank the Carathay Desert and deliver a series of raids and then
conquests in Enathpaneah, Xerash and Amoteu, capturing the Holy City of Shimeh
along the way. General Naxentas, who delivered this stunning victory, declared himself
the first Emperor of Cenei. He would be assassinated within the year, but his successors
built on his achievements.

Triamis I became Emperor in 2478, beginning the Ceneian Golden Age. In 2483, he
conquered Nilnamesh, defeating King Sarnagiri V. The following year he invaded
Cingulat, on the far north-western coast of the continent of Kutnarmu. Triamis turned
west, leading his armies to the borders of Zeüm, the great Satyothi power of far western
Eärwa which had succeeded ancient Angka. He defeated a mighty host at the Battle of
Amarah and would have invaded but his homesick troops mutinied. He returned to
Cenei and consolidated his gains.

Returning home, he found the empire caught in a religious conflict between the
Kiünnat cults and Inrithism, which was threatening to spill over into outright war.
Triamis spoke to leaders on both sides, but found that Ekyannus III, Shriah of the
Thousand Temples, was both more reasonable and convincing as a religious leader. In
2502 Ekyannus instituted the ``Emperor Cult'' of the Thousand Temples and dubbed
Triamis the Aspect-Emperor of the Three Seas. In 2505 Triamis himself converted to
Inrithism, naming it the state religion of the Ceneian Empire. He then spent ten years
putting down religious rebellions whilst also concluding the conquests of the island
nations of Cironj (2508) and Nron (2511). Shortly afterwards he invaded the eastern
Three Seas, conquering the successor-nations of the old Shiradi Empire: Ainon (2518),
Cengemis (2519) and Annand (2525). For his achievements in conquering almost the
entire Three Seas, Triamis was dubbed ``The Great''.

Eärwa circa 3000 Year-of-the-Tusk, at the height of the Ceneian Empire. The Empire dominated the
continent for eight hundred years prior to its collapse in the 34 th Century. It was the largest and most
powerful nation-state in history, unrivalled until the rise of the New Empire of Anasûrimbor Kellhus a
thousand years later.

The following Aspect-Emperors would maintain the borders of the empire, keeping the
Ceneian Empire as the centre of political, military and religious power in Eärwa for
eight centuries. The weakness of the Ceneian Empire was not in its military strength,
but in its political succession, with brief but bloody civil wars often being the
mechanism for a transference of power. The constant instability eventually resulted in
the Empire growing lax and overconfident. In 3351 Cenei was sacked by the Scylvendi
under Horiötha King-of-Tribes, triggering the collapse of the empire. The destruction of
Cenei was brutal, with the city burned to the ground and all of its treasures, including
the Heron Spear, lost or stolen. The great fortress of Batathent was destroyed shortly
afterwards.

The final collapse is generally dated to 3372, when General Maurelta surrendered his
legions to Sarothesser I. Sarothesser had led the south-eastern part of the empire in
breaking away from Cenei. In this year, he ascended the Assurkamp Throne in
53Carythusal as the King of High Ainon. Cengemis and Nilnamesh also broke away,
spelling the end of the Ceneian Age. In 3374 Aöknyssus became the capital of a new
nation, Conriya.

By 3411 the port city of Momemn at the mouth of the Phaysus, had supplanted lost
Cenei as the pre-eminent city of the Kyranae Plains. Under the Trimus Dynasty
Momemn became the capital of Nansur, first a small kingdom and then (under the
succeeding Zerxei and Surmanate dynasties) a mighty empire, proclaiming itself the heir
to both Kyraneas and Cenei. By 3619 the Nansur Empire had conquered Shigek and
Amoteu but failed to expand those conquests into the eastern Three Seas, where the
power of High Ainon was unassailable. Later in the century Nansur and High Ainon
formed a brief military pact, perhaps planning to carve up the Three Seas between
them, but ultimately this idea foundered and the pact dissolved. In 3643 Norsirai
tribesmen living north-east of Nansur consolidated into the kingdom of Galeoth,
followed in 3742 by the founding of Ce Tydonn, which supplanted and replaced
Cengemis. In 3787 the Thunyeri, a robust warrior-people descended from the ancient
Meöri Empire, were displaced by growing numbers of Sranc from the lands south of the
Sea of Cerish, moving down the Wernma River and becoming raiders and pirates
which would trouble the Three Seas for two centuries before they consolidated as the
kingdom of Thunyerus in 3987.

The ambitions of Nansur to once again seize control of the Three Seas were thwarted
by a series of events along the fringes of the Great Carathay Desert. Fane, an Inrithi
priest living in Eumarna, was found guilty of heresy by the Thousand Temples in 3703
and cast into the Carathay Desert to die. Fane went blind in the desert, but also
experienced a series of religious insights and revelations. He emerged from the
southern sands wielding a power known as the Water of Indara, a form of sorcery both
unknown and alien to the Schools of the Three Seas. The Kianene, the raiders and
tribesfolk of the Great Salt, welcomed him amongst their ranks and listened to his
teachings. Fan'oukarji I, Fane's son, took those teachings and translated them into a
holy mission to destroy Inrithism, with the ultimate goal of casting down the Tusk (the
``Cursed Thorn'' in their tradition).

54A Cishaurim sorcerer wielding the Water of Indara. Although still vulnerable to Chorae, Cishaurim are
not Marked as other sorcerers are and their presence cannot be felt by others of the Few. The reasons for
this remain unknown to the sorcerous schools of the Three Seas. The Water of Indara is believed to be
more powerful than most of the anagogic sorcery of the Three Seas, checked only by the Gnosis of the
Mandate.

The Kianene swept out of the desert in the so-called White Jihad (3743-71). The
Kianene armies were supported by followers of Fane who had also cast out their eyes
and gained the powers of the Water. They became known as the Cishaurim. As proof
of their righteousness, the Cishaurim showed that, unlike followers of the sorcerous
schools like the Mandate and the Scarlet Spires, they were not Marked. According to
some, they were not damned to an eternity of torment as other sorcerers were (although
this remains highly debated amongst some commentators). Their presence could not be
felt by other sorcerers, but Chorae were still anathema to them.

By 3771 the Kianene had conquered Mongilea and large portions of Eumarna,
founding a new capital at Nenciphon on the River Sweki, and converted the Girgashi
people of the desert to Fanimry. Kian had emerged as a powerful new player on the
shores of the Three Seas, although not one yet taken seriously by the Nansur Empire or
the Thousand Temples. In 3798 the Shriah, Ekyannus XIV, ordered the extermination
of the sorcerous schoolmen, declaring them to be unclear abominations. The Scholastic
Wars raged for the next eighteen years and saw several lesser schools destroyed.
However, it also provided the impetus for the Scarlet Spires to seize control of High
55Ainon, bringing the might of one of the great powers of the Three Seas under their
control. The Mandate survived a ten-year blockade and siege of Atyersus, although it
curtailed its mainland activities, and the Mysunsai ``mercenary'' school came into
existence. By 3818 the pogrom had been called off, but many sorcerers throughout the
northern Three Seas had lost their lives.

This, of course, reduced the ability of the Thousand Temples and the Nansur Empire
to resist the onslaught of the Kian and their Cishaurim. The rest of Eumarna fell in
3801, followed by Enathpaneah in 3842 and Xerash and Amoteu by 3845. The Sack of
Shimeh outraged both the Thousand Temples and all followers of Inrithism as a whole,
but there was no appetite for a counter-assault. The Kianene maintained the initiative.
In 3933 the Dagger Jihad of Fan'oukarji III saw both Shigek and Gedea fall to the
Kianene, bringing the borders of Kian to the very doorstep of Nansur. In the resulting
turmoil, the Surmante Dynasty was destroyed and replaced by the Ikurei family. The
Ikurei then reorganised the Nansur army and were able to defeat no less than three
Kianene invasions of the empire over the next several decades.

Meanwhile, the Mandate were facing mixed fortunes. Early in the 3900s they lost track
of the last Consult agents in the Three Seas. For three centuries, they scoured the lands
for any sign of the enemy, only to find that they had completely disappeared. This
disconcerting event was accompanied by a more positive one: House Nersei of Conriya
forged a strong alliance with the Mandate, accepting their schoolmen as tutors and
advisors. The Nersei dynasty used this advice to shore up their political support and
eventually take the throne of the kingdom. The Mandate also gifted their secondary
fortress of Attrempus to the Nersei family, giving them a strong bulwark to use against
possible attack.

Towards the end of the 41st Century the Three Seas were posed on a knife's edge.
Nansur had checked the advance of the Kianene Empire, but was unable to mount an
effective counter-offensive. The nations of the eastern Three Seas schemed as usual and
the sorcerous schools intrigued. The Mandate kept a watchful eye for the Consult, but
could find no trace of them anywhere.

56The beginnings of the road that led to the Holy War and the new Great Ordeal were
modest. In 4079 the Scylvendi leader of the Utemot tribe, Skiötha urs Hannut, died.
He was succeeded by his son, Cnaiür urs Skiötha, a warrior of tremendous repute for
savagery and intelligence. Cnaiür was advised by a strange man from the Ancient
North, from the lands beyond Atrithau which were believed to be completely lost to
Sranc. This man convinced Cnaiür to kill his father, having seduced his mother.
Afterwards he vanished into the southern deserts, to Cnaiür's fury and declarations of
vengeance.

This man was named Anasûrimbor Moënghus.

Eärwa circa 4109 Year-of-the-Tusk, on the eve of the Holy War and the arrival of Anasûrimbor Kellhus in the Three Seas.

\hypertarget{part-5-the-holy-war}{%
\chapter{Part 5: The Holy War}\label{part-5-the-holy-war}}

After the great disaster at Eleneöt Fields and the resulting fall of Kûniüri, the house of
Anasûrimbor was presumed destroyed and its line extinguished. But this was not so.
Anasûrimbor Ganrelka survived the disaster and managed to escape to Trysë. There he
gathered his household and retreated to Ishuäl, the stronghold that Celmomas II had
constructed high in the Demua Mountains as a last redoubt. After their arrival, a
sickness spread through the refugees and killed them all, one by one, until only
Ganrelka's bastard son and his court poet, a man of dubious repute, survived. The poet
was hurled to his death from the ramparts of the fortress by the young boy, but the
prospects for his survival were bleak. Only the arrival of more survivors saved him.

These survivors called themselves the Dûnyain. Their true origins are unknown, but
theorised to lie in the ascetic sects that arose across the Ancient North prior to the
Apocalypse, prioritising reason and intellect ahead of passion, sentiment or emotion.
They believed that true volition and control -- a union with the Absolute -- could only
come through the Logos, or reason unmarred by sentiment, and the ability to adapt to
circumstances rather than reacting to them or clinging to false precepts out of ideology
or obstinacy. Their primary belief was that if a person can master ``what comes before'',
they can control and predict all the outcomes that follow. Before the Apocalypse they
were, reluctantly, part of the world and its problems. But, fleeing the shadow of the No-
God, they stumbled across Ishuäl. Its utter isolation gave them a chance to fulfil what
they saw as their destiny.

Ishuäl, stronghold of the Dûnyain, hidden in an isolated valley of the Demua Mountains.

The Dûnyain and Ishuäl fell out of history for almost two thousand years. Left alone in
the high peaks, they continued to develop their skills of reading faces and voices and
developing the skills of pure reason. Things may have stayed that way, but in the 4070s
Year-of-the-Tusk they were discovered by a roving band of Sranc, unusually driven into
the high peaks. The Dûnyain destroyed these creatures, not knowing what they were
(having lost their own history along the way). Concerned that Ishuäl's location had
been compromised, they selected one of their number to go out into the world and
investigate. They chose Anasûrimbor Moënghus.

Moënghus's exploration of the outside world confirmed that Ishuäl remained safe, and
that the lands were filled with these ravaging Sranc for hundreds of miles in all
directions bar to the south, where a city of men known as Atrithau lay at the feet of the
Demua Mountains. The Dûnyain were satisfied that they were secure, but concerned
that Moënghus had been ``polluted'' by his contact with the outside world. He was
accordingly sentenced to die.

Moënghus, however, survived the execution order and fled into exile. He travelled
south, past Atrithau and across the Sranc-infested lands of Suskara to reach the Jiünati
Steppe. There, after surviving capture by a Sranc band (itself a remarkable fact), he was
in turn captured by the Utemot tribe of the Scylvendi and forced into servitude in the
household of Skiötha urs Hannut. Moënghus had soon seduced Skiötha's wife and
59turned his son Cnaiür against him. Cnaiür murdered his father, securing his leadership
of the Utemot, but Moënghus soon departed, disguised as a Scylvendi warrior. Cnaiür,
realising the depth of his betrayal, became enraged and vowed vengeance. Moënghus
passed south into the Kianene Empire -- having scarred his arms and dyed his hair to
survive the Scylvendi lands but also making him unable to pass into the Nansurium -
but was again taken prisoner, this time to be sold into slavery.

Anasûrimbor Moënghus after blinding himself and becoming a Cisharuim, known as Mallahet. His lack of emotions and passion meant that he was unable to fully master the Water of Indara.

Again, Moënghus rose to a position of power and influence. He travelled to the Holy
City of Shimeh to learn the Psûkhe, the ways of channelling the Water of Indara, but
even after putting out his eyes he discovered that he was unable to use more than a
trickle of sorcery. Too late, he realised that the Psûkhe relied on passion to empower it,
the very trait the Dûnyain had bred out of themselves. However, in this process of
trying to master the Water he also trained his other senses to compensate. During a
discussion with one man, he noted many discrepancies in his manner of speech that
60could not be explained by simple lying or emotion. He subjected this man to torture
and eventually discovered the truth: the man was a ``skin-spy'', a creation of the Consult.
For three centuries, since the last Consult agent was publicly slain by the Mandate, the
Consult had infiltrated these creatures into positions of power across the Three Seas as
part of their plan to bring about the Second Apocalypse. Moënghus learned from the
skin-spy that the Consult believed they could resurrect the No-God and unleash the
end of the world in a matter of decades, at best.

Moënghus determined that the destruction of the world would not be an optimal
outcome for the Dûnyain struggling to master the Logos and it should be prevented.
He sent in motion a multi-pronged plan to this end. He had sired a number of children
by world-born women but all but one of them had shown significant defects,
abnormalities and mutations (the Dûnyain's multi-millennial breeding efforts had
rendered them mostly incompatible with the rest of humanity). He had them all put to
death apart from the most promising: Maithanet. Although not a true Dûnyain,
Maithanet was intelligent and canny with considerable skills at manipulation, taught by
his father. Moënghus ordered him to enter the Thousand Temples in Sumna and rise
to a position of power and influence. Maithanet complied, within a matter of years
rising far through the religious orders thanks to his intellect, reason and expertly-
feigned religious fervour.

For three centuries, the Consult infiltrated the Three Seas through the use of their skin-spies, such as this one taking the appearance of Esmenet of Sumna. Undetectable to sorcery or mundane scrutiny, they
were defeated by Dûnyain analytic conditioning, which allowed them to be identified through
inconsistencies in their speech, as well as their skin and bone structure.

As Maithanet rose high in the ranks of the Temples, Moënghus used the little of the
Psûkhe he had mastered to send a Cant of Calling to Ishuäl. Speaking to the Dûnyain
in their dreams, he demanded that they send his son, Kellhus, to his side. The Dûnyain
debated and decided that Moënghus had gone insane and was a danger to the security
of their order. Anasûrimbor Kellhus was dispatched with a simple mission: to find and
kill Moënghus. The Dûnyain knew he dwelt in a distant city called Shimeh, but
nothing beyond that.

In the rest of the Three Seas, controversy had arisen around the rise of the hitherto
unknown Maithanet to the rank of Shriah of the Thousand Temples. Maithanet
exposed and defeated three plans to assassinate him, and using his considerable
charisma and power he soon had the fractious religious leaders of Inrithism unified as
they had not been in centuries. The leaders of the Mandate learned that Maithanet
planned to announce a Holy War, but they feared this would be directed against the
sorcerous schools. They ordered one of their number, a worldly agent called Drusas
Achamian, to travel to Sumna to investigate further.

62In Sumna Achamian reunited with his lover, a prostitute named Esmenet, as well as
several political allies and a former student-turned-informant, Inrau. Inrau was soon
murdered by agents of the Consult, sparking Achamian's paranoia that the Consult
survived and yet remained a threat. The Three Seas awaited the news of the Holy War's
target and all were relieved to learn that it was to be directed against the heathen Fanim
of Kian. As thousands of warriors from across the northern and eastern Three Seas
converged on the Nansur Empire, which guarded the frontier with Kian, a shocking
alliance was announced: the Thousand Temples had forged an agreement with the
Scarlet Spires, the sorcerous rulers of High Ainon, to provide a counterbalance to the
Fanim Cishaurim (with whom the Spires had been fighting a secret war for a dozen
years). These great events saw Achamian ordered to accompany the Holy War and spy
on it for the Mandate.

Meanwhile, the Nansur Empire had instigated a military confrontation with the
Scylvendi. At the Battle of Kiyuth, early in 4110 Year-of-the-Tusk, a Nansur army under
Exalt-General Ikurei Conphas, the nephew and heir to the Emperor, defeated a
significant Scylvendi army under the overconfident King-of-Tribes, Xunnurit. The
defeat was unprecedented, the Scylvendi driven from the field in disarray with
tremendous loss of life and Xunnurit taken in chains back to Momemn. Among those
forced to flee the battlefield was Cnaiür urs Skiötha. In the years since his betrayal by
Anasûrimbor Moënghus, Cnaiür had become the chieftain of the Utemot, famed for
his both his savage intelligence and his unrelenting skill at war, the self-declared ``Most
Violent of All Men''. This claim was supported by his extensive swazond, the ritual scars
Scylvendi carved into their own flesh to celebrate kills in battle.

In the aftermath of Kiyuth, Cnaiür visited the graves of his ancestors only to find a
wounded man of the north, surrounded by hordes of dead Sranc. Helping him heal, he
learned that this man was Anasûrimbor Kellhus, travelling south to kill his father, the
hated Moënghus. Cnaiür decided to travel with Kellhus to help him achieve this goal.
Crossing the steppe and approaching the Nansur border, they slaughtered a band of
Scylvendi slavers and freed a young woman named Serwë. Serwë revealed that the
63armies of the Three Seas were gathering around Momemn, the capital of the Nansur
Empire, in preparation for the gruelling march on Shimeh, eight hundred miles or
more to the south.

The approximate route of the Holy War.

The Holy War gathered its strength, tens of thousands of soldiers marching from
Galeoth and Thunyerus, Ce Tydonn and Conriya, High Ainon and the Nansurium
itself. However, the Nansur Emperor, Ikurei Xerius III, saw a chance to manipulate the
Holy War to his own purpose. He agreed to provide the Holy War with his armies, the
support of his sorcerous school, the Imperial Saik, and the leadership of his famed
general Ikurei Conphas, in return that the lands conquered by the Holy War should be
64returned to Nansur control, as the heir to Cenei. This demand proved incompatible
with the notion of a Holy War fought for one religious purpose, with thousands of
troops from other nations potentially slaughtered for the gain of the Emperor in
Momemn. As the debate raged, the first contingent of the arriving armies decided to
march on Shimeh immediately rather than wait for the rest of the host to assemble.
The so-called Vulgar Holy War was destroyed at the Fourth Battle of Mengedda, the
heads of its leaders sent back to Momemn. Xerius attempted to use this knowledge to
press home the need for Nansur leadership in the war to come.

During this controversy, Cnaiür, Kellhus and Serwë arrived at the city, as well as Drusas
Achamian, who had attached himself to the retinue of Krijates Xinemus, the Marshal
of Attrempus. In his youth, Achamian had served as tutor to Crown Prince Nersei
Proyas of Conriya. Although Proyas loved Achamian, he became a devoted follower of
Inrithism and severed his ties with the schoolman, whom he considered damned.
Although Proyas refused to talk to Achamian as their retinue marched on Momemn, he
permitted Achamian to travel with them under Xinemus's parole. Achamian and Proyas
meet the three strangers from the north and Proyas saw an opportunity to outflank the
Emperor's unreasonable demands.

Meanwhile, in Sumna, Esmenet was visited by a man who somehow bewitched and
seduced her to gain intelligence on Achamian's activities. Horrified and believing that
the man may be linked to the Consult, Esmenet travelled to Momemn to try to find
Achamian. Along the way, she was almost stoned to death in an Nansur village for
bearing the caste-mark of a prostitute, but was saved by a Shrial Knight named
Sarcellus. He offered her protection on the road to Momemn.

In Momemn the leaders of the Holy War gathered to discuss the situation. To
everybody's shock, Nersei Proyas proposed that the Holy War accept Cnaiür as their
battle commander. Cnaiür had helped engineer a great Scylvendi victory over the
Kianene at the Battle of Zirkirta several years earlier and knew the ways of their enemy.
Cnaiür also unexpectedly acquitted himself well in a verbal battle of wits with Ikurei
Conphas. Kellhus, posing as a Prince of Atrithau who had foreseen the Holy War in his
65dreams, offered a reasoned analysis of the situation which cut to the heart of the
matter. The assembled nobles agreed to accept Cnaiür as their commander and the
delegate of the Holy Shriah commanded the Emperor to provision as the Holy War as
required under religious order. Outmanoeuvred, the Emperor was forced to stand
down. So as not to appear petty, he also allowed the imperial forces to join the Holy
War. However, during the meeting Kellhus visually identified one of the delegates,
Skeaös, as having something wrong with his face. The Emperor noted Kellhus's interest
and had Skeaös seized and interrogated. In this way, the Emperor came to learn of the
existence of the skin-spies, and that the mad old stories of the Mandate may have some
truth to them.

The Holy War marched from the Nansur Empire, crossing the mountainous frontier
with the northernmost Kianene province, Gedea. However, attempts to delay the
march to allow consolidation of the main army with late-arriving elements met with
disapproval from the leading forces, most notably Prince Coithus Saubon and his
headstrong forces from Galeoth. On the advice of Kellhus, whose intelligence, analysis
and prophetic dreams were the talk of the army, Saubon marched and secured an early
victory at Mengedda. This battle was hard-fought, with many of the Shrial Knights slain
and the Kianene only withdrawing once the bulk of the rest of the Holy War arrived,
but the victory enhanced Saubon's position and made him more trusting of Kellhus.

The rest of the Holy War consolidated. Esmenet was reunited with Achamian, who, in
violation of Mandate law, declared her his wife, but Kellhus immediately identified
Sarcellus as a skin-spy. He chose not to give this information away, knowing it risked
exposing himself as well.

Gedea and the northern half of Shigek fell to the Holy War, the Kianene armies
retreating south of the Sempis. Kellhus gave a series of sermons under the famed
Ziggurats of Shigek which attracted thousands of listeners. More than a few of the army
began to refer to him as the Warrior-Prophet.

Drusas Achamian, a sorcerer of the Mandate, turns the Gnosis against his captors from the Scarlet Spires.

Achamian taught Kellhus in the ways of the world, finding him a quick and formidable
study in history, mathematics and philosophy. Achamian soon discovered that Kellhus
was one of the Few and could use sorcery, but refused to betray his school by teaching
him the Gnosis. Torn by his respect for Kellhus, his desire for Esmenet and his loyalty
to the Mandate, Achamian sought solitude to think things through, but was captured
by the Scarlet Spires. The Spires had long desired mastery of the Gnosis, which eclipsed
their own sorcery, and had now learned of the existence of the skin-spies. Eleäzaras, the
master of the Spires, put Achamian to the question, even blinding his friend Xinemus
to try to get Achamian to talk. However, Seswatha's gift also rendered Mandate
sorcerers immune to torture. Achamian was eventually able to escape and turn the full
might of the Gnosis upon his captors.

The Holy War marched on without Achamian (an absence that caused Esmenet great
distress), crossing the Sempis Delta and fighting a major battle under the fortress walls
of Anwurat. Despite heavy losses the Men of the Tusk prevailed and marched on into
Khemema. This was the most dangerous part of the journey, as Khemema was where
the Great Carathay Desert met the Meneanor Sea. No food grew there and no water
could be found. The Holy War had to brave the desert coastlands southwards for
almost two hundred miles. To survive the crossing the fleet had to be resupplied with
food and water by the Imperial Nansur navy. But the Kianene Padirajah had
67anticipated this move and deployed the Kianene fleet to intercept. In a great battle in
Trantis Bay, the Nansur fleet was defeated and put to rout by the Fanim. The Holy War
was cut off from succour and left to die in the burning wastes.

But the Holy War survived. Anasûrimbor Kellhus found great reserves of water far
below the desert sands and the army was saved, although much-reduced. The army
burst from the desert and besieged the great, ancient mercantile city of Caraskand.
Although ravaged by disease and starvation, the Holy War was able to take the city,
helped by treachery, and sacked it savagely. No sooner was this done, however, than the
Padirajah himself took the field. Kascamandri ab Tepherokar led a vast army out of the
south to besiege Caraskand and starve the Men of the Tusk into surrender or death.

Around this time Kellhus received a message from his father, borne by a Cishaurim.
Moënghus told Kellhus that soon he would grasp the Thousandfold Thought. The
nature of this concept eluded Kellhus, save it was an extension of the Dûnyain method
of foretelling future events through probability trances, predicting the future by
mastering what comes before. He was forced to execute the Cishaurim to maintain his
cover before he could learn more.

The Holy War had become torn between traditional Inrithi, led by Sarcellus and Ikurei
Conphas, and those who revered Kellhus as the Warrior-Prophet. The former became
known as the Orthodox and the latter, led by Nersei Proyas and Coithus Saubon but
with Esmenet placed high in their ranks, as the Zaudunyani , the Tribe of Truth. The
tensions between the two sides rose and resulted in a failed assassination attempt on
Kellhus and a failed counter-assassination attempt on Sarcellus and Conphas. The
chaos finally resulted in a trial. Sarcellus and Conphas won this trial and had Kellhus
denounced as a false prophet. Serwë, whom Kellhus had taken as wife, was executed
and her body was tied to Kellhus, who was then hung upside down from a tree on a
massive iron ring, a circumfix. Achamian returned at this point, learning that Esmenet
was pregnant by Kellhus (and that Serwë has borne a son, named Moënghus for
Kellhus's father, given to Esmenet to raise). Furious, he confronted the dying Kellhus
only to be told that many skin-spies had infiltrated the Holy War and only Kellhus
68could identify them. Reluctantly, Achamian begged for Kellhus's release but was
rebuffed by Ikurei Conphas.

But in this moment Cnaiür exposed Sarcellus as a Consult skin-spy by defeating him in
battle and severing his head. This causes the creature's face to return to its normal
appearance, to the horror of the witnesses. The Holy War repented, lowing Kellhus
from the Circumfix to find that he had survived. However, during his ordeal Kellhus
had nearly been broken, weeping and having visions of the Apocalypse, including
hearing the voice of the No-God. He recovered swiftly and produced a miracle: he
pulled forth Serwë's heart from his own chest.

The Warrior-Prophet, now hailed as something more than a man, led the Men of the
Tusk from Caraskand in a direct assault on the Padirajah's army and, despite their
starved frames and lesser numbers, defeated it, with Kellhus himself killing
Kascamandri. Fanayal, Kascamandri's son, was declared the new Padirajah and fled the
field with as many surviving Kianene forces as possible. The Holy War had triumphed
and the road to Shimeh lay open.

At this time, the Consult descended upon the Ancient North. From the Neleöst to the
Cerish Sea and beyond hordes of Sranc suddenly acted with purpose, turning on
remote tribes of men who had survived - or been allowed to survive - on the Plains of
Gâl and the Istyuli Plains. Caravans daring the great crossing from Sakarpus to Atrithau
were taken prisoner and everywhere one question was asked, again and again: ``Who are
the Dûnyain?''

Rested and, to an extent, resupplied, the Holy War issued forth from Caraskand and
marched south, though ancient Xerash and Amoteu. Kellhus, now universally accepted
as the Warrior-Prophet, had grasped what his father called the Thousandfold Thought:
a web of probability and consequence designed to defeat the Consult and halt the
resurrection of the No-God and the destruction of the world. Kellhus again asked
Achamian to teach him the Gnosis and this time Achamian complied. Kellhus told
Achamian that the time for sorcerers to be hated and feared and damned was over, and
69that Kellhus would declare their damnation to be at an end. In addition, Cnaiür's
public revelation of the skin-spy Sarcellus, Achamian's relating of the Celmomian
Prophecy (confirming that an Anasûrimbor would return at the end of the world) and
the awe that Kellhus was now held in combined to convince the Men of the Tusk that
the ancient stories were true: the Consult was real and working to bring about the
return of the No-God. The Mandate overnight were transformed from a joke to
prophets and guardians standing against the Second Apocalypse.

Meanwhile, Cnaiür was given the task by Kellhus of arranging the death of Ikurei
Conphas, whose mad dreams of becoming emperor and bringing about the rebirth of
Cenei and Kyraneas now posed a threat to the Warrior-Prophet. The deed was to be
done in the port city of Joktha, but Conphas turned the trap on Cnaiür and almost
killed him. The Scylvendi was rescued by a detachment of Consult skin-spies, eager to
win the allegiance of one of their former minions (the Scylvendi having fought for the
Consult in the Apocalypse).

Meppa, the most powerful Cishaurim to survive the Holy War.

The Holy War marched on. Mighty Gerotha, capital of Xerash, fell. To avoid a brutal
sacking the masters of the city were commanded to killed four-tenths of the city's
70population. Twenty thousand were put to the sword to appease the Men of the Tusk.
This example spread ahead of the Holy War and cities and fortresses and towns the
length of Xerash and Amoteu threw open their gates to avoid the same fate. Fanayal's
forces skirmished with the Holy War, eventually destroying their main scout formation,
but ultimately had to fall back on Holy Shimeh, leaving the way open for one last push
by the Holy War.

And at that moment, the Inchoroi came before Anasûrimbor Kellhus.

Using a vessel known as a Synthese, taking the form of his lover Esmenet, Aurang
spoke to Kellhus, trying to divine the nature of the Dûnyain and that of Kellhus
himself. Instead, it gave up more of itself and its goals. Kellhus learned that the
Inchoroi considered themselves a race of lovers, consumed by appetites of the flesh.
This was their nature and they were damned for it by the metaphysics of the universe,
condemned to roil and burn for all eternity in scouring fire. To avoid this fate the
Inchoroi had to destroy the people of the world. By slaughtering the population of the
world and bringing about the return of the No-God, the Inchoroi would seal shut the
world from the Outside, barring it from the view and the judgement of the gods. Only
then could the Inchoroi die, satisfied that they would not suffer eternal damnation as
their slain kin and as sorcerers still did.

Kellhus gave little in return, but told Aurang that the No-God spoke to him in his
dreams, that Mog-Pharau blamed the Inchoroi and the Consult for his defeat on the
plains of Mengedda and he would have his revenge.

The confrontation yielded little useful intelligence for the Consult, but it served to
distract Kellhus whilst an attempt was made on Achamian's life, to deny Kellhus the
Gnosis. This also failed.

The Holy War reached Shimeh and prepared to assault the city. The Scarlet Spires
summoned three Ciphrang, demons of the Outside, to cause panic and terror in the
city and divert the attention of the Cishaurim. A final push would ensure victory, but
71the Men of the Tusk were divided by the need for a rapid, final assault and the need for
caution: less than a sixth of the forces that set out from Momemn over a year and a half
earlier survived. Any prolonged siege or assault would sap their strength dangerously.
At the moment of the great battle, however, Kellhus left them. He commanded the
battle to the valour of the Men of the Tusk, but he had a task to attend to elsewhere.

Kellhus struck west for Kyudea, an ancient and ruined city built near the remains of an
old Nonman mansion. In that mansion Kellhus finally found his father: Anasûrimbor
Moënghus, known to the Kianene as Mallahet. Moënghus told Kellhus that he knew
that Kellhus's journey would open his eyes to the secrets and mysteries that he had
encountered himself, setting out from Ishuäl thirty years earlier, and he set the Holy
War in motion to clear the way for Kellhus's journey. However, Moënghus was unable
to predict what would happen when the Holy War turned on Kellhus and tried to kill
him. When Kellhus explained how he survived, through having visions of the
Apocalypse and the No-God, Moënghus concluded that his son had been driven
insane. Kellhus, as befitting a Dûnyain, analysed the possibility but rejected it.

Moënghus revealed that twelve years earlier the Cishaurim had discovered the first skin-
spies. Reasoning they were the creations of the Scarlet Spires, despite Moënghus's
claims otherwise, they assassinated the Grandmaster Sasheoka, beginning a clandestine
war between the two orders. He then interrogated the skin-spies and learned of the
Consult and the threatened Second Apocalypse. Kellhus realised that if his father
accepted that he was damned to eternal torment he may take the same view as the
Inchoroi, that destroying humanity may be the only way to seal the Outside and end
the threat. To remove the danger, Kellhus stabbed his father and then left, innovating
the use of a Cant of Transposition to transport himself to Shimeh.

In the meantime, Cnaiür had returned to the Holy War and sought out Achamian. He
told the sorcerer of the Dûnyain and the true nature of both Kellhus and Moënghus.
He and his skin-spy liberators then left, following Kellhus's trail to Moënghus where
they found the man dying. Achamian tried to convince Esmenet to abandon Kellhus's
cause but she refused.

Drusas Achamian uses the Gnosis to defeat Zioz, a Ciphrang demon from the Outside.

The Holy War launched its assault on Shimeh. Initial successes turned sour when it was
revealed that Fanayal and the Cishaurim had prepared a trap, allowing part of the Holy
War to enter the city before trapping and destroying it. The ferocity of the Men-of-the-
Tusk again took the Fanim by surprise, but their numbers were no longer enough to
deliver them victory.

Ikurei Conphas's Nansur columns, pursuing Cnaiür, prepared to enter the fray. Having
learned of the death of his uncle in Momemn, Conphas had declared himself Emperor
and prepared to use his military might to end the threat of the Warrior-Prophet once
and for all. However, Conphas overreached and was killed on the battlefield by
Saubon. His forces were then redeployed against the Fanim of Kian, helping deliver a
shocking defeat to them. Elsewhere on the field the Scarlet Spires, driven into an
enraged frenzy to avenge their slain Grandmaster, drove the Cishaurim to the brink of
defeat, aided by the late-arriving Imperial Saik. But victory was still poised on a knife's
edge. It was only gained when Kellhus translocated himself into the midst of the last
surviving Cishaurim, slaying them before they even knew what was happening. Kellhus
had now mastered the Metagnosis, a more powerful and formidable form of the Gnosis
lost for millennia, even to the Nonmen.

The Warrior-Prophet delivered Shimeh to the Men of the Tusk.

For this last, great victory, Anasûrimbor Kellhus was proclaimed Aspect-Emperor of the
Three Seas, acclaimed so by the Holy Shriah, the Thousand Temples, the School of
Mandate, and all the princes and kings who had followed the Holy War on its great
journey. He was acclaimed by all\ldots{}bar one.

Drusas Achamian came before his former pupil and repudiated him. He renounced his
role as tutor and advisor to Kellhus, his place in the Mandate, he renounced his
prophet and his wife before going into exile. Kellhus told him that the next time he
came before the Aspect-Emperor, Drusas Achamian would kneel.

A stylised representation of Anasûrimbor Kellhus. Born in Ishuäl in 4076, he was sent out into the World in 4109 at the command of the Dûnyain. By 4122 he had conquered the Three Seas, mastered the Gnosis and been crowned Aspect-Emperor.

\hypertarget{part-6-the-unification-wars}{%
\chapter{Part 6: The Unification Wars}\label{part-6-the-unification-wars}}

Word of the great victory at Shimeh spread to all the corners of the Three Seas. The
Holy War had triumphed. The heathen Fanim had been put to rout and the Holy City
restored to the Faithful. But even more remarkable were the stories that accompanied
the news. A new leader had emerged from the ranks of the Holy War. He had survived
death, performed great miracles and pulled the battered, bloodied remnants of the
crusade to a victory against odds unthinkable. Here was a story from the very Sagas
brought to life.

75Anasûrimbor Kellhus was proclaimed the Aspect-Emperor of the Three Seas by the
Shriah of the Thousand Temples. Tens of thousands of Men of the Tusk, forged in the
burning heat of the Great Carathay and tempered on the battlefields of Caraskand and
Shimeh, swore themselves his eternal subjects, his Zaudunyani , the ``Tribe of Truth''.
Even three of the sorcerous schools (the Imperial Saik, the Scarlet Spires and the
Mandate) had sworn to his service. His victory, his rule, seemed unquestionable.

But history is never so simple. Across the Three Seas there was shock that this man, this
prince of nothing , had come out of nowhere and seemingly subverted the Holy War to
his own ends. Many dismissed him as a fraudster, or even a Ciphrang, a demon from
the Outside sent to lead men to their destruction. Some who may have been tempted to
hear him were disgusted to hear that he preached of the threat of the Consult and the
Inchoroi: children's stories that no-one but those doddering old fools in Atyersus took
seriously. Armies were summoned, swords forged and bows strung as the opponents of
the new Aspect-Emperor, the Orthodox, braced themselves for war. Likewise,
Maithanet's support for Kellhus had shattered the Thousand Temples, leading to many
priests -- the Schismatics -- taking up arms in defence of the faith and in opposite to the
Shriah.

Only one nation declared for Kellhus in its totality: Conriya, united under the rule of
Nersei Proyas. Every other nation splintered, the entire caste-nobility of the Three Seas
divided. Provinces and palatinates and principalities declared for or against Kellhus,
often depending on the zeal of their troops and rulers still encamped with the Holy
War around Shimeh. Most of civilised Eärwa teetered on the brink of civil war, moreso
in the Nansurium after the unexpected deaths of both Emperor Ikurei Xerius and his
heir, Ikurei Conphas, on campaign, leaving the Ikurei Dynasty extinguished.

But the Holy War was not done. Refreshed, reinforced (by the Mandate and other
sorcerers flocking to Kellhus's banner) and resupplied, the Holy War struck south and
west into Kian proper. The long war had exhausted the fighting strength of the Fanim
and they could offer no effective resistance. Fanayal ab Kascamandri was unable to rally
his people and melted away into the Carathay Desert. By the end of 4113 the Holy War
76had seized Nenciphon and installed the Emperor and Empress in the White-Sun
Palace. Many soldiers formerly loyal to the Empire now switching their loyalty to
Anasûrimbor Kellhus. Massar ab Kascamandri, the brother of Fanayal, underwent the
Whelming, the spiritual induction into the ranks of the Zaudunyani, and swore his
entire nation to the service of Kellhus.

In 4114 Kellhus published a tract on sorcery. The Novum Arcanum attracted great
attention for its revelations and insights into sorcery and logic. The following year
Kellhus announced a great gathering of sorcerers from across Eärwa and they came in
unprecedented numbers to learn from him and hear his great Rehabilitation of Sorcery.
All Shrial and Tusk condemnations of the practice were rescinded and sorcerers were
no longer held to be anathema. Through such acts Kellhus won every sorcerer of rank
and power in Eärwa (save one) to his side, the sorcerous schools united under his
banner.

A witch of the Swayal Compact wielding the Gnosis. In magical power they matched the Mandate, whilst in terms of sheer numbers they outstripped all of the other schools, making them arguably the most powerful force in Eärwa apart from the Aspect-Emperor himself.

Kellhus also made his second great proclamation: the Manumission of the Feminine.
All limitations -- legal, spiritual or moral -- placed on the comportment of women were
struck down. Women now had full equal rights to men across the Three Seas. This was
initially a more controversial declaration, and seized upon by Kellhus's opponents as
77proof of his madness, but it was also popular amongst, of course, the women of the
Three Seas, particular with regard to inheritance and property rights. Even more
dramatic was that the combination of the two declarations effectively ended the ban on
women joining the Few. For centuries women wielding sorcery had been scorned as
witches , burned at the stake or stoned to death even by those men who trafficked with
sorcerers themselves. Now they were allowed to come out of the shadows, in numbers
which caught the men of the Three Seas by surprise.

Even more breath-taking was what Kellhus did for these women: he commanded the
Mandate to instruct them in the ways of the Gnosis, and gave to them the abandoned
Cûnuroi Mansion of Illisserû in Holy Amoteu as their stronghold, now renamed
Orovelai. He made them a simple promise, to support and empower them in return for
their support in turn. This became known as the Swayal Compact, the name also taken
by the witches (a name many of them now wore with pride). Within a decade their
knowledge and mastery of the Gnosis rivalled that of the Mandate and their numbers
far outstripped them.

Kellhus won loyalty, even fanatical and maddened loyalty, in his own way. Within a
year of the fall of Nenciphon, his missionary-zealots had begun making their way across
the Three Seas. They became known as the Zaudûn Angnaya, the ``floating college'' of
young aspirants who learned from Kellhus whenever they could. They sought to
persuade through argument, reason and, whenever that failed, conviction . Horrified
stories spread amongst the Orthodox of ``suicide sermons'', when Angnaya would slit
their own throats in front of the vast crowds to prove their absolute faith. At first, they
used such demonstrations as proof of Kellhus's danger and insanity, but the
unshakeable faith and certitude of the zealots shook the Orthodox, who had no
spiritual answer for them.

The Unification Wars. In less than ten years, Anasûrimbor Kellhus brought the Three Seas under his
rule. With the final capitulation of Nilnamesh in 4122, Kellhus became its sole political, military,
religious and sorcerous leader: its Aspect-Emperor, with approximately 75 million souls living under his ule.

By the end of 4114 war had come: the Fanim inspired a massive uprising in Shigek, but
this had been crushed by Rash Soptet, Lord of the Sempis. The growing rift in the
Thousand Temples erupted in bloodletting, the War-between-Temples. Nilnamesh,
long separated from its Inrithi brethren by the width of the Kian Empire, also declared
against Kellhus.

In 4115 Prince Shoddû Akirapita assembled a large army in Nilnamesh and moved to
defend the border. The Zaudunyani were defeated at the Battle of Pinropis, to their
surprise. Kellhus took time to regroup, during which time his allies achieved greater
victories: in 4116 Coithus Narnol declared for Kellhus and delivered Galeoth almost
intact to his banner. King Hringa Vûkyelt likewise unified Thunyerus in Kellhus's
name and expelled the Schismatics from the kingdom. The following year both Ce
Tydonn and High Ainon became divided in a bitter civil war, followed by the
declaration of Ce Tydonn for Kellhus in 4118. Cironj also fell in this year.

High Ainon presented Kellhus with a major problem: the nation was vast and unruly at
the best of times but unified in its fear of the Scarlet Spires. But the Holy War had
almost destroyed the order altogether, with barely a dozen sorcerers-of-rank surviving
the conflagration at Shimeh. To their humiliation, Kellhus award the Mandate
command of Kiz, the former Scarlet Spires stronghold in Carythusal. From there the
Mandate was able to bring the rule of the Aspect-Emperor to lower Ainon, but the full
capitulation of the kingdom took longer. In 4120 the Sack of Sarneveh took place,
Kellhus himself leading the capture of the city. Although successful, the Toll of
Casualties (a meticulous accounting of the cost of each victory, which Kellhus
abandoned after this point) recorded more than five thousand children slain. This news
escaped the city, encouraging further resistance to Kellhus. However, by the end of
4121 High Ainon had fallen and declared for Kellhus.

At this point, a curiosity took place, one which even the most fanatical Zaudunyani
have struggled to reconcile with their extolling of Kellhus as a messenger of the divine.
Following the conquest of High Ainon, Kellhus spent four months in Kiz as a student
of Heramari Iyokus, the famed Blind Necromancer and a master practitioner of the
Daimos, the sorcerous art of communing with and summoning demons. At the end of
this tutelage Kellhus emerged with two grotesque demon head bound to his hip by
their hair: the Decapitants. Kellhus demurred on explaining their origin, often ignoring
the question altogether. Rumour said that the Aspect-Emperor had somehow plumbed
the very Hells themselves for knowledge and returned with the heads as trophies of his
victory, and to remind the Aspect-Emperor of the fate awaiting all those who were
damned.

Also in 4121, the Nilnameshi capital of Invishi had finally fallen to the Zaudunyani.
However, Prince Akirapita refused to capitulate, gathering a new army. It was not until
this army was destroyed at the Battle of Ushgarwal in 4122 and the Prince slain (his
body was found in a well in Girgash in 4123) that Nilnamesh could finally be said to
have been brought into the fold. This left only Fanayal ab Kascamandri out of the
Aspect-Emperor's many foes, and his forces were reduced to a few tribesfolk of the
Great Salt.

The Unification Wars were declared over in 4122. Maithanet, having won the War-
between-Temples, crowned Anasûrimbor Kellhus the Aspect-Emperor of the Three Seas
in Momemn, which Kellhus had taken as his capital. Kellhus and his wife, Esmenet,
now had several children -- Kayûtas (b. 4112), Theliopa (b. 4114, in Nenciphon), Serwa
(b. 4115) and Inrilatas (b. 4117) -- and more would follow, the twins Kelmomas and
Samarmas (both b. 2124). They had also adopted the son of Cnaiür urs Skiötha and
Serwë, Moënghus II (b. 4111) as their own. The result was that they had already
established a dynasty, one with the power to rule the Three Seas for generations to
come.

But the new goal of the Anasûrimbor family was not to simply rule. Kellhus declared
war on Golgotterath and the Unholy Consult. He declared his goal was to destroy the
dread Ark and cast down its Golden Horns forever. His purpose was to forestall the
return of the No-God, prevent the Second Apocalypse and to save the World itself. To
this end he commanded the establishing of the greatest army in human history. Swords
and armour were forged on a titanic scale. Horses were bred in their tens of thousands.
Supply caches were established in the northern Empire, near the Kathol Pass leading to
the vast Istyuli Plains. Sorcerers were called to train and learn as they never had before,
and to prepare for the war to come, which would be known as the Great Ordeal.

Drusas Achamian, the only Wizard (a sorcerer-without-a-school) of the Three Seas.

\hypertarget{part-7-the-great-ordeal}{%
\chapter{Part 7: The Great Ordeal}\label{part-7-the-great-ordeal}}

At one time Drusas Achamian was an agent of the Mandate, a sorcerer haunted by
dreams of Seswatha, hero of the First Apocalypse, and by fears that the Second was
coming. During the chaotic swirl of the Holy War he found a man whom he believed
could save humanity and lead it to victory over the ancient foe, the Unholy Consult.
Anasûrimbor Kellhus led the Holy War to victory, but in doing so he stole away
Achamian's love, Esmenet, and subverted the religious fervour and faith of millions to
build himself an empire.

Faced with the choice of kneeling to the Aspect-Emperor or repudiating him,
Achamian chose the latter. Unimpeded, at the Aspect-Emperor's express command,
82Achamian fled into the wilds of Galeoth, erecting a tower to live in solitude and
meditate on one question: ``Who is the Aspect-Emperor?''

Achamian knew only a few facts and hints, gained from the all-too-brief revelations of
Cnaiür urs Skiötha in the dying hours of the war. He knew that Kellhus was Dûnyain,
the member of an almost unknown sect that had survived the Apocalypse and the two
thousand years since in utter solitude, pursuing intellectual and philosophical pursuits.
He had teased out the name Ishuäl from Seswatha's dreams, the secret redoubt of the
House Anasûrimbor ere the fall of Kûniüri, but did not know where it could be found.
In desperation, he plunged further into the dreams, ploughing into them night after
night, writing down every nuance and every detail for signs of clues to the Dûnyain's
origins, which he now believed lay in the catastrophes of those times. Over time he
teased some new revelations from the dreams, such as the fact that Seswatha had
seduced his friend Anasûrimbor Celmomas II's wife, and that the famed hero Nau-
Cayûti may have been born from Seswatha's line rather than the house of
Anasûrimbor. But, although historically scandalous, Achamian could not tease
meaning from such revelations.

In the early spring of 4132 Year-of-the-Tusk, nigh on twenty years since the Fall of
Shimeh, Drusas Achamian received a surprising visitor: Anasûrimbor Mimara, the
oldest child and first daughter of Esmenet of Sumna. Born long before the Holy War,
when famine stalked the city and her mother was forced to give her away to save her
life, Mimara's life had been hard and cruel. Not long after Esmenet was installed in
Momemn as Empress, she sent agents to scour the Three Seas to find Mimara and
eventually they succeeded. But in the cold hallways of the Andiamine Heights, Mimara
did not found the home she sought. Her half-brothers and half-sisters, the children of
Kellhus, were not quite human and she found forging a bond with the mother who had
abandoned her too difficult. Mimara eventually learned that she was of the Few, but
sorcery was denied to her at her mother's command, even among the newly-crafted
School of Witches, the Swayal Compact. Frustrated, she abandoned Momemn and
sought out Achamian, the sorcerer-without-a-school, a wizard.

83Mimara tried to convince Achamian to teach her the Gnosis, but Achamian resisted,
even after she seduced him. Instead they swapped stories, Achamian revealing his quest
to find Kellhus's birthplace and Mimara revealing her own harsh upbringing. In the
midst of the conversation, Mimara revealed something Achamian was not expecting:

The Great Ordeal marches, old man.

The Great Ordeal arrives at Sakarpus, the City of the Plains and home to the famed Chorae Hoard.

From across the Three Seas assembled an armed host dwarfing any in history. A third
of a million men, clad in the finest armour and carrying the stoutest blades. A
thousand or more sorcerers, assembled from all of the Schools. Tens of thousands of
horses, millions of arrows. An army ten years in the planning. It gathered near the
Kathol Pass, assembling under the watchful eyes of the Believer-Kings and the great
heroes of the Holy War, chief amongst them Coithus Saubon and Nersei Proyas.

84The army marched north and west, through the mountains and into the edges of
civilisation. They closed around Sakarpus, the mighty city-of-the-plains which had
survived even the No-God itself during the First Apocalypse and whose mighty Chorae
Hoard struck fear into the hearts of sorcerers. But not the Aspect-Emperor. Kellhus
moved to take the city and King Horweel chose resistance, but also knew that his
resistance was doomed. He commanded that his son, Sorweel, be kept safe, even as his
own death drew near. After the city's surrender, Kellhus came before Sorweel and told
him that he was now King of Sakarpus and that his father would forgive him for
surviving where his father had not. Sorweel was inducted into the ranks of the Great
Ordeal, riding in the Scions, a horse company made up of princely hostages from across
Eärwa. Sorweel's new friends and allies including Zsoronga ut Nganka'kull, the
Successor Prince of Zeüm, and Eskeles, a Mandate sorcerer who taught the young king
Sheyic, the common tongue of the Three Seas. Sorweel also learned of the Aspect-
Emperor's rise to power and why so many regarded him as a prophet.

Also assigned to Sorweel was a slave, Porsparian. However, Porsparian was also an
initiate of the Cult of Yatwer, the fertility goddess. He moulded the face of Yatwer out
of the dirt and took mud from her lips to paste over Sorweel's face. Afterwards, Sorweel
found himself able to lie to the face of the Aspect-Emperor and not be discovered .
Yatwer's blessing rendered Sorweel immune to the Aspect-Emperor's supposedly holy
sight, which could discern almost instantly the presence of Consult skin-spies and read
the untruths from the lips of practiced liars. Instead, Kellhus believed that Sorweel had
become a true follower and named him a Believer-King, one of the most exalted rulers
in all Eärwa\ldots{}and a dagger positioned close to the Aspect-Emperor's heart.

Mimara's news panicked Drusas Achamian, who believed that momentous events were
passing, maybe even the events that would trigger the Second Apocalypse rather than
forestall it. Indeed, from his dreams he recalled the great First Ordeal of Celmomas, an
army fashioned at the urging of Seswatha to destroy Golgotterath ere the rise of the No-
God but which had ultimately failed. Leaving Mimara behind, Achamian travelled to
Marrow and there commissioned the services of the Skin Eaters, a party of ``scalpers'',
85mercenaries who travelled the far side of the Osthwai Mountains in search of Sranc
bounty.

Achamian told them of fabled Sauglish, the greatest seat of learning in Eärwa. It had
fallen to the No-God during the Apocalypse, but its Great Library had not been
completely destroyed. There, Achamian hoped to find the secret location of Ishuäl.
However, he also claimed the fabled Coffers, the Library's great treasury, likely still
endured and would reward the mercenaries with great riches. The mercenaries were
dubious, since the mission would involve crossing many hundreds of miles of
dangerous territory, but their greed and the urging of their mysterious Nonman ally,
Incariol or ``Cleric'', convinced them to take the commission. Soon after setting out
they were joined by Mimara, who had tracked Achamian from his home. Fearing for
her safety amongst such dangerous men, Achamian claimed her as his daughter.
Mimara disclosed to Achamian a secret that she had been hiding her whole life: she
bore a ``secret eye'', one which can see the absolute good and evil in all people.
Achamian recognised this as the Judging Eye , an exceedingly rare and potent ability.
Troubled, he would not reveal more.

The Skin Eaters make their way through the ancient ruins of Cil-Aujas.

Soon after setting out, the party discovered that the passes through the mountains had
been closed by spring blizzards. They instead chanced the ancient Nonman Mansion of
Cil-Aujas. The Mansion had survived the ancient Cûno-Inchoroi Wars, the wars against
the invading hordes of men and even the No-God, but had been betrayed and sacked
by the armies of men allied to it, an act of treachery renowned in history. The scalpers
passed through the mountain only to find it infested with Sranc. A dangerous running
battle through the mountain eventually ended in an even greater horror: the shade of
Gin'yursis, the Nonman King of Cil-Aujas, possessed Cleric and attacked the rest of the
party. The rest of the group was outmatched, but Mimara used her Judging Eye on one
of their Chorae, transforming it into a shining white Tear of God. This caused
Gin'yursis's shade to dissipate, and a troubled Cleric to return to normal. Shaken, the
party emerged on the far side of the Osthwai Mountains and began the gruelling march
-- the slog-of-slogs -- through the forested wilderness.

In Momemn, capital of the New Empire, the Empress Esmenet struggled to hold the
Empire together in her husband's absence. Her most capable children and stepson --
Kayûtas, Serwa and Moënghus -- had joined the Great Ordeal. Her daughter Theliopa
remained as an advisor, but she was scarcely human, possessing uncanny insights and
wisdom that had cracked her sanity. Inrilatas was locked away atop the Andiamine
Heights for his own safety and the safety of those around him. Only the twins
Kelmomas and Samarmas gave her any joy, but she was unaware that Kelmomas was a
cunning, conniving little creature, possessing both the analytical power of the Dûnyain
and a callous disregard for the consequences of his actions. Kelmomas engineered the
death of his idiot twin, determined to make his mother love him all the more.

But a new threat grew from within the Empire. Psatma Nannaferi, the outlawed
Mother-Supreme of the Cult of Yatwer, declared her goddess's war against the House of
Anasûrimbor. A holy assassin, the White-Luck Warrior (so-called because how he
walked the perfect path of circumstance and fortune), revealed himself and his mission
to lay waste to the House Imperial. Yatwer, as the god most fervently and widely-
worshipped by the slaves and caste-menials of the Three Seas, commanded enormous
respect and power and soon unrest began to spread throughout the entire Empire.

87Distracted by Samarmas's death, Esmenet employed the aid of her brother-in-law
Maithanet, Shriah of the Thousand Temples, to overcome this new threat. Maithanet
reminded her that the gods could not see, perceive or even comprehend the No-God
and were utterly blind to the coming threat of the Second Apocalypse (as they were to
the First, seeing it instead as inexplicable carnage wrought by man against himself).
Maithanet advised turning the Yatwerians against themselves by convincing their
official Matriarch, Sharacinth, to condemn Nannaferi. This plan was successful, albeit
only when Kellhus translocated from the Ordeal to the palace to cower the woman, but
sabotaged when Kelmomas, seeking to further isolate his mother from her other
concerns, murdered Sharacinth and her retinue. This further turned the poor of the
Empire against the imperial family.

The Skin Eaters fled from Cil-Aujas into the Mop, the vast, untamed forests covered
the lands north and east from the Osthwai Mountains to the Sea of Cerish. Their
course would take them through the ruins of the Apocalypse, skirting the edges of the
fallen Meörn Empire. The road was long and beset by challenges: Sranc in ever-
increasing numbers, and rival scalpers eager to kill them and loot their bodies. There
was also a threat from within. Using her conditioned training gained whilst in
Momemn, Mimara discovered that one of their number, Soma, was a skin-spy.
Achamian attempted to parley with the creature, but it chose to flee. It shadowed the
party, intervening during a Sranc raid to kill one of the creatures before it could harm
Mimara, to her bemusement. It informed her that she was pregnant with Achamian's
child. Soma was working on the orders of the Inchoroi, communicating with them by
Synthese. Their order was that Mimara must be protected all the way to Golgotterath if
necessary. All of the prophecies must be respected, ``the false as much as the true''.

The Great Ordeal entered the heart of the vast Istyuli Plains, a colossal tableland
extending across thousands of miles. Concerned with supply, the Aspect-Emperor
ordered that the Ordeal should split into four armies and march separately to improve
their chances of foraging. Believer-King Sorweel earned respect for his scouting (using
both his native knowledge of the plains and also information gained from his divine connection to Yatwer), which uncovered evidence of Sranc having been in the region
recently but having now fallen back before the Ordeal. Such signs grew until they
become indisputable: a vast Sranc Horde was gathering ahead of the Ordeal, giving way
before it and accumulating all of the tribes of the plains. The numbers lying beyond the
northern horizon defied all rationality, and the Scions raced back to the Ordeal to give
them warning.

The Culling. Hundreds of sorcerers scour the edges of the Sranc Horde, forcing them back whilst outriders of the Great Ordeal slaughter the stragglers. This would continue day after day for months as the Ordeal crossed the Istyuli Plains, and it barely held the Sranc at bay.

They halted at last, swamped by the enemy and protected from death only by Eskeles'
sorcerous wards. At Sorweel's urging, Eskeles found a way of alerting the Ordeal by
illuminating the sky above him with a Bar of Heaven. They were rescued by witches led
by Anasûrimbor Serwa and returned to the Ordeal. Sorweel was praised for his actions,
which had saved the Ordeal from a costly ambush. In the aftermath of the action, the
Ordeal deployed its sorcerers to begin the Culling, with hundreds of sorcerers using
vast amounts of power to burn Sranc in their thousands and tens of thousands from
above. The slaughter was immense, but the losses were made good -- and more -- by
freshly-arriving Sranc clans from the High Istyuli.

89Sorweel confessed his relationship with Yatwer to his friend Zsoronga, who realised that
Sorweel had been positioned as a Narindari , a divine assassin sent to murder Kellhus.
This seemed confirmed when Kellhus sent the order to execute all slaves and
unnecessary non-combatants to help preserve food supplies, Porsparian amongst them.
Before his death, Porsparian summoned a visage of Yatwer. Yatwer gave Sorweel a
mighty gift: a single Chorae, hidden in a pouch that could shield its detection by the
Few. The image disappeared and Porsparian killed himself. Sorweel was now trusted
enough to get close to the Aspect-Emperor and kill him, but feared that Yatwer's
protection would not be enough. He sought out Serwa to thank her for saving him, and
discovered that she could also not detect either his hidden Chorae nor his deception.

Soonafter, surprising news came: Ishterebinth, last of the Nonmen Mansions, had
rallied to Kellhus's cause. A Nonmen emissary of King Nil'giccas offered to support the
Ordeal if, in return according to the ancient codes, three hostages were sent to the
Mansion and the Ordeal could prove its worth by retaking Dagliash, the ancient
fortress of men built atop the fallen Mansion of Viri. For his hostages Kellhus chose his
stepson Moënghus, his daughter Serwa\ldots{}and Sorweel. Using metagnostic cants of
translocation, Serwa took her brother and Sorweel to Ishterebinth at great speed. On
this journey Sorweel learned that Serwa and her stepbrother were involved in an ill-
advised, passionate relationship.

The Great Ordeal approached the shores of Neleöst, the Misty Sea, but also fell into a
trap. The Army of the South fortified the ruined stronghold of Irsûlor on the western
flank of the Ordeal, unaware that the Sranc Horde had swung to the west in an attempt
to catch them by surprise. The Mandate and the Vokalati schools joined forces to burn
the Sranc, but the creatures suddenly surged through the curtain of magic to swamp the
fortress. The Consult had also hidden a legion of Bashrags within the ruins. At a key
moment, they emerged to shatter the human lines. In the slaughter that followed the
Army of the South was utterly destroyed, many sorcerers dying with them.

In the wake of the catastrophe, with near a quarter of the Ordeal lost, the three
remaining armies of the Ordeal reunited and heard the command of the Aspect-Emperor on the holy heights of the great hill Swaranûl, overlooking the sea. The
Ordeal had exhausted its supplies and would now have to eat the bodies of the fallen
Sranc.

Cleric, a Nonman Erratic working with the Skin Eaters.

The Skin Eaters pressed on across the Istyuli Plains, crossing the line of march of the
Ordeal. With rations low, Cleric had taken to dispensing Qirri , a form of Nonman
sustenance, to the company. It fortified them, allowing them to march longer and
harder than would otherwise be possible. Mimara eventually learned the truth: Qirri
was the powdered remains of ancient Nonmen heroes, and the pouch Cleric carried
contained none of than the remains of Cû'jara Cinmoi himself. Horrified, Achamian
attempted to refuse the Qirri. He was bound and gagged by the other scalpers. Captain
91Kosoter then revealed that he was an agent of Anasûrimbor Kellhus, begging Mimara to
help save him from damnation. Mimara used the Judging Eye on Kosoter and
discovered that his soul was blackened from the atrocities he had carried out at the
Aspect-Emperor's command, and he was utterly beyond redemption.

The party reached the ruins of Sauglish, but Mimara had discovered another truth:
Cleric was none other than Nil'giccas himself, the High King of Ishterebinth and the
great hero of the Cûno-Inchoroi Wars. She disclosed this to Achamian, who was
released. Kosoter ordered Cleric and Achamian into the ruins to find the Library and
the Coffers, whilst he kept Mimara as surety. However, Kosoter's men, traumatised by
the horrors of their journey, turned on and murdered him. Mimara would have been
killed had not the skin-spy Soma returned and saved her for reasons still unclear.

In the ruins of Sauglish Achamian attempted to reason with Cleric by awakening his
memories of Nil'giccas, but this backfired, awakening instead the Nonman's desire for
pain and suffering, for only those things made impressions in his multi-millennial
memory. Their burgeoning discord was interrupted by the discovery within Sauglish of
Wutteät, the Father-of-Dragons, a dread Wracu of the Ark who had once borne the
obscene King Sil on his back. Achamian attempted to reason with the creature in
exchange for the map to Ishuäl, learning that he had travelled on the Ark with the
Inchoroi from other worlds. The goal of the Inchoroi was to reduce the number of
souls on any given world to just 144,000, in pursuit of their cause of sealing the world
against the Outside and avoiding damnation.

Battle erupted and the Mandate schoolman and Qûya sorcerer combined their power
to drive the creature from the skies, as in the days of old. But Nil'giccas turned on
Achamian and would have killed him, had Achamian not noticed that Nil'giccas had
raised no sorcerous wards of his own, inviting his own destruction. Achamian,
sorrowfully, complied. Afterwards he burned the great Nonman's remains, taking the
ashes as Qirri. In the ruins of the library, using Seswatha's memories, he found the map
showing where House Anasûrimbor had secretly built Ishuäl, high in the Demua
Mountains.

Returning to the camp, Achamian found Mimara safe but the skin-spy, Kosoter and
most of his men dead and the rest fled. They now set out north and west for the
mountains.

Second Negotiant Malowebi, a Mbimayu Schoolman of Zeüm, was sent to the camp of
Fanayal to assess his strength and see if the Fanim and Zeüm could strike up an alliance
to oppose the Aspect-Emperor. Malowebi was unimpressed with Fanayal, whose twenty
years of exile in the desert had taken a toll, but was more heartened by the continued
survival of Meppa, the Last Cishaurim. The Fanim marched on the Sempis River
Valley, Malowebi accompanying them as an observer on behalf of the Great Satakhan.

The paths of the Great Ordeal and the Skin Eaters.

In Momemn, the Empress learned of the advancing Fanim armies. Disturbed, and
aware how militarily weak the New Empire was in Kellhus's absence, she showed
Kelmomas a secret network of tunnels riddling the Andiamine Heights. Suspicious of
Maithanet's role in events, the Empress asked her children for help. Theliopa suggested
that she use Inrilatas to sound out Maithanet's goals. Although mad, Inrilatas had also
93inherited more of his father's gifts than any of his siblings. Esmenet complied,
convincing Maithanet to visit Inrilatas. The only other person present was Kelmomas.
Maithanet admitted he had doubts about Kellhus's plans, but only because Kellhus has
allowed his love for his children and wife to cloud the Thousandfold Thought (to the
surprise of Kelmomas, who believed his father incapable of such an emotion). Inrilatas
exposed Kelmomas's crimes, shocking even Maithanet, before attempting to murder
him. He failed and Maithanet killed Inrilatas in self-defence. Kelmomas denounced his
uncle as an assassin and murderer, throwing the city into an uproar. Esmenet
contracted an assassin to kill her brother-in-law, unaware that the man she found for
the job was the White-Luck Warrior himself.

The Fanim armies took Iothiah, the capital of Old Shigek, impressing even the sceptical
Malowebi. After the conquest Fanayal was confronted by Psatma Nannaferi who,
speaking for the Cult of Yatwer (and Yatwer herself), cultivated an alliance with the
Fanim to bring down the demon-emperor Kellhus. Malowebi reported on these events
to his Satakhan but cautioned that Kellhus emptying the Three Seas of armies
suggested that he genuinely believed in the threat of the Consult and the No-God. The
Satakhan nevertheless sensed an opportunity. He told Malowebi to offer Fanayal an
alliance: High Holy Zeüm would support the Fanim if they could take Momemn itself.

Esmenet met with her assassin in secret in the city, only to find that Maithanet's troops
had captured the palace in her absence. Taken into hiding by her captain of the guards,
Esmenet fretted over the fate of her children, especially Kelmomas. However,
Kelmomas made use of the secret passages she had shown him to hide from guards,
occasionally emerging to steal food (and, horrifically, to kill guards and use them for
sustenance).

Esmenet was eventually found and taken back to the palace. Maithanet demanded to
know why she had tried to kill him, but she said this had not been her intent. Using his
Dûnyain conditioning, Maithanet realised she spoke true. Maithanet released her and
told her his suspicion: that Kellhus had created the New Empire purely as a machine to
create the forces he needed to march on Golgotterath. That achieved, he no longer
94required the Empire and had abandoned it to its ruin. Esmenet was horrified.
Maithanet announced the reconciliation of the Imperial Throne and the Thousand
Temples, but was suddenly struck down by the White-Luck Warrior, acting on
Esmenet's orders. Esmenet declared Maithanet a traitor and heretic, the murderer of
the Aspect-Emperor's son. The city rallied to her cause, just the trumpets of the Fanim
sounded out and Fanayal's army arrived to besiege the city.

Achamian's dreams had been changing for years, but now they had taken on a new
level of detail, images from not just Seswatha but also his great friend Celmomas and
his son Nau-Cayûti. On the Fields of Eleneöt, Achamian saw Celmomas's death from
the perspective of Celmomas himself . At the moment of his death he uttered his
famous prophecy, that at the end of the world an Anasûrimbor would return. But when
he uttered this prophecy it was because a vision of Gilgaöl, God of War, had come
before him, holding an image of a man. And that man was Anasûrimbor Kellhus. This
revelation shocked Achamian but he did not know what to make of it.

In Ishterebinth, the envoys Sorweel, Serwa and Moënghus were seized by the Nonmen
and put to the question. Sorweel quickly discovered that Nil'giccas was missing, having
fled in ages past, and that the remaining Nonmen had formed an alliance with the
Consult. The interrogation of Sorweel discovered his connection to Yatwer and his
destiny was to kill the Aspect-Emperor. Upon the revelation of this news, he was
released into the custody of Oinaral Lastborn, who could provide him aid in his quest.
He was given the Amiolas, one of the greatest sorcerous artifacts in Nonman history,
which bonded his soul to that of a slain malcontent, Immiriccas. This resulted in
Sorweel learning the Nonman language, but also learning the truth of the Unholy
Consult: through Immiriccas's memories he relieved some of the Cûno-Inchoroi Wars.
Thus, the Consult's gambit backfired: Sorweel became convinced of the righteousness
of the Aspect-Emperor's cause.

Anasûrimbor Serwa, daughter of the Aspect-Emperor and Grandmistress of the Swayal Compact.
Wielder of the Metagnosis, the most powerful sorcerer since Titirga himself, save only her father.

Oinaral had also become doubtful of the wisdom of allying with the great enemy, so
took Sorweel into the deepest part of the mountain. They sought Oirûnas, Oinaral's
father, a great hero of the wars against the Inchoroi. Find him they did, but the hero
had been driven utterly insane by the passage of the millennia. In a rage, he slew his
son and tore the Amiolas from Sorweel's face, but Sorweel was able to guide him to the
inhabited part of the mountain. There Oirûnas, Lord of the Watch, slew Nin'ciljiras,
the Consult pretender to the throne of Ishterebinth and the last survivor of the line of
Nin'janjin. During the resulting chaos Sorweel freed Moënghus, who had been near
broken by his captivity, and Serwa, who had not. Serwa sang her metagnostic cants of
destruction\ldots{}

96The Great Ordeal marched north from the charnel fields of Irsûlor, crossing into the
ancient, long-fallen kingdom of Aörsi. With their food gone, the Ordeal followed the
command to consume Sranc. The Culling took on a new form, with Sranc corpses now
collected for consumption. As the Ordeal marched, its character changed, becoming
something more animal. The Exalt-General, Nersei Proyas, became concerned at this
turn of events, believing it threatened both the spiritual and moral superiority of the
cause. He turned to his Aspect-Emperor, hoping for reassurance, but instead Kellhus
admitted the unthinkable: that he was just a man searching for answers and doing the
best he could. He was no god, no prophet, just a man, with a man's needs. He
``seduced'' an uncomprehending Proyas.

This act broke Proyas's faith in the Aspect-Emperor. He sought advice and aid from his
brother in rank and war, Coithus Saubon, but Saubon remained constant in his belief
in the Aspect-Emperor's plans.

Now the Great Ordeal drew close to the River Sursa and the small mountain range
known as the Urokkas, which lay above the fallen Nonman mansion of Viri. During
the time of the First Ordeal and the Apocalypse, a great fortress called Dagliash had
been built atop Antareg, the northern-most of the mountains. Kellhus now planned to
use the geography of the region to go on the offensive.

The great Sranc Horde was forced to divide around the mountains and skirt the river,
allowing the Ordeal to concentrate the bulk of its forces against a smaller fraction of
the Horde and destroy it in detail. The sorcerers brought their full power to bear
against the Horde and bloody slaughter was wrecked in a fashion never before seen.
The Sranc were suddenly being annihilated at a speed they could not replace, and their
destruction seemed imminent.

However, Kellhus perceived movement in the fallen mansion and realised that the
Consult had seeded many Sranc and Bashrags in the ruins, preparing to catch the
Ordeal unawares. Saubon led an assault on the mountain of Antareg and engaged the
Bashrags instead, defeating the Consult's plans.

Victory seemed near, so the Inchoroi Aurang took the field, challenging the sorcerers.
Saccarees, the head of the Mandate, met the challenge but Aurang suddenly fled.
Sensing this was a distraction, Kellhus returned to the peak of Antareg and emptied it,
pulling the contents of the fallen fortress of Dagliash and the Nonman mansion below
up into the sky and hurling them down onto the Horde. Amongst the artifacts pulled
out of the ruins was a strange Tekne device, a cube with unknown figures on it,
changing, counting down\ldots{}

The Aspect-Emperor translocated from the field, leaving behind a command to the
Great Ordeal to flee Dagliash, to abandon the mountain. The army complied but for
those closest to the device there was no time. A tremendous light filled the sky, a
roaring blast that consumed the mountain top, most of the Sranc Horde and an
appreciable fraction of the Great Ordeal. Coithus Saubon and those closest to the
device were killed instantly, vaporised by a fireball that rose into the sky and became a
dark mushroom cloud filling the sky. This was a weapon that the Inchoroi had
deployed before but rarely, only in the most desperate days of their wars with the
Nonmen. A Scalding.

Drusas Achamian and Anasûrimbor Mimara arrived at Ishuäl, the stronghold of the
Dûnyain. They sought answers but instead found ashes: the Thousand-Thousand Halls
were a desolate ruin, the bones of men, Bashrags and Sranc and Nonmen everywhere.
They also learned what had already been suspected, that the Dûnyain were not fully
human. Through surgery and genetic manipulation, they had reduced their women to
little more than breathing wombs for their sons, the so-called Whale-mothers. The
Judging Eye confirmed what was now obvious: despite their claims to rationality and
cold logic, the Dûnyain were absolute evil, beyond any question. The question now
arose if Kellhus was also evil. Achamian told Mimara that their mission now had a new
goal: to apprehend Kellhus with the Judging Eye.

They also found survivors, the son and grandson of Kellhus. They learned that the
Consult had finally found Ishuäl and assaulted it in a fury. It had taken years of bitter
98fighting, but they finally destroyed the fortress, leaving behind just two survivors
scrabbling for sustenance in the dark. The man and the boy were Dûnyain, quick to
analyse and grasp Mimara and Achamian's desires and motivations, but they were also
stymied by both Achamian's sorcery and by the Judging Eye, which rendered their
attempts at deception moot. The father -- the Survivor -- whose sanity was already
precarious after the fall of Ishuäl, was driven to madness by the revelations and by the
Qirri he was given to consume. He threw himself to his death.

The three travellers descended the mountains north and eastwards. On the horizon
they saw a storm and, using sorcery, Achamian magnified the image to show the horror
cloud rising above the far northern shore of the Misty Sea. Then they were captured,
taken prisoner by the Scylvendi. To his surprise, Achamian found himself once again in
the company of Cnaiür urs Skiötha. He learned that Cnaiür survived the aftermath of
Shimeh and, helped by the Consult skin-spies, had seized control of the entire
Scylvendi race. They now marched to Golgotterath, to the relief of the Unholy Consult.
Cnaiür planned to kill Anasûrimbor Kellhus, his hatred undimmed by the passage of
twenty years. Learning of Mimara's ability and that they might be enemies of Kellhus
themselves, Cnaiür chose to release them. The boy from Ishuäl fled in another
direction, both the Scylvendi and Achamian content to let him go.

In the wilds of Kûniüri, Achamian dreamed of the First Apocalypse, and the revelations
shook him. Anasûrimbor Nau-Cayûti was poisoned by his wife Iëva, who was jealous of
his love for Aulisi, the love that had compelled him to risk even the dread Ark. At her
insistence, Nau-Cayûti was buried rather than burned, but he was not dead. The poison
had merely given him the appearance of death. His still-breathing body was dug up by
Aurang and borne to the Incû-Holoinas. There, he was tormented and tortured by the
Consult, who demanded to know the location of the Heron Spear. He discovered that
Shaeönanra still lived thanks to a hideous contrivance, a device which bound several
still-living people together. Shaeönanra soul moved between the bodies as a way of
constantly avoiding death and thus prolonging his life for millennia. Nau-Cayûti was
subjected to every horror imaginable but did not break. He refused to disclose the
location of the Heron Spear, enduring two years or more of interrogation. Finally, he
99was forced to join a line of prisoners, weighed down by chains. They were slowly drawn
through the Ark and taken, one by one, inside a strange object: a sarcophagus with
eleven Chorae embedded in it.

The Carapace of the No-God. Awakening, Achamian was horrified at the revelation,
but could not yet grasp its full implications.

A representation of Ajokli, the Four-Horned God of Deceit.

In Momemn Esmenet had restored order and commanded the defence of the city,
defying several attempts to attack the walls. Kelmomas was more intrigued by the
White-Luck Warrior, seeing him stalk the halls and bring about the death of his sister
Theliopa, who had been the greatest threat to him. Kelmomas has been driven mad,
hearing the voice of his twin Sarmamas in his ear and believing that Ajokli, the evil
Four-Horned God, was now his protector. In an unguarded moment, Kelmomas was
caught celebrating his sister's death by his horrified mother and fled.

A powerful earthquake struck Momemn, the gods again moving against the House of
Anasûrimbor in its moment of weakness. The Fanim prepared to attack, only for
Kellhus to translocate straight into their midst. He killed Fanayal and Nannaferi
100without hesitation, brought down Meppa in a sorcerous exchange and took Malowebi
as a prisoner. Malowebi attempted to invoke the Blue Lotus Treaty between Zeüm and
the Empire to ensure his safety, but the Aspect-Emperor was unimpressed. Using his
sword, he decapitated Malowebi and replaced his head with that of one of the
Decapitants, the demon-heads affixed to his hip. The possessing demon took on the
form of Malowebi as he was ordered by Kellhus to return to Domyot and kill the Grand
Satakhan. Malowebi's head, still conscious and aware, was tied to his hip instead.

Kellhus then returned to the palace, telling Esmenet he had returned to rescue her. The
White-Luck Warrior struck, Kelmomas distracting his father at a crucial moment\ldots{}

A fresh earthquake struck the city, the Andiamine Heights collapsing in on themselves
and everyone inside.

Thousands of miles to the north, Nersei Proyas regrouped the Ordeal, finding that only
a third of the force that had left the Empire now survived. With the Aspect-Emperor
missing, he commanded them to muster, to take as much sustenance as they could and
to cross the Sursa. The Golden Horns of Golgotterath were very close now.

Momemn fell. The city was destroyed, the waters of the Meneanor flooding the city
even as earthquakes wracked it. Kellhus, Esmenet and Kelmomas translocated through
the Metagnosis, saved from death at the last moment by Kellhus's sorcery. Kellhus told
Esmenet that the Empire had served its purpose: it was a ladder he needed to reach
Golgotterath. Only the Great Ordeal matters now. He also discerned Kelmomas's role
in the murders and destruction that had befallen the palace prior to the Fanim attack.
Esmenet was struck with horror by her son's actions. Kellhus told his son he would
spare him to spare his mother's heart. They resumed their journey back to the Ordeal.

Ishterebinth fell in to chaos. Sorweel, freed from the Amiolas, fled the mountain with
Serwa and Moënghus. Moënghus was filled with despair from his torture, but Serwa
seemed invigorated from the experience. Sorweel now knew the truth of the horror: he
had seen the halls of the Min-Uroikas, he had witnessed battlefields eight millennia
gone and had beheld the Whirlwind through ancient eyes: the No-God was real and
the Second Apocalypse was coming. This now wedded Sorweel to the Anasûrimbor
cause and he now believed fully in Kellhus\ldots{}and his daughter. For her part, Serwa told
Sorweel from her Mandate memories that Seswatha had worn the Amiolas three times
and she recalled its power through her inherited knowledge. Astonished by the love
that Sorweel held for her, Serwa submitted to an un-Dûnyain impulse and took him as
her lover. During their flight north along the Demua, the trio were cornered by a
Scylvendi scouting party on the mountain known as Shaugiriol , ``Eaglehorn,'' and
Serwa translocated them across the Leash, but at the last moment Moënghus fell from
her grasp -- or was pushed -- and was left behind. Serwa warned Sorweel that Moënghus
had been broken by Ishterebinth and the greater cause was served by leaving him
behind rather than reuniting his broken remnant with the Ordeal.

The Great Ordeal crossed the Wair Chirsaul, the Mandible Ford, and marched into the
choking emptiness of Agongorea. At first the march was triumphant, the surviving
members of the Ordeal filled with confidence from their survival of the Scalding at
Dagliash. But soon despair -- and Dolour -- replaced the triumph. Their Aspect-
Emperor had left them, and they had left their food behind. Each Ordealman had
scavenged as much Sranc-flesh as they could find, but soon this neared exhaustion.
102Agongorea, the Field Appalling, was lifeless. Nothing lived there, not the meanest weed
or the smallest ant. The Ordeal risked starvation\ldots{}until they saw the two golden threads
gradually taking shape on the horizon. Their destination was finally in sight.

To the south, Achamian and Mimara finally reached the Leash to find the straits
clogged with untold thousands of corpses, most of them Sranc, washed out of the
Neleöst by the Scalding. Achamian realised this meant that Kellhus had achieved a
great victory and was now marching on Golgotterath. Time grew short.

The Ordeal made the final crossing and descended into madness. Many of their horses
were eaten, and then men turned on men, first criminals and would-be deserters, and
then those who had been Scalded. To survive, the Ordeal had to feast on itself. Sorweel
and Serwa returned to find the army gripped in a fever of horror, one they had to
endure even as the Golden Horns grew taller on the horizon. The Ordeal breached the
Occlusion, the vast debris ring thrown up by Arkfall, and made camp before the Horns,
There, a semblance of normality returned to the host, but so did a sense of guilt for
what they had endured to come so far. Sorweel was horrified, his disgust for the Aspect-
Emperor returning. He gave himself up to the White Luck utterly.

Then, just as despair threatened to grip the Ordeal, Anasûrimbor Kellhus returned. He
found the Ordeal gripped by darkness so lit the way to sanity by claiming that Nersei
Proyas had betrayed them and led them to depravity on the orders of the Consult.
Proyas was struck down from his rank and imprisoned. Zsoronga of Zeüm was held to
blame for his nation's betrayal and put to death. The Great Ordeal was given scapegoats
to blame its own madness upon, and thus was absolved of its sins. When Proyas
demanded to know why, Kellhus told him a horrific truth: that in order to stand
outside the perception of time and the Gods, the Consult must at some point win and
close the World to the Outside. Whether that happened now or millennia hence was
immaterial: at some point the Consult's victory was inevitable. Against that, all
sacrifices were just.

Moënghus awoke to find himself a prisoner of Cnaiür urs Skiötha\ldots{}of his father.
Cnaiür and the Scylvendi marched on Golgotterath and Moënghus was made to march
with them.

Achamian and Mimara joined the Ordeal as it arrayed for battle before the Golden
Horns. Learning of the fate of Proyas, Achamian begged Kellhus for mercy, falling to
his knees (as long ago foretold) to ask for that boon. But Kellhus refused. For her part,
Mimara tried to behold Kellhus with the Judging Eye, as she had long ago promised,
only to find that the Eye refused to open. When it finally did, outside of Kellhus's
presence, it told her only that her mother Esmenet was truly holy, radiating with the
golden light of the undamned.

Before the final confrontation, Kellhus called his lords together for a benediction, the
Last Whelming, to prepare them for war. Sorweel came before Kellhus as had been
promised. A Chorae in his hand, the White Luck protecting him from Kellhus's
Dûnyain sight, all Sorweel had to do was strike. But the White Luck did not protect
Sorweel from Kelmomas's piercing gaze. Before he could strike down Kellhus, Sorweel
was stabbed and killed by Kelmomas. The threat was eliminated, but it could not be
seen. To Kellhus and the others, Kelmomas had murdered a true believer in the Aspect-
Emperor for no reason. Kelmomas was taken away in chains. His father counselled his
death, but his mother gave the child a file so he might free himself.

The battle came at last. Cet'ingira, who claimed to have witnessed each of the eight
thousand summers since Arkfall, bandied words with the Aspect-Emperor and then the
Anasûrimbor unleashed the first fruits of his long-ago alliance with the Daimyos. A
host of Ciphrang descended on the Ark, causing great destruction among its defenders
and triggering many of the magical traps set to destroy attackers. The Schools of the
Ordeal pelted the great walls and fortress redoubts with sorcery, and Chorae were used
to weaken the ensorcelled walls. Golgotterath was breached , the Ordeal penetrating the
walls even as the ancient First Ordeal had failed to do so.

But Golgotterath had reserves. A horde of Bashrag was unleashed from within, causing
great destruction in the ranks of the Ordeal. And now the darkest threat of all, a vast
shroud rising from the slopes of the Yimaleti Mountains: a second Sranc horde, equal
or greater to the one destroyed at Dagliash.

Kellhus visited death upon the second Horde but found its numbers so vast that his
attacks were negligible. The Ordeal's only hope was to gain the mound of Golgotterath
and then hold the broken walls against the new threat. But a fresh attack came from an
unexpected quarter: a traitor-Nonman bearing the Sun Lance, sister-weapon to the
Heron Spear, an Inchoroi Spear-of-Light from the dawn of time. Many great sorcerers
were incinerated by the Lance until Kellhus confronted the Nonman and killed him,
taking the weapon for his own. He turned the Sun Lance not against the Horde, but
against the Canted Horn of Golgotterath itself.

The Canted Horn fell, two miles or more of metal, more ancient than could be
comprehended, which had borne the Ark through the Void itself. How many Sranc
died in the titanic collapse could not be fathomed, nor the number of men whose
hearing was destroyed by the cacophony, a sound unmatched by anything in the World
since Arkfall itself.

The Intrinsic Gate of Golgotterath, and its guardian.

But the battle was not done. The Horde had lost a vast portion of its number, but more
remained, and they now assailed the host of the Ordeal. The Scylvendi descended to
take the Ordeal's camp, Moënghus putting an end to Proyas's torment. Mimara, now
deep in the throes of childbirth, had to flee into the battlefield with Achamian and
Esmenet to find succour in the Ordeal's lines. Skuthula the Black, Worm of Legend,
emerged to hold the Intrinsic Gate itself, the interface between the Ark and
Golgotterath, and there held it against all attackers.

Then Aurang took to the skies. Forced into battle as all hope for the Consult seemed to
fade, he confronted the Holy Aspect-Emperor and fell before him. The Inchoroi's
corpse was thrown from the sky and left rotting in the chaos below. Spying the gate
through which Aurang had exited the Ark, high on the Upright (and now Only) Horn,
Kellhus entered the Ark and finally confronted his destiny.

106The battle continued. Serwa confronted Skuthula the Black in a battle which would
have stood for the ages, had it not been seen by so few. Serwa avoided a hail of Consult
Chorae bowmen and finally unleashed destruction on the Worm, only for one last
Chorae that had escaped her detection to lay her low. Her brother Kayûtas found her
body and bore it from the field, but it was unclear if she was dead or alive.

The first of Mimara and Achamian's two children was born.

Inside the Ark, Kellhus found his way to the Golden Court, the heart of the Consult's
power, and there confronted Cet'ingira and Aurax, brother to Aurang, now the last
surviving member of the Inchoroi species on this World. Cet'ingira told Kellhus that
the Ark had been built by a race of Progenitors who had achieved total technological
mastery and had grasped the Absolute through it, but in doing so had discovered the
threat of damnation that waited for everyone. The Ark was built and populated by their
Tekne creations -- including the Inchoroi themselves -- and sent forth into the Void,
destroying the population of every world they encountered, reducing the numbers of
each to 144,000 as they had come to believe was necessary to seal the worlds against the
Outside. But to no avail, until the Ark arrived, exhausted, on this World, a holy and
promised place.

Cet'ingira told Kellhus of the truth, that hell and damnation waited for everyone, and
that the only way to avoid this fate was for the Consult to win . Cet'ingira bade that
Kellhus stare into the Inverse Fire and be convinced of the truth\ldots{}only for Kellhus to
be left unimpressed by it. Kellhus knew that the Inverse Fire showed the reality that
followed death and confirmed that all sentient beings, bar the few especially favoured
by the Gods, were doomed to torment, but it did not interest him. Even the growing
threat of hundreds of skin-spies, each bearing a Chorae, who had surrounded him
seemed of no interest. Cet'ingira, who had seen thousands swayed by the Fire, was left
baffled.

The Throne of Sil, in the Golden Court of the Ark of the Skies.

Kellhus now demanded that the true masters of the Consult show themselves and they
did so: five Dûnyain, the last survivors of Ishuäl. Taken prisoner during the Consult
attack a decade earlier, they had been taken back to Golgotterath for interrogation.
Inevitably, they had suborned it just as Kellhus had suborned the First Holy War.
Shaeönanra had objected and been destroyed. Even the Inchoroi had surrendered their
command of the Unholy Consult to the Dûnyain, realising that only they could
possibly stop Kellhus. Five Dûnyain to defeat one. Everything they had worked for now
led to this place, to Kellhus confronting them and then either surrendering to their
argument or being defeated. Either way, the only way out of the Golden Court lay
through the Carapace, now shorn of its protective Chorae as to admit a sorcerer.

The only way for Kellhus to leave was if he became the No-God, as his ancestor Nau-
Cayûti had before him.

Kellhus agreed this plan was sound, but it had not accounted for his own foresight in
that this could happen: that Dûnyain would survive Ishuäl, would return to
Golgotterath and conquer it. To defeat this eventuality, Kellhus therefore had to make
alliance with an even stronger power. And this he had done.

When he learned the art of the Daimyos and plumbed the hells, returning with the
Decapitants, he had also made alliance with Ajokli, Lord of Hate. A part of Ajokli now
rode in Kellhus, and through the Topos of Golgotterath, a wound in the fabric of the
World ever darker and more terrible than that of Cil-Aujas, Ajokli could now manifest.
And as he saw through Kellhus's eyes, Ajokli could now comprehend (although at a
remove) the Consult, stepping outof the circle of ignorance that had blinded the Gods
to the threat posed by them.

Kellhus-and-Ajokli killed Cet'ingira and one of the Dûnyain before wresting control of
the skin-spies as a sign of his power. Kellhus's interest was not in just stopping the
Consult, but in conquering hell itself . Kellhus would evade the threat of damnation by
destroying or enslaving the Gods himself. In return, Ajokli would have all the souls he
wanted to feast upon. The Dûnyain would act as their viceroys in this new world.

Kellhus would save the World by making of it a charnel house. But this was not to be.
Kelmomas appeared, having found his way into the Golden Court thanks to a skin-spy
infiltrator. Kellhus was so shocked to see his son where he was not expected that his
control slipped for at least a moment, the Ajokli aspect bewildered because he could
not perceive the child. This confusion lasted long enough for a skin-spy to strike
Anasûrimbor Kellhus with a Chorae. The Aspect-Emperor turned to salt, killed on the
instant. The Dûnyain seized Kelmomas and threw the son of Kellhus into the
Carapace.

The battle outside had ended, the remaining Upright Horn pulsing with power.
Kellhus apparently descended to greet his victorious troops, but it was an illusion
wrought of the Tekne. Mimara's second child was stillborn in this moment and with
109the Judging Eye she beheld the truth: the shining Carapace floating down from on
high.

The host of the Ordeal fled, streaming from the mound of Golgotterath even as the
winds began to howl and the Whirlwind took shape once again. The Ordealmen,
reduced from a third of a million souls to a paltry few tens of thousands, were
surrounded by the Sranc Horde and destroyed utterly. Thousands of Chorae flew
through the air, gathered up in their tens of thousands from the detritus of
Golgotterath and cast out by the Whirlwind to strike the fleeing sorcerers, killing them
by the hundreds. The remnants of the Great Ordeal perished in smoke and salt.

Realising he had been cheated of his vengeance, Cnaiür gave himself over to Ajokli,
who was screaming in rage as Kellhus's soul evaded him, and then to the Whirlwind
itself. He left behind a command that his son Moënghus would succeed him as King-of-
Tribes of the Scylvendi, to command them in the war to come.

Out of the darkness a tiny band fled, flying high over the Sranc Horde in the opposite
direction to the rest of the host. The ancient Wizard, his wife, his son and his old lover
fled before the horror that had been unleashed by the Aspect-Emperor's miscalculation.

The No-God had returned. The Second Apocalypse had begun.

\hypertarget{appendices}{%
\chapter{Appendices}\label{appendices}}

\hypertarget{appendix-i-house-anasuxfbrimbor}{%
\section{Appendix I: House Anasûrimbor}\label{appendix-i-house-anasuxfbrimbor}}

The House of Anasûrimbor is the oldest dynasty of human nobility, spanning almost
four thousand years of history. The house originated in the city of Ûmerau, during the
rise of the civilisation of the Ancient North, centred on the Aumris River Valley. By the
8 th Century after the Breaking of the Gates the house had become one of the primary
noble houses in the Ûmeric Empire.

\textbf{House Anasûrimbor of Ûmerau}

Anasûrimbor Sanna-Neorjë (772-858), noble of Ûmerau, father of Omindalea.

Anasûrimbor Omindalea (808-825), daughter of Sanna-Neorjë. Raped by the
Nonman Jiricet, thus ending the Nonman Tutelage. She died in childbirth,
giving birth to Sanna-Jephera.

Anasûrimbor Sanna-Jephera (825-1032), the half-Nonman son of Omindalaea. He
was made Sanna-Neorjë's heir. As half-Nonman, he and his descendants
enjoyed a longer lifespan than the human norm.

\textbf{House Anasûrimbor of Aörsi}

Anasûrimbor Nanor-Ukkerja I (1378-1556), the first Anasûrimbor High King,
founder of Kûniüri (in 1408). Upon his death he divided his empire in three,
founding Aörsi and Sheneor in addition to Kûniüri proper.

Anasûrimbor Mygella (2065-2111), High King of Aörsi, the Hero-King whose
deeds are recounted in The Sagas.

Anasurimbor Nimeric (2092-2135), the son and heir of Mygella, High King of Aörsi
during the First Apocalypse. He was killed at the Battle of Hamuir, just before
the fall of Aörsi.

\textbf{House Anasûrimbor of Kûniüri}

Anasûrimbor Nanor-Ukkerja I (1378-1556), the first Anasûrimbor High King,
founder of Kûniüri (in 1408). Upon his death, he divided his empire in three,
founding Aörsi and Sheneor in addition to Kûniüri proper.

Anasûrimbor Celmomas II (2089-2147), High King of Kûniüri, leader of the First
Ordeal and the leader of the forces opposed to the Consult in the opening years
of the Apocalypse. He was killed at the Battle of Eleneöt Fields. As he died, he
had a vision of Gilgaöl showing him a descendant of the house would return at
the end of the world, a prophecy he passed to his friend and ally Seswatha.

Anasûrimbor Nau-Cayûti (2119-2142), the youngest son of Anasûrimbor
Celmomas II and his wife Suriala, although some believe that he was the son of
Suriala by Seswatha. A great hero of the First Apocalypse whose reputation was
made at the Battle of Ossirish, where he killed the dragon Tanhafut the Red
and routed the Consult. His favourite concubine and the love of his life, Aulisi,
was kidnapped by the Consult and borne to Golgotterath, possibly to
demoralise Nau-Cayûti or provoke him into attacking the Ark precipitously.
Instead, he allied with Seswatha and, using intelligence gained in Ishterebinth
through the Amiolas, they mounted a two-man raid on the Ark. They did not
find Aulisi but did locate the Heron Spear, Seswatha admitting that he
deliberately tricked Nau-Cayûti to this end. After returning home, Nau-Cayûti
was poisoned by his wife Ieva. According to history, he died in 2140 and was
buried, rather than being cremated. In reality, the poison left him alive but
immobile. He was excavated by the Inchoroi Aurang and taken to Golgotterath.
He resisted two years of torture before being placed in the Carapace, a Tekne
112artifact. Shortly after this, the Initiation of the No-God took place. The causal
link between the two events remains unclear.

\textbf{House Anasûrimbor of Ishuäl}

Anasûrimbor Ganrelka II (2104-2147), the heir of Celmomas II. He was believed
killed at the Battle of Eleneöt Fields, but in reality, he survived and sought
sanctuary in Ishuäl, a hidden redoubt in the Demua Mountains. He died of
plague within months of arriving, as did all of his court save his bastard son.

Anasûrimbor ? (c. 2137-?), the unknown bastard son of Celmomas II and his sole
heir. He survived the plague that consumed Ishuäl and welcomed the Dûnyain
to the stronghold.

Anasûrimbor Moënghus (c. 4050-4112), a Dûnyain monk who was sent out into the
world to investigate after the stronghold was discovered by Sranc in 4079.
Moënghus was exiled, eventually making the acquaintance of Cnaiür urs
Skiötha of the Scylvendi. After manipulating Cnaiür like a child, to his fury,
Moënghus escaped south and joined the Cishaurim of Kian. In 4100 he
captured and tortured a Consult skin-spy, learning of the Consult's plan to
bring about the Second Apocalypse. Moënghus created the Thousandfold
Thought, an intellectual weapon to oppose the resurrection of the No-God. He
used a Cant of Calling to send for his son before initiating the Holy War as a
mechanism to bring him to Shimeh. When Kellhus reached Moënghus and told
the story of the Holy War, Moënghus realised his son had gone insane. Kellhus
stabbed his father and left him to die; however, it was left to Cnaiür to execute
Moënghus with a Chorae.

Anasûrimbor Kellhus (4076-4132), a Dûnyain monk. He was summoned into the
outside world by his father to assist him in his secret war against the Consult,
but his superiors ordered him to find and execute Moënghus instead. Kellhus
113joined the Holy War, ultimately subverting it to his cause by the time it
captured Shimeh in 4112. He learned of the Consult and the No-God and
made it his mission to destroy the organisation. However, his father believed he
had gone mad. Kellhus stabbed his father and defeated the last Cishaurim at
Shimeh. He was proclaimed Aspect-Emperor of the Three Seas in 4112, but
faced ten years of warfare before being able to finally secure the Three Seas. He
then turned the Three Seas into a machine to build the greatest amassing of
armed and sorcerous might ever seen: the Great Ordeal. In 4132 the Great
Ordeal marched against Golgotterath and was destroyed. It was revealed that
Kellhus had allied with the God Ajokli to conquer the hells and destroy the
Consult, but he was betrayed by his son Kelmomas and slain with a Chorae. In
the aftermath of his death, Ajokli railed because he was unable to find Kellhus's
soul and feast upon it as he desired.

Maithanet (c. 4085-4132), the son of Anasûrimbor Moënghus. Trained in Dûnyain
ways by his father, he infiltrated the Thousand Temples and rose to the rank of
Shriah at a young age. On his father's instruction, he declared the Holy War
against the Fanim. In the closing weeks of the war he forged an unprecedented
alliance between the Thousand Temples and the Mandate school before
recognising Maithanet's half-brother Kellhus as Aspect-Emperor. Their mutual
support allowed Kellhus to win the Unification Wars. Maithanet remained a
close supporter of Kellhus, but became concerned that Kellhus was letting
emotion for his wife and children cloud his judgement. He was executed by the
White-Luck Warrior on the order of the Empress Esmenet in 4132.

Anasûrimbor Koringhus (c. 4105-4132), Anasûrimbor Kellhus's first son, born from a
``whale-mother'' in Ishuäl ere Kellhus's departure. Known as ``the Survivor'', he
survived the ruination of Ishuäl by the Consult in great secret before being
rescued by Drusas Achamian and Anasûrimbor Mimara. Driven insane by his
experiences, he killed himself by hurling himself from the mountains.

``The Child'' (c. 4122- ), the son of the Survivor, born in Ishuäl. Cunning and
114ruthless, he survived the Sack of Ishuäl by the Consult before being rescued by
Drusas Achamian and Anasûrimbor Mimara. After being captured by the
Scylvendi under Cnaiür urs Skiötha, he fled into the night and escaped. His
current whereabouts are unknown.

\textbf{House Anasûrimbor of the New Empire}

Anasûrimbor Kellhus (4076-4132), Aspect-Emperor of the Three Seas (proclaimed
4112, crowned 4122). In 4132 the Great Ordeal marched against Golgotterath,
led by Kellhus, and was destroyed utterly.

Anasûrimbor Esmenet (4078- ), Empress of the Three Seas (proclaimed 4112,
crowned 4122), the wife of Anasûrimbor Kellhus, whom she met in 4110,
during the Holy War. The former lover of Drusas Achamian. She survived both
the downfall of Momemn and the New Empire, and the battle at Golgotterath.
Anasûrimbor Kayûtas (4112- ), eldest son of Kellhus and Esmenet, general of the
Kidruhil. Serving in the Great Ordeal. It is believed he survived the final battle
at Golgotterath.

Anasûrimbor Theliopa (4114-4132), eldest daughter of Kellhus and Esmenet, born
in the White-Sun Palace in Nenciphon after the fall of Kian, during the
Unification Wars. Slightly unhinged from her intelligence, she remained
alongside her mother as an advisor. She was killed by the machinations of the
White-Luck Warrior in Momemn.

Anasûrimbor Serwa (4115- ), daughter of Kellhus and Esmenet. Noted for her
formidable grasp of sorcery. Taught in the Metagnosis by her father, Serwa
because Grandmistress of the Swayal Compact as a teenager. Her sorcerous
skills are utterly formidable; aside from her father, she may be the most
powerful human sorcerer to have lived since Titirga himself, three thousand
115years earlier. Serving in the Great Ordeal. She was severely wounded by both
dragonfire and Chorae during the final battle at Golgotterath. Her fate is
unknown.

Anasûrimbor Inrilatas (4117-4132), son of Kellhus and Esmenet. He inherited his
father's skills for analysis and manipulation, but he was also insane, unable to
use reasoned judgement. He attempted to kill his uncle Maithanet and was
killed in self-defence.

Anasûrimbor Samarmas (4124-4132), son of Kellhus and Esmenet. Twin of
Kelmomas. A quiet boy, described by his brother as his ``idiot twin''. Samarmas
was killed by the machinations of his brother. After his death, Kelmomas could
hear his twin speaking to him in his mind.

Anasûrimbor Kelmomas (4124- ), youngest surviving son of Kellhus and Esmenet.
Noted for his utter cunning, ruthlessness and obsession with retaining the love
of his mother. Highly distrustful of his father and his siblings. Murderous and
possibly insane. At the final battle at Golgotterath, Kelmomas betrayed his
father and brought about his death by Chorae. Kelmomas was then fed into the
Carapace and became the second incarnation of the No-God.

Anasûrimbor Mimara (4095- ), the eldest child of Esmenet, born to her by an
unknown father. During a time of famine in Sumna, Esmenet sorrowfully gave
her away so she would survive. After becoming Empress, Esmenet tracked
Mimara down and she was adopted into the House of Anasûrimbor. However,
Mimara was unable to adjust to life in the Imperial Family. Her mother refused
for her to learn sorcery and, aided by insinuations by Kelmomas, she fled the
Andiamine Heights. She sought the aid of Drusas Achamian, whom she later
seduced and became pregnant by him. She wields the Judging Eye, a powerful
and mysterious ability to discern who is good and who is evil. She survived the
final battle at Golgotterath, bearing one living child and one dead one.

Anasûrimbor Moënghus II (4111- ), the son of Cnaiür urs Skiötha and Serwë, born
in Caraskand during the Holy War. He was adopted by Esmenet after his
mother's execution and then by Kellhus following the end of the war. Raised as
the oldest of the Prince Imperials, although he does not have any of Kellhus's
blood. He marched in the Great Ordeal but was later taken prisoner by Cnaiür
and commanded to succeed him as commander of the Scylvendi in the Second
Apocalypse.

PIC

\hypertarget{appendix-ii-the-schools-of-sorcery}{%
\section{Appendix II: The Schools of Sorcery}\label{appendix-ii-the-schools-of-sorcery}}

``Sorcery'' or ``magic'' is the term given to the ability of a small number of people
(Cûnuroi and humans both) to manipulate the fabric and energies of reality through
the use of song-like incantations, stringing together phrases and words with magical
effects. These individuals are known as the Few, for their numbers are tiny in
comparison to the majority of the population, although the ability is more common
among Cûnuroi than humans. Sorcery, however, differs in its practical effects and how
it is wielded.

At heart, all sorcery requires the wielder to be one of the Few, to be able to perceive the
onta or very fabric of reality. They then utter audible cants (the utteral string) whilst
focusing the power with a thought (the inutteral string). These forces allow the World
to be physically shaped by the wielder into a desired outcome: desire trumps reality on a
local scale, in defiance of orthodox logic.

The first sorcerers were the Cûnuroi, the Nonmen. According to their tradition, they
were gifted the art by their god-founder, Imimorûl. The Nonmen Qûya, their mage-
caste, practised several forms of magic. The most frowned upon -- and eventually
outlawed -- was the Aporos, the art of negating sorcerous effects and destroying the
118Few. The most common and the most powerful was the Gnosis, based in reason, logic,
mathematics and knowledge.

The Inchoroi underestimated the power of the Gnosis, a weakness exposed at Pir Pahal
when they relied too much on their spears of light and other Tekne weapons. Although
formidable, they were outmatched by the Gnosis and the battle was lost. To level the
field, they seduced the outlawed students of the Aporos and convinced them to create
weapons based on negation . The results were the Chorae, small spheres which
destroyed the Few on contact and rendered the wielder immune to sorcery. The Chorae
were a formidable force-equaliser and helped win the great victory of Pir Minginnial but
were not sufficient to completely destroy the Nonmen; later in the Cûno-Inchoroi
Wars, the Nonmen adapted to their presence and won the great Battle of Isal'imial
despite the Inchoroi deploying significant numbers of Chorae against them. In later
years the Inchoroi grafted the ability to use sorcery onto themselves, but only a few
survived the process.

In Eänna beyond the Great Kayarsus, men developed their own sorcery but it was a
wild thing. Originally the role of sorcerer and prophet was combined into one, that of
Shaman. Only the Shamans could use sorcery, but it was a primitive approach, based
on superstition and ritual. This was the forerunner of the Anagogis. The Shamans were
displaced by the Prophets, non-sorcerous religious leaders who proclaimed themselves
servants of the Hundred Gods (who sent miracles and evidence of the divine, or
infernal, to show their favour or displeasure). Sorcery was outlawed and the Few driven
to the margins of society as witches and wizards even before The Chronicles of the Tusk
were compiled (with Inchoroi interference, including the command to enter Eärwa and
kill the False Men). When the Inchoroi gifted the Tusk to men, they also gave them
many Chorae, which further helped isolate the Few.

However, the ban on sorcery could not survive the initial clash with the Cûnuroi at the
Gates of Thayant. The howling masses of the Four Tribes far outstripped the defending
Nonmen in numbers, but the Cûnuroi could unleash the Gnosis on a massive scale,
scything down the attackers in their hundreds or thousands. Eventually male sorcerers
119were permitted to fight (but never women, who were held to be witches and anathema),
first solely as Wizards and later as organised Schools. The Anagogis was far inferior to
the Gnosis in strength, but the numbers of human wielders of the Anagogis far
outstripped those of the Gnosis. In combination with Chorae archers, they finally
overthrew the Qûya at the gates, smashed them down and swarmed into Siöl,
destroying what had once been the greatest of the Mansions.

The Nonmen were overthrown in four centuries of warfare, but eventually a great truce
was called in the Ancient North. Pioneered by the Nonmen of Ishoriöl and the men of
the Aumris River Valley, the two sides struck up an alliance. This led to the Nonmen
Tutelage, when the Nonmen gave men the secrets of the Gnosis and the founding of
the Gnostic mage-schools, most formidably the Mangaecca and the Sohonc.

During the First Apocalypse, the Gnostic schools of the North (then numbering a
dozen or so) were utterly destroyed, leaving behind only a bare handful of Gnostic
sorcerers. They founded the School of Mandate to preserve knowledge of the Gnosis,
but also to guard it against the jealous Anagogis schools of the Three Seas.
Almost as notable was the Iswazi, a form of magic rooted in the use of totemic items.
Held by some to be a derivation of the Anagogis and others to be a wholly unique form
of magic, the Iswazi gained favour in High Holy Zeüm and became the dominant form
of magic in that far land. The Iswazi is held to be superior to the Anagogis in individual
combat but inferior in massed battle, offering slightly greater raw strength but less
flexibility (due to the need for physical totems, which can be destroyed or removed).

Rarest of all, and believed to be a myth until the Holy War, was the Metagnosis. A
superior form of the Gnosis which uses three strings (one utteral, two inutteral) rather
than two, the Metagnosis can string together different phrases in new ways to create
vastly more formidable effects. Most notable of these is the ability to translocate (or
``teleport'') from one location to another within a line of sight (thus limited to the
horizon line), which is simply impossible otherwise. Only four practitioners of the
Metagnosis are known to history: Su'juroit, the Nonman Witch-King; Anasûrimbor
120Kellhus; his daughter, Anasûrimbor Serwa; and Apperens Saccarees, head of the School
of Mandate (although he is reported to have only uttered one of the Metagnostic cants).

One other form of sorcery was developed by the Fanim of Kian. The Psûkhe was a form
of sorcery completely alien to both the Gnosis and Anagogis. The Psûkhe is rooted in
emotion, passion and feeling, rather than logic (the Gnosis) or ritual (Anagogis).
Practitioners of the Psûkhe blinded themselves to better focus on feeling rather than
analysis. The Psûkhe was accounted superior to the Anagogis in strength but inferior to
the Gnosis.

\textbf{Anagogic Schools}

The Anagogic schools were founded during the Cûno-Halaroi Wars, borne out of
necessity and the need to check the power of the Nonmen Qûya despite the scriptural
condemnation of sorcerers. Although the Anagogis was far inferior to the Nonmen
Gnosis in raw power, the sheer number of sorcerers that the invaders could field more
than levelled the field.

The Anagogis has derivative branches, one of which is the Daimos, the art of
communing with and controlling Ciphrang. The Daimotic arts are highly dangerous
and proscribed by the Schools, but the Scarlet Spires (among others) are known to have
explored this form of magic in defiance of this tradition. Practitioners of the Gnosis
have also flirted with this form of magic (the Cûnuroi describing it as ``summoning
Agencies''), but largely rejected it as too dangerous, with the summoned entity
frequently turning on its captor with violent results.

The Surartu (Scarlet Spires). A school of ``hooded singers'', the Surartu banded together
in the Three Seas for mutual protection in the face of religious persecution. They
became a player in the politics of the Shiradi Empire, but their influence increased
markedly after the collapse of Shir during the Apocalypse. Around 2350 the rebuilt
their fortress of Kiz in Carythusal with red tiles, leading to their renaming. Eventually
121the Scarlet Spires became the ruling power of High Ainon. During the Holy War, they
fought both the Cishaurim and the Mandate, with whom they had a longstanding
dispute over control of the Gnosis. The Scarlet Spires were almost obliterated at the
Battle of Shimeh, emerging with only a dozen sorcerers-of-rank still alive. To their
humiliation, they lost control of High Ainon during the Unification Wars and it fell to
the Mandate to reassert control of the kingdom in the name of the Aspect-Emperor.

The Myunsai, the Mercenary School. The largest of the Anagogic Schools, the Myunsai
were founded in 3804 from the amalgamation of three minor schools during the
Scholastic Wars (the Mikka Council of Cironj, the Oarant from Nilnamesh and the
Nilitar Compact of Ce Tydonn). The Myunsai sided with the Inrithi during their
invasion of High Ainon, an act of startling hypocrisy from the Inrithi (who were
fighting to slaughter the damned sorcerers) and self-defeating stupidity from the
Myunsai. The Myunsai are noted as the Mercenary School, lending their services out
for coin.

The Saka, the Imperial Saik (Sorcerers of the Sun). Originally the state-sanctioned
school of the Cenei Empire, which dominated the Three Seas for a thousand years in
alliance with the Aspect-Emperors, the school survived the fall of the empire and
reconstituted itself as the Imperial Saik of the Nansur Empire. One of the larger
schools, the Imperial Saik's numbers have been reduced by constant skirmishing with
the Cishaurim. During the Holy War the Imperial Saik suffered significant losses, but
later swore allegiance to Kellhus (alongside the Scarlet Spires and the
Mandate) and
became the core of his magical strength during the Unification Wars.
The Vokalati (Sun-Wailers). A school of sorcerers hailing from distant Nilnamesh who
have lusted for the Gnosis for two thousand years but constantly failed to seize it. Their
numbers were depleted during the Unification Wars but now they serve Anasûrimbor
Kellhus.

\textbf{Iswazi Schools}

A derivation of the Anagogis, according to some, based around the manipulation of
forces channelled through totemic artifacts. This branch of sorcery failed to find much
favour in the Three Seas or Ancient North, but did become popular in Angka (later
Zeüm).

The Mbimayu. The major school of Zeüm, based in Domyot, serving the Satakhan in a
similar manner to how the Imperial Saik served the Emperor of Nansur.

\textbf{Gnostic Schools}

Four Gnostic schools were founded during the Nonmen Tutelage; they were later
joined by others, so that by the time of the Apocalypse around a dozen Gnostic schools
were in existence. All bar the Sohonc and the Mangaecca were destroyed during the
Apocalypse (although the latter had long been subsumed into the Consult); the Sohonc
were effectively refounded as the Mandate by Seswatha.

\textbf{The Sohonc}. The foremost and largest of the original four Gnostic schools, founded
during the Nonmen Tutelage. The Sohonc was formidable in strength and knowledge.
Its most famed grandmasters were Noshainrau (who appears to have refounded the
school in a different form circa 1050), and his successor Titirga, the Glorious Pupil and
most powerful sorcerer the World has reportedly ever seen. Its final grandmaster was
Seswatha, who led it during the First Apocalypse. Fully one third of the Sohonc were
massacred by the Chorae Hail during the First Investiture, a blunder laid at Seswatha's
feet despite him acting on the orders of Anasûrimbor Celmomas II. Only a tiny
handful of Sohonc survived the Apocalypse and were later reconstituted as the
Mandate.

\textbf{The Mihtrûl, the School of Contrivers}. Founded in 661 by the Nonman Emilidis, the
Artisan. This school invested heavily in the study and creation of sorcerous artifacts.
The Mihtrûl were destroyed during the First Apocalypse.

\textbf{The Mangaecca}. Founded in 684 by Sos-Praniura (a student of Gin'yursis), the
Mangaecca was one of the original four Gnostic Schools. The Mangaecca hungered for
knowledge, secrets and historical truth. They were corrupted by the Nonman Erratic
known as Cet'ingira, who made alliance with them in 777. With his help, they
discovered the Incû-Holoinas and constructed the fortress of Golgotterath. Under their
Grandmaster Shaeonanra, the Mangaecca finally gained the Ark and discovered the
Inverse Fire and the truth of their damnation. They made alliance with the Inchoroi,
founding the Unholy Consult circa 1111. The Mangaecca effectively ceased to exist at
that point and became the Consult.

\textbf{The Mandate}. Founded in 2156 by Seswatha and the few surviving Sohonc to continue
the war against the Consult and prevent the advent of the Second Apocalypse. Upon
his death Seswatha bequeathed his memories of the Apocalypse to all members of the
Mandate via the sorcerous ritual known as the Grasping. Based in Atyersus, the
Mandate protected the Gnosis for two thousand years until Drusas Achamian agreed to
teach Anasûrimbor Kellhus, who in turn commanded them to share their knowledge
with the Swayal Compact.

\textbf{The Swayal Compact, the School of Witches}. Founded after 4114 at the order of
Anasûrimbor Kellhus. He rescinded all restrictions (spiritual, moral and legal) based on
gender, allowing women to join the Few. The Swayal Compact was established as a
school for these witches , to the shock of the Three Seas. Kellhus also commanded the
Mandate to share their knowledge of the Gnosis with the witches, to their reluctance.
Within just ten years, the Swayal sisters had matched the Mandate in skill and far
outstripped them in numbers, becoming the largest school in the Three Seas. At the
time of the Great Ordeal, they were led by the seventeen-year-old Anasûrimbor Serwa.
However, she earned this position on merit, due to her mastery of the Metagnosis.

\textbf{Psûkari Schools}

The Psûkhe is a form of sorcery not known to exist to either Cûnuroi or humanity
prior to the 38 th Century. However, odd reports and legends surrounding those
sorcerers blinded or who had troubles with their eyesight. The Sohonc Grandmaster
Titirga, for example, had issues with his vision which have led some to believe he may
have been able to grasp the Psûkhe in a primitive fashion. This would explain both the
relative clarity of his mark and also his immense sorcerous powers, otherwise
inexplicable for someone who had not grasped the Metagnosis.

The Cishaurim. Founded by the Prophet Fane circa 3705. Fane had been an Inrithi
priest sentenced to die in the Carathay Desert. He went blind but gained the power of
the Water of Indara, a form of sorcery rooted in passion. The Cishaurim were all but
obliterated during the Holy War, the majority of their number (including their effective
Grandmaster, Seökti) slain by Anasûrimbor Kellhus at the Battle of Shimeh in 4112.
Only Meppa, the most powerful of the Cishaurim, survived. Meppa was defeated in
combat with Kellhus in 4132 and presumably killed, although this is not altogether
certain.

\textbf{Known Cants}

Here follows a list of the names of the known cants of the Schools.

\textbf{Anagogic Cants}

Dragonhead
Gotaggan Scythes
Houlari Twin-Tempests
Memkotic Furies
Ramparts of Ur

\textbf{Gnostic Cants}

Bar of Heaven
Bisecting Planes of Mirseor
Cant of Sidewise Stepping
Cants of Torment
Compass of Noshainrau
Cirroi Loom
Ellipses of Thosolankis
Huiritic Ring
Ishra Discursia
Mathesis Pin
Ninth Merotic
Noviratic Spike
Odaini Concussion Cant
Seventh Quyan Theorem
Skin Ward
Surillic Point
Thawa Ligatures
Weära Comb

\textbf{Metagnostic Cants}

Cant of Translocation

\textbf{Mbimayu Cants}

Iswazi Cant
Muzzû Chalice

\hypertarget{appendix-iii-timeline}{%
\section{Appendix III: Timeline}\label{appendix-iii-timeline}}

\textbf{The First Age}

Due to a Cûnuroi lack of interest in linear chronology, no reliable calendar systems
exist during the First Age and events can only be inferred by their relationship to one
another. The principle event of this age is the Cûno-Inchoroi Wars, which some hold
began with the Arkfall but others contend began only with the Womb Plague.

The primary source of information for the First Age is the Isûphiryas , the Cûnuroi
living history, as well as first-hand accounts of these events from those Intact Nonmen
who witnessed them and still live.

Nonman Prehistory: Imimorûl is banished from the heavens and exiled to the World.
He founds Cûnuroi civilisation. Siöl and Nihrimsûl are founded. Imimorûl dies and is
succeeded by Tsonos. Siöl founds Ishoriöl, Viri, Cil-Aujas and Illisserû; they in turn
found Curunq, Cil-Aumûl and Incissal.

\emph{Millennia pass.}

Thousand Year Siege: Siöl and Nihrimsûl clash over interpretations of their mutual
history and fight numerous conflicts. Eventually, an uneasy peace develops. Halaroi --
humans -- appear in Eärwa and Eänna. The Cûnuroi enslave many of the Halaroi of
Eärwa for their own ends.

Millennia pass.

The Succession: Morimhira, heir to the throne of Siöl, rejects the Seal of the House
Primordial. His younger brother Cu'huriol becomes High King in his stead. Cu'huriol's
children, Cet'moyol and Linqirû, engage in incest, resulting in the birth of Cû'jara
Cinmoi. Cu'huriol executes his incestuous children but holds the child blameless,
naming him as his own heir. Upon Cu'huriol's death, Cû'jara Cinmoi becomes High
King of Siöl. Shortly after taking the Seal, he gains a new name: the Tyrant.

Decades pass.

Arkfall: The Incû-Holoinas, the Ark of the Skies, crashes into the western Reach of
Viri, reducing the lands of the south-western Yimaleti Mountains to a blasted
wilderness, Agongorea. Viri is devastated and calls for aid. Cû'jara Cinmoi agrees to
relieve Viri in return for its submission; King Nin'janjin reluctantly agrees. First
Contact is made with the Inchoroi, the ``People of Emptiness'' of the Ark. Cû'jara
Cinmoi is repelled by them and executes their emissaries. The First Watch is placed on
the Ark.

Years pass.

Siölan Expansion: Siöl conquers Nihrimsûl and Cil-Aujas after protracted campaigns.

Years pass.

The Battle of Pir Pahal: An Inchoroi embassy reaches Nin'janjin and offers him an
alliance to defeat Siöl. He agrees. The First Battle of the Ark; the First Watch is
128overthrown and the Siege of the Incû-Holoinas is lifted. The army of Viri joins that of
the Inchoroi under their dread King Sil on the Field of Pir Pahal, but the sight of the
Inchoroi revolts the Cûnuroi of Viri; they repudiate Nin'janjin, who joins the Inchoroi.
The Inchoroi turn on the Viri and nearly destroy them, but they are relieved by Cû'jara
Cinmoi's forces. Cû'jara Cinmoi wins the battle, killing Sil and wresting the Heron
Spear from his grasp. The Inchoroi fleet in terror back to the Ark. The Second Watch
is placed on the Ark, but Cû'jara Cinmoi is prevented from attacking the Ark by news
of rebellions in Cil-Aujas and Nihrimsûl.

Years pass.

Siöl-Nihrimsûl Wars: Siöl retakes Cil-Aujas but Nihrimsûl refuses to capitulate. The
battles of Ciphara and Hilcyri are fought, along with the Siege of Asargoi. King
Sin'niroiha of Nihrimsûl sues for peace on the grounds of mutual respect, but Cû'jara
Cinmoi rejects the notion. Emilidis, the Artisan of Ishoriöl, creates the first of his
Sublime Contrivances, the Diurnal or Day Lantern, and gifts it to Sin'niroiha.
Sin'niroiha uses the artifact to win the allegiance of Ishoriöl and the hand of Tsinirû in
marriage. Hearing this news, Cû'jara Cinmoi sues for peace and ends the war.
Nil'giccas, son of Sin'niroiha and Tsinirû, is born.

\emph{Decades pass}. Cû'jara Cinmoi was at the height of his power when the Arkfall took
place, but by the time of the Inoculation he had grown aged and stooped. Given the
approximate lifespan of Nonmen (400 years), this suggests that between 200 and 300
years passed between Arkfall and the Inoculation.

The Inoculation: Nin'janjin, Traitor-King of Viri, emerges from the Incû-Holoinas and
begs for parley. He is taken to Cû'jara Cinmoi, now old and failing, who is amazed to
see that Nin'janjin has not aged a day since Pir Pahal. Nin'janjin tells of the surviving
Inchoroi dwelling in misery in the besieged Ark. He begs for peace and says the
Inchoroi will pay any price. Cû'jara Cinmoi demands the price of immortality, that the
Inchoroi give to the entire Cûnuroi race the gift they have given Nin'janjin. The
Inchoroi comply; the Second Watch is dismantled and the Inchoroi move among the
129Nonmen as their physicians. Cû'jara Cinmoi allows himself to receive the Inoculation
against death first. Many Nonmen view the treaty with disgust and suspicion, but as
they see their fellows becoming young and hale again, so they give in. Of the Ishroi
noble caste, only Sin'niroiha refuses the treatment. During this time, the Inchoroi win
a secret alliance with the Qûya mages practising the Aporos, the sorcerous art of
negation, and convince them to create the Chorae, the magic-destroying Tears of God.

\emph{One hundred years pass}.

The Womb Plague: Over one century after the Inoculation begins, the Womb Plague
consumes the Cûnuroi species. Every female to have received the Inoculation dies;
some of the aged males die as well. The Inchoroi evacuate Eärwa, falling back on the
Incû-Holoinas. Realising the depth of the Inchoroi betrayal, Cû'jara Cinmoi declares
bloody vengeance against them. He summons the armed might of all of the Nine
Mansions and marches on the Ark. According to some chronicles, the Cûno-Inchoroi
Wars (lasting over five centuries) begin with this event.

Battle of Pir Minginnial: The Second Battle of the Ark. Cû'jara Cinmoi's army of thirty
thousand assaults the Ark and is defeated. The Inchoroi deploy Chorae, Wracu,
Bashrags and Sranc for the first time in battle, wrecking a bloody slaughter amongst the
attacking force. Cû'jara Cinmoi is slain by Nin'janjin. Sin'niroiha manages to withdraw
the army before its destruction.

Sieges of Ishoriöl: Ishoriöl is besieged five times, each for more than a decade. During
the Siege of the Second Delve, Sin'niroiha finally succumbed to old age (the last
Nonman to die of simple mortality) and was succeeded by his son Nil'giccas. As a Son
of Tsonos through his mother's line, Nil'giccas was able to rally the Nonmen as his
father could not.

Battle of Imogirion: The southern Mansion of Illisserû mounts a surprise attack on the
Ark by sea. The battle turns into a fiasco, the army destroyed and the remnants
130attacked by Sranc on Agongorea in a nocturnal slaughter. Only one Illisserû warrior
lives to see home.

Battle of Isal'imial: The Third Battle of the Ark. Five centuries after the debacle of Pir
Minginnial and spurred by reports of the failing Inchoroi weapon systems, Nil'giccas
convinces the Mansions to mount a massive assault on the Incû-Holoinas. They agree
and a hard-pitched battle is won on the Black Furnace Plain. The victory is so complete
that the Sranc are sent reeling back into the Yimaleti Mountains, with the Cûnuroi
hunting them to near extinction, whilst the Inchoroi are (apparently) completely
exterminated. For the first and only time in history, the Incû-Holoinas is stormed by an
attacking force. Nin'janjin is taken prisoner.

The Cleansing of the Ark: For twenty years, the Cûnuroi explore, burn out and seek to
destroy the Ark, but are unsuccessful. They partially loot the Ark, using material from it
to reinforce the gates of Ishoriöl and also to seal all but one portal into the vessel. The
Golden Room and the Inverse Fire are discovered, driving all who attempt to
comprehend it insane. At this time Cet'ingira beholds the Fire and becomes convinced
of the righteousness of the Inchoroi's cause, but manages to hide his conversion.
Deeply troubled, Nil'giccas has the Ark evacuated and sealed. He commands Emilidis
to raise a glamour about the Ark, rendering it inaccessible. Emilidis complies, creating
the Barricades. All mention of the Inverse Fire is stricken from the Isûphiryas .
Nin'janjin vanishes from history, but is presumably executed.

Millennia pass.

The Time of the Shamans: In Eänna, the sorcerer-priests known as Shamans arise
among the Halaroi tribes. Using the magic of the Anagogis, they guide the tribes of
men to greater wisdom in their worship of the God-of-Gods, but in time are opposed by
the Prophets, non-sorcerous servants of the Hundred Gods. The Hundred Gods,
through their Prophets, declare that sorcerers are damned to hell. Religious
proscriptions on sorcery are commanded and the Shamans are hounded to the fringes
of civilisation, reduced to solitary witches and wizards surviving in secret or the
131forbearance of close allies. The Kiünnat religious tradition becomes dominant,
reducing the God-of-Gods to a divine placeholder and extolling the Hundred.

Centuries pass.

The Deliverance of the Tusk: The Chronicle of the Tusk is written: the great stories
and myths of humankind inscribed on the huge tusk of a monstrous, long-dead animal.
According to later history, The Chronicle was written in Eänna by the Prophets; in
reality, it was gifted to the Five Tribes -- along with Chorae -- by strangers out of the
west. Notable in The Tusk is a command to ``kill the False Men'' of Eärwa.

Centuries pass.

The Summoning of the Tribes: Angeshraël, the Burned Prophet, has a profound
revelation when encountering Husyelt, God of the Hunt, at the foot of Mount Eshki.
He summons representatives of the Five Tribes to Mount Kinsureah. He urges them to
fulfil the command of the Tusk to exterminate the False Men. In the end four of the
Five Tribes -- the Norsirai, Satyothi, Scylvendi and Ketyai -- hearken to the call, with
only the Xiuhianni resisting.

Years pass.

The Testing of the Gates: The first assaults are made on the Gate of Thayant, the
Cûnuroi fortification in the northern Great Kayarsus Mountains which bars the largest
pass linking Eärwa and Eänna. These assaults are unsuccessful, with the tribes unable to
overcome the Qûya of Siöl whose Gnostic magic is too powerful. After lengthy religious
and secular arguments, the religious proscription on sorcery is relaxed; human wizards
and warlocks are recruited to help check the Cûnuroi Qûya. The first schools of sorcery
are established.

Years pass.

The Breaking of the Gates: With the help of sorcery, the Four Tribes shatter the Gate
of Thayant and destroy the Nonman Mansion of Siöl. They spill from its great gates
onto the plains of far north-eastern Eärwa, their promised land. The Cûno-Halaroi
Wars begin. Recorded history begins.

\textbf{The Second Age}

0: The Breaking of the Gates and the beginning of the Cûno-Halaroi Wars, a
series of military conflicts between Man and Nonman lasting over 300 years.
Also, the beginning of the Age of Bronze, or Far Antiquity.

\begin{enumerate}
\def\labelenumi{\alph{enumi}.}
\setcounter{enumi}{2}
\item
  300: By this time the Jiünati Steppe has been settled by the Scylvendi and Angka has
  been settled by the Satyothi. By this time Trysë, Sauglish, Etrithatta (sometimes
  called just Etrith) and Ûmerau have been established along the River Aumris by
  the Norsirai tribe. Tentative trade between the river valley cities and the
  Nonmen of Ishoriöl (Ishterebinth) have begun.
\item
  350: Cûnwerishau, the God-King of Trysë, unites the River Aumris cities. He and
  Nil'giccas, King of Ishterebinth, sign a treaty. Effective end of the Cûno-Halaroi
  Wars; Ishterebinth and Cil-Aujas are the sole High Mansions to survive. The
  first copy of the Isûphiryas is given to humans as part of the treaty, leading to
  the first human study of deep Nonman history.
\item
  430: God-Kings of Trysë are overthrown. The River Aumris cities compete for
  ascendancy.
\item
  500: The city of Ûmerau gains ascendancy, leading to the founding of the Ûmerau
  Empire. A number of Hamori Ketyai tribes settle the length of the River
  Sayut and the Secharib Plains, becoming more sedentary and socially stratified
  as they exploit the rich cereal yields afforded by the fertile soils of the
  region. Seto-Annaria, as it came to be called (after the two most dominant
  tribes), remains a collection of warring city-states.
\item
  549: Nincaerû-Telesser, fourth God-King of the Ûmeri Empire, and famed patron
  133of the ancient Gnostic Schools, is born. The beginning of his reign is unknown.
  The 3rd Ûmeri God-King is still reigning in 570, when Nincaerû-Telesser would
  have been 21.
\end{enumerate}

555: Beginning of the Nonman Tutelage, the great period of Norsirai-Cûnuroi trade,
education, and strategic alliances. The Nonmen who served men at this time
were called Siqû. The Gnosis, first developed by the Nonmen Qûya, is imparted
to the early Norsirai Anagogic sorcerers. First references to benjuka are from
this period.

\begin{enumerate}
\def\labelenumi{\alph{enumi}.}
\setcounter{enumi}{2}
\item
  560: Great Library of Sauglish, is founded by Carû-Ongonean, the third Ûmeri God-
  King. During the reign of the Carû-Ongonean, five Ûmeri translations of
  the Isûphiryas were bequeathed to the Library of Sauglish.
\item
  570: The fortress of Ara-Etrith (``New Etrith''), latterly called Atrithau, is founded
  by Carû-Ongonean, third Ûmeri God-King.
\end{enumerate}

574: Nincaerû-Telesser II, who will transform Great Library of Sauglish into the
cultural heart of the Ancient North, is born.

622: Palpothis III, Old Dynasty God-King of Shigek is born. He will raise
the Ziggurat that bears his name.

\begin{enumerate}
\def\labelenumi{\alph{enumi}.}
\setcounter{enumi}{2}
\tightlist
\item
  642: Nincaerû-Telesser, fourth God-King Ûmeri, dies at age 93.
\end{enumerate}

661: Emilidis founds the Gnostic School of Mihtrûl, or Mihtrûlic.

668: Nincaerû-Telesser II, dies at age 94. During his reign (574-668) the Nonman
Siqû Gin'yursis founds the Gnostic School of Sohonc.

\begin{enumerate}
\def\labelenumi{\alph{enumi}.}
\setcounter{enumi}{2}
\tightlist
\item
  670: Xijoser, the Old Dynasty God-King of Shigek, who will raise the largest of
  the Ziggurats of Shigek, is born.
\end{enumerate}

678: Palpothis III, the Old Dynasty God-King of Shigek, dies at age 56.
684: The Gnostic School of Mangaecca is founded by Sos-Praniura (the greatest
student of Gin'yursis).

\begin{enumerate}
\def\labelenumi{\alph{enumi}.}
\setcounter{enumi}{2}
\item
  687: Gotagga, the great Ûmeri sorcerer credited with the birth of philosophy apart
  from theological speculation, is born.
\item
  720: Xijoser, the Old Dynasty God-King of Shigek, dies at age 50.
\end{enumerate}

735: Gotagga, the great Ûmeri sorcerer, dies at age 48.

\begin{enumerate}
\def\labelenumi{\alph{enumi}.}
\setcounter{enumi}{2}
\tightlist
\item
  750: The Heron Spear, or Suörgil (``Shining Death'' in Ihrimsû), is stolen
  134by Cet'ingira, or Mekeritrig, from the Nonmen of Ishoriöl and delivered to
  Golgotterath. This is at least 2,000 years after the Heron Spear was first taken to
  Ishoriöl (as the spear was said to reside there for ``millennia'').
\end{enumerate}

777: Cet'ingira, or Mekeritrig, reveals the Incû-Holoinas, or Min-Uroikas, to the
School of Mangaecca. The Mangaecca begin studying the Ark to find a way of
bringing down the Barricades. To cover their lengthy absences from Sauglish,
they build the fortress of Nogaral atop Mount Iros in the Urokkas and claim to
be exploring the ruined mansion of Viri.

\begin{enumerate}
\def\labelenumi{\alph{enumi}.}
\setcounter{enumi}{2}
\tightlist
\item
  800: The fortifying of Golgotterath begins. Gwergiruh, the accursed Gatehouse of
  Ûbil Maw, the Extrinsic Gate, is raised.
\end{enumerate}

809: The City of Cenei is founded.

811: Akksersia is founded by Salaweärn I, following the dissolution of the Cond Yoke, originally confined to the city of Myclai. The Cond were pastoralists from the Near Istyuli Plains.

825: Nonman Tutelage ends with the Expulsion, following the famed Rape of Omindalea.

\begin{enumerate}
\def\labelenumi{\alph{enumi}.}
\setcounter{enumi}{2}
\item
  840: Symaul, a Skettic chieftain, executes Wulta-Ongorean, Emperor of Ûmerau.
  After a time, his daughter Avalunsil kills Symaul in vengeance with a fish knife.
  She becomes Empress of All, the first and only female ruler of Ûmerau. She is
  famed for her refusal to marry.
\item
  850: The city of Kelmeöl is founded as a trading stronghold by Akksersian
  colonists, these people would come to be known as the Meöri.
\end{enumerate}

c .860: Assassination of Empress Avalunsil by a spurned suitor.

917: The Cond tribesmen of Aulyanau the Conqueror defeat Ancient Ûmeria at the
Battle of the River Axau. The Cond Yoke collapses rapidly leading to a second
period of Trysean dominance.

927: The Cond tribesmen of Aulyanau the Conqueror defeat Ara-Etrith (``New
Etrith''), latterly called Atrithau, and settle several Cond tribes in the vicinity.
These tribes quickly abandon their pastoral ways and assimilate into Aumris
culture.

\begin{enumerate}
\def\labelenumi{\alph{enumi}.}
\setcounter{enumi}{2}
\tightlist
\item
  1000: Ingusharotep II, Old Dynasty Shigek King who conquered the Kyranae
  Plains, is born.
\end{enumerate}

135c. 1005: Noshainrau the White, the re-founding Grandmaster of the Gnostic School
of Sohonc and author of the Interrogations, the first elaboration of the Gnosis
by Men, is born.

1021: Borswelka I declared King of the Meöri, an aggressive, militaristic city-state.

1023: Beginning of the Old Invishi period in Nilnamesh, when Nilnamesh was united
under a series of aggressively expansionist Kings based in Invishi.

1072: Noshainrau the White, founding Grandmaster of Sohonc, dies at age 67.

1080: Ingusharotep II, Old Dynasty Shigek King, dies at age 80.

1086: Shaeönanra, Grandvizier of the Mangaecca, is born (or this is the year he
became Grandvizier, it is unclear).

1104: Borswelka II King of Meöri, grandson of Borswelka I, dies. Meöri controls most
of the River Vosa Basin and had established trading contacts with Shir to the
south through a series of forts along the River Wernma.

1111: Shaeönanra and Cet'ingira overcome the glamour around Golgotterath and
shatter the Barricades. At some point in the next several years, the
Inchoroi Aurax and Aurang are released and the Unholy Consult of Man,
(represented by Shaeönanra), Nonman (Cet'ingira) and Inchoroi (Aurang) is
founded.

1119: Shaeönanra and Aurang defeat and kill Titirga, Grandmaster of the Sohonc.
Destruction of Nogaral. The Day Lantern disappears from history.

12th c.: Various Ketyai tribes begin asserting their independence from Shigek on the
Kyranae Plains, and the God-Kings of Shigek start waging incessant war.

1123: Shaeönanra, Grandvizier of the Mangaecca, claims to have rediscovered a
means of saving the souls of those damned by sorcery. Mangaecca was promptly
outlawed for impiety. Mangaecca abandon Sauglish and flee to Golgotterath.

13th c.: The city-state of Shir on the River Maurat, subdues all the cities of Seto-
Annaria.

1228: Beginning of the Scintya Yoke, the migratory invasions of White Norsirai
Scintya, in the area of River Aumris and Ara-Etrith, latter Atrithau. Etrithatta,
the original city of the Aumris, is destroyed by the Scintya.

1251: The First Great Sranc War. Akksersia is the largest of the Norsirai nations,
incorporating almost all the White Norsirai tribes save those of the Istyuli
136Plains, covering length of the River Tywanrae, the Plains of Gâl and the entire
north shore of the Cerish Sea.

14th c.: Trysean annals begin referring to Shaeönanra as Shauriatas.

1322: Anzumarapata II, Nilnameshi King of Invishi, inflicts a crushing defeat on the
Shigeki, and transplants hundreds of thousands of indigent Nilnameshi on
the Plains of Heshor, or Amoteu.

1326: Anzumarapata II, Nilnameshi King of Invishi, defeats the Shigeki again, at
compels tribute for some thirty years.

1349: Shigek re-conquers the Middle-Lands of Amoteu.

1378: Anasûrimbor Nanor-Ukkerja I, ``Hammer of Heaven'' (Kûniüric from
Ûmeritic nanar hukisha ), the first Anasûrimbor High King, is born.

1381: End of the Scintya Yoke and emergence of Eämnor as one of the preeminent
nations of the Ancient North.

\begin{enumerate}
\def\labelenumi{\alph{enumi}.}
\setcounter{enumi}{2}
\tightlist
\item
  1400: Zeüm is unified by Mbotetulu, Satakhan of the Ojogi Dynasty.
\end{enumerate}

15th c.: Xiuhianni invaders from Jekk, ravaged the Shiradi Empire and Shir was
razed to the ground. The survivors move the capital to Aöknyssus, and after
some twenty years manage to oust the Eännean invaders.

1408: Anasûrimbor Nanor-Ukkerja I defeats Scintya, seizes the Ur-Throne in Trysë
and declares himself the first High King of Kûniüri, at age 30.

1440: Sranc incursions across the Leash into Wuor, the north-western province of
Kûniüri, begin.

1450: Birth of Iswa, the founder of the Iswazi doctrine of sorcery, in Domyot.

1556: Anasûrimbor Nanor-Ukkerja I dies at age 178, his long life reputedly the result
of the Nonman blood in his veins. In the 148 years of his reign, he had
extended Kûniüri to the Yimaleti Mountains in the north, to the westernmost
coasts of the Cerish Sea in the east, to Sakarpus in the south, and to the Demua
Mountains in the west. At his death, he divided this empire between his sons,
creating Aörsi and Sheneor in addition to Kûniüri proper.

1572: End of the Old Invishi period, of aggressively expansionist Kings, in Nilnamesh.

1591: God-King Mithoser II of Shigek is decisively defeated by the Kyraneans at Narakit, and Shigek begins its long tenure as a tributary to greater powers. Shigek loses regional dominance over Amoteu, the Jarti attempt to reassert ancestral control, with disastrous consequences. The resulting war gave rise to a brief Amoti Empire, which reached the length of the Betmulla Mountains to the frontier of the Carathay Desert.

1601: The fortress of Dagliash is raised by High King Nanor-Mikhus of Aörsi, atop the
ruins of Viri.

1680: Far Wuor is finally abandoned to the Sranc, with a new defensive line drawn up
to the south-east.

1703: The Middle-Lands, the area of Amoteu, fall to Kyraneas.

1798: Girgalla, ancient Kûniüric poet famed for his Epic of Sauglish, is born.

\begin{enumerate}
\def\labelenumi{\alph{enumi}.}
\setcounter{enumi}{2}
\tightlist
\item
  1800: The Scarlet Spires, originally called the Surartu, secured the river fortress
  of Kiz in Carythusal.
\end{enumerate}

1841: Girgalla, ancient Kûniüric poet, dies at age 43.

\begin{enumerate}
\def\labelenumi{\alph{enumi}.}
\setcounter{enumi}{2}
\item
  1896: Ajencis, father of syllogistic logic and algebra, is born in the Kyranean
  capital of Mehtsonc. He would write Theophysics , The First Analytic of
  Men and The Third Analytic of Men.
\item
  1904: At age 8, Ajencis was granted Protection by the Kyranean High King, allowing him to say anything without fear of reprisal, even to the High King.
\end{enumerate}

1966: Ingoswitu, far antique Kûniüric philosopher, is born. He would write Dialogia, and was critiqued by Ajencis.

1991: Horrific plagues inflict the Kyranean capital of Mehtsonc.

2000: Ajencis suffers a stroke and died at the venerable age of 103.

2050: Ingoswitu, far antique Kûniüric philosopher dies at age 84.

2056: Anasûrimbor Mygella, Hero-King of Aörsi, whose deeds are recounted in The Sagas , is born.

2089: Anasûrimbor Celmomas II, last High King of Kûniüri, is born. Seswatha, founder of the School of Mandate, is born to a caste-menial Trysean bronze smith.

2092: Anasûrimbor Nimeric, High King of Ancient Aörsi before its destruction in the Apocalypse, is born.

\begin{enumerate}
\def\labelenumi{\alph{enumi}.}
\setcounter{enumi}{2}
\tightlist
\item
  2100: Uthgai, folklore hero and Scylvendi King-of-Tribes during the Apocalypse,
  is born.
\end{enumerate}

2104: At age 15, Seswatha becomes the youngest sorcerer of rank in the history of the
Sohonc. Anasûrimbor Ganrelka II, successor of Celmomas II and last reigning High King of Kûniüri, is born.

2109: Anaxophus V, Kyranean High King, is born.

2111: Anasûrimbor Mygella, famed Hero-King of Aörsi, dies at age 46.

2115: Ginsil, wife of General En-Kaujalau in The Sagas , who pretended to be her
husband to fool the assassins coming to kill him, is born.

2118: Shikol, King of Ancient Xerash, famed for sentencing Inri Sejenus to death, is
born.

2119: Anasûrimbor Nau-Cayûti, youngest son of Anasûrimbor Celmomas II and his
most prized wife Suriala, is born. Legends have long circulated that Nau-Cayûti
was in fact Seswatha's son.

\textbf{The Apocalypse (2123-2155)}

2123: Nonman Siqû inform the Grandmaster of the Sohonc that the Mangaecca,
or Consult as they had come to be called, had uncovered lost Inchoroi secrets
that would lead to the world's destruction. Seswatha in turn convinced
Anasûrimbor Celmomas to declare war on Golgotterath, known as the Great
Ordeal.

2124: The Great Ordeal battle Consult forces on the Plains of Agongorea. The battle
is indecisive. Celmomas and his allies winter in Dagliash.

2125: The following spring, the Great Ordeal ford the River Sursa, catching their foe
unawares. The Consult withdraw to Golgotterath, and so begin what would be
called the First Investiture. For six years the Ordeal attempt to starve the
Consult into submission, to no avail. Every assault proves disastrous.

2131: Celmomas abandons the Holy War following a dispute with King Nimeric of
Aörsi.

2132: Consult legions, apparently utilizing a vast subterranean network of tunnels,
appear in the Ring Mountains to the rear of the Ordeal. The coalition host is all
139but destroyed. Embittered by the loss of his sons, Nil'giccas, the Nonman King
of Ishterebinth, withdraws altogether, leaving the Aörsi to war alone.

2133: The Aörsi are defeated at the Passes of Amnerlot, and Dagliash was lost soon
after. King Nimeric withdraws to his capital of Shiarau.

2134: Burning of the White Ships; falling back before the Consult legions,
Anasûrimbor Nimeric dispatches the Aörsic fleet to shelter in the Kûniüri port
of Aesorea. Mere days after its arrival, it is burned by agents unknown.
Celmomas acknowledges his folly and mobilizes to relieve Aörsi King Nimeric
at Shiarau.

2135: Anasûrimbor Nimeric is mortally wounded in the Battle of Hamuir, and dies at
age 43.

2136: Shiarau capital of Aörsi falls in spring, and Aörsi is destroyed. The Worldhorn,
a ceremonial sorcerous artifact, is lost with Shiarau.

2137: Nau-Cayûti manages to rout the Consult at the Battle of Ossirish, where he
earns the name Murswagga, or ``Dragonslayer,'' for killing Tanhafut the Red.
His next victory, within sight of Shiarau's ruins, is more complete still. The
Consult's remaining Sranc and Bashrag flee across the River Sursa.

2139: Nau-Cayûti besieges and recaptures Dagliash, and launches several spectacular
raids across the Plains of Agongorea.

2140: Nau-Cayûti's beloved concubine, Aulisi, is abducted by Sranc marauders and
taken to Golgotterath. According to The Sagas Seswatha, after consulting the
Amiolas and the wisdom of Ishterebinth, is able to convince the Prince (who
was once his student) that she can be rescued from the Incû-Holoinas, and the
two of them embark on an expedition that is almost certainly apocryphal.
Mandate commentators dispute the account found in The Sagas , where they
successfully return with both Aulisi and the Heron Spear, claiming that Aulisi
was never found. Whatever happened, at least two things are certain: the Heron
Spear was in fact recovered, and Nau-Cayûti apparently died shortly after at age
21. He is actually poisoned by his first wife, Iëva (later executed for the crime)
and only given the semblance of death. He is retrieved by Aurang and borne to
Golgotterath, where he is tortured for his knowledge of the Heron Spear.

2141: The Consult return to the offensive. At the Battle of Skothera, the Sranc hordes
are crushed by General En-Kaujalau, though he died of mysterious causes
within weeks of this victory (according to The Sagas , he was another victim of
Iëva and her poisons, but again this is disputed by Mandate scholars).

2142: General Sag-Marmau inflicts yet another crushing defeat on Aurang and his
Consult legions, and by fall he had hounded the remnant of their horde to the
Gates of Golgotterath itself. This siege is known as the Second Great
Investiture. During the early part of the Investiture, sixty-one Sohonc sorcerers
(more than a third of the School) are killed in the Chorae Hail.

2143: In spring the No-God is summoned: Initiation. Across the world, Sranc,
Bashrag, and Wracu, all the obscene progeny of the Inchoroi, hearkened to his
call. Sag-Marmau and the greater glory of Kûniüri are annihilated. All Men
could sense his dread presence on the horizon, and all infants were born dead.
The 11 years when all infants were still born comes to be known as the Years of
the Crib. Anasûrimbor Celmomas II had little difficulty gathering support for
his Second Ordeal. Nil'giccas and Celmomas were reconciled. Across Eärwa,
hosts of Men began marching toward Kûniüri.

2146: Battle of Eleneöt Fields is fought between the Horde of the No-God and the
Second Ordeal on Kûniüri's northeastern frontier. Despite having assembled
the greatest host of their age, Anasûrimbor Celmomas and his allies are
unprepared for the vast numbers of Sranc, Bashrag, and Wracu gathered by the
No-God and his Consult slaves. The battle is an unmitigated catastrophe,
signaling the eventual destruction of Norsirai civilization. With his dying words
Anasûrimbor Celmomas II predicts the return of an Anasûrimbor at ``the end
of the world'' to Seswatha. This would come to be known as the Celmomian
Prophecy. Celmomas II dies at age 57. The Heron Spear, which could not be
used because the No-God refused to give battle, was lost. Anasûrimbor
Ganrelka II becomes the last reigning High King of Kûniüri.

2147: All the ancient cities of the Aumris are destroyed, including Trysë and
Sauglish. Four Ûmeri copies of the Isûphiryas were destroyed along with the
Library of Sauglish. The fifth is saved by Seswatha, who later delivered it to the
scribes of the Three Seas. The surviving Kûniüri are either enslaved or scattered. Ganrelka II, last reigning High King of Kûniüri dies at age 43, ending
the Anasûrimbor Dynasty. Ginsil, wife of General En-Kaujalau, dies at age
32. The Nonmen of Injor-Niyas retreat to Ishterebinth and stand siege.
Seswatha is captured during the fall of Trysë and pinned to the Wall of the
Dead at Dagliash, where he is tortured by Cet'ingira for knowledge of the
Heron Spear. Seswatha resists interrogation and escapes.

2148: Eämnor is laid waste, though its capital, Atrithau, survived.

2149: Akksersia, including the capital, Myclai, falls after three disastrous
defeats. Harmant falls as well. Siege of Ishterebinth is raised after the No-God is
repulsed from the Minror Gates, possibly due to the soggomant from the Ark
used to reinforce them.

2150: Kelmeöl falls and the Meöri Empire falls with it.

2151: Inweära falls, though the city of Sakarpus was spared. In autumn, the remnant
Meöri and the Nonmen of Cil-Aujas are victorious against the Consult at
the Battle of Kathol Pass.

2152: In spring, the Meöri turn on their benefactors and sack the ancient Nonman
Mansion of Cil-Aujas.

2153: Forces of the No-God inflict a disastrous defeat on the Shiradi at the Battle of
Nurubal. The next two hundred years of chaos and internecine warfare
effectively destroyed what remained of the Shiradi Empire and its central
institutions.

2154: The Battle of Mehsarunath is fought between the Kyraneas and the host of the
No-God on the Attong Plateau. Aurang, the No-God's Horde-General, won the
battle, but the Kyranean High King, Anaxophus V, is able to escape with much
of his host intact. He abandoned Mehtsonc and Sumna to the Scylvendi. The
Tusk is evacuated and brought to Ancient Invishi in Nilnamesh. Mehtsonc is
destroyed, sealing the fate of Kyraneas. Anaxophus V reveals to Seswatha that
he rescued the Heron Spear from the Fields of Eleneöt in 2146.

2155: The Second Battle of Mengedda, Anaxophus V and his southern tributaries and
allies make their victorious stand against the Horde of the No-God. Seswatha
slays Skafra the Wracu. Wielding the Heron Spear Anaxophus V strikes down
the No-God. Free of his terrible will, his Sranc, Bashrag, and Wracu slaves
disperse. Pestilence swept up from the No-God after his defeat causing
the Indigo Plague, one of the worst in recorded history. The end of the
Apocalypse marks the end of Far Antiquity, and the beginning of Near
Antiquity.

\textbf{Near Antiquity (2156-3351)}

2156: Anaxophus V, Kyranean High King, dies at age 47. Seswatha founds the School
of Mandate.

2157: The Great Pestilence, also known as the Indigo Plague, a devastating pandemic
sweeps Eärwa following the death of the No-God. The tower of Atyersus is
founded by Seswatha as the primary stronghold of the Mandate.

\textbf{Age of Warring Cities (c. 2158-2477)}

\begin{enumerate}
\def\labelenumi{\alph{enumi}.}
\setcounter{enumi}{2}
\tightlist
\item
  2158: The Age of Warring Cities begins. Following the dissolution of Kyraneas,
  cities of the Kyranae Plains are characterized by perpetual warfare. This allowed
  Amoteu independence, though now the Xerashi, the descendants of
  Anzumarapata's settlers, had become its primary competitors.
\end{enumerate}

2158: The tower Attrempus, sister fortress of Atyersus, is founded by Seswatha and
the nascent School of Mandate.

\begin{enumerate}
\def\labelenumi{\alph{enumi}.}
\setcounter{enumi}{2}
\tightlist
\item
  2159: Inri Sejenus, the Latter Prophet, is born.
\end{enumerate}

2168: Seswatha, founder of the School of Mandate, dies at age 79.

\begin{enumerate}
\def\labelenumi{\alph{enumi}.}
\setcounter{enumi}{2}
\tightlist
\item
  2170: Uthgai, Scylvendi King-of-Tribes during the Apocalypse, dies at
  approximately age 70.
\end{enumerate}

2198: King Shikol of Ancient Xerash, sentences Inri Sejenus to death. Shikol is 80
and Inri Sejenus is 39.

2202: King Shikol dies at age 84. Inri Sejenus is said to ascend to the Nail of
Heaven at age 43.

2300: Teres Ansansius, most famed theologian of the early Thousand Temples, is
born. He would go on to write, The City of Men, The Limping Pilgrim, and Five Letters to All which are revered by Shrial scholars.

2304: Ekyannus I, first ``institutional'' Shriah of the Thousand Temples, and author
of 44 Epistles, is born.

2338: Stajanas II, ``Philosopher-Emperor'' of Cenei, author of Ruminations, is born.

2349: The city of Cenei conquers Gielgath, sealing its regional dominance. In the
ensuing decades the Ceneians under Xercallas II would secure the remnants of
what had once been Kyraneas. Xercallas's successors continued his aggressive,
expansionist policies, first pacifying the Norsirai tribes of Cepalor.

2350: Kiz, home of the Scarlet Spires, is severely damaged in an earthquake. The
fortress is covered with red enamel tiles in the reconstruction, thus leading to
the School's now-famous moniker.

2351: Teres Ansansius, famed theologian of the early Thousand Temples, dies at
approximately age 51.

2372: Ekyannus I, first ``institutional'' Shriah of the Thousand Temples, dies at age
68.

2390: The Zealot Wars, a prolonged religious conflict between the early Inrithi and
the Kiünnat, begin.

2395: Stajanas II, ``Philosopher-Emperor'' of Cenei, dies at age 57. Pirras Boksarias,
Ceneian Emperor who standardized trading protocols within the empire and
established a thriving system of markets in its major cities, is born.

2397: Shigek falls to Cenei after three consecutive wars.

2412: Stajanas II, ``Philosopher-Emperor'' of Cenei, begins ruling.

2414: General Naxentas of Cenei conquers Enathpaneah, Xerash, and Amoteu. He
then staged a successful coup and declared himself Emperor of Cenei.

2415: Naxentas, self declared Emperor of Cenei, is assassinated.

2431: The reign of Stajanas II, ``Philosopher-Emperor'' of Cenei, ends.

2432: Ekyannus III, ``the Golden,'' Shriah of the Thousand Temples, is born.

2437: Pirras Boksarias, Ceneian Emperor, dies at age 42.

2456: Triamis the Great, first Aspect-Emperor of the Ceneian Empire, is born.

2458: Inrithi fanatics lead the province of Amoteu in a vicious rebellion against
Cenei. As punishment, Emperor Siaxas II butchers the inhabitants of Kyudea and razes the city to the ground.

2466: Memgowa, famed near antique Zeümi sage and philosopher, is born. He would
later write Celestial Aphorisms and The Book of Divine Acts.

2469: Part of the Zealot Wars, Sumna capitulates to Shrial forces, but hostilities
continue.

2477: Age of Warring Cities ends.

\textbf{Age of Cenei (2478-3351)}

2478: Triamis I (the Great) is anointed Emperor at age 22. Triamis I enacts the
constitution governing the division of powers between the Imperium and the
Thousand Temples, leading to the end of the Zealot Wars. This year marks the
beginning of the Age of Cenei, also known as the Ceneian Golden Age.

2483: Triamis I defeats Sarnagiri V, leading a coalition of Nilnameshi Princes.
Nilnamesh becomes a Ceneian province for more than a thousand years.

2484: Triamis I conquers Cingulat.

2485: Triamis I defeated a great Zeümi host at Amarah, and would have invaded the
Satyothi nation had not mutinies among his homesick troops prevented him.
He spent the next decade consolidating his gains, and striving against the
internecine religious violence between followers of the traditional Kiünnat sects
and the growing numbers of ``Inrithi.''

2500: Shriah Ekyannus III (then age 68), formally institutionalizes the so-called
Emperor Cult. Triamis the Great (age 44), in the twenty-third year of his rule,
takes the title Aspect-Emperor, which is adopted by all his successors.
2505: Triamis I converts under Ekyannus III and declares Inrithism the official state
religion of the Ceneian Empire. He spent the next ten years putting down
religious rebellions.

2506: Memgowa, near antique Zeümi sage and philosopher, dies at age 40. Memgowa
is primarily known in the Three Seas for his Celestial Aphorisms and The Book
of Divine Acts.

2508: Triamis I invades and occupies Cironj.

2511: Triamis I invades and occupies Nron. Ekyannus III ``the Golden'' founds
the Shrial Knights, a monastic military order charged with prosecuting the will
of the Shriah.

2516: Ekyannus III ``the Golden'' dies at age 84.

2518: Triamis I conquers Ainon.

2519: Triamis I conquers Cengemis.

2525: Triamis I conquers Annand.

2568: The Triamic Walls, Caraskand's outermost fortifications, are raised by Triamis
the Great.

2577: Triamis the Great dies at age 121.

2789: Muretetis, ancient Ceneian scholar-slave, is born. He will go on to write Axioms
and Theorems , the founding text of Three Seas geometry.

2847: Xius, great Ceneian poet and playwright, famed for The Trucian Dramas , is
born.

2864: Muretetis, ancient Ceneian scholar-slave, dies at age 75.

2870: Protathis, famed near antique poet of Ceneian descent, is born. He will go on
to write many works, including The Goat's Heart , One Hundred Heavens , and
the magisterial Aspirations.

2875: Ontillas, near antique Ceneian satirist most famous for On the Folly of Men , is
born.

2881: Olekaros, Ceneian slave-scholar of Cironji descent, famed for his Avowals , is
born.

2914: Xius, great Ceneian poet and playwright, dies at age 67.

2922: Protathis, famed near antique poet of Ceneian descent, dies at age 52.

2933: Ontillas, near antique Ceneian satirist, dies at age 58.

2956: Olekaros, Ceneian slave-scholar of Cironji descent, dies at age 75.

2981: Gaeterius, Ceneian slave-scholar, is born. He will go on to write commentaries
on The Chronicle of the Tusk collected under the title Contemplations on the
Indentured Soul.

3045: Gaeterius, Ceneian slave-scholar, dies at age 64.

3081: Casidas, famed philosopher and historian of Near Antiquity, best known for his
magisterial The Annals of Cenei, is born.

3142: Casidas, famed philosopher and historian of Near Antiquity, dies at age 61.

3174: Hatatian, infamous author of the Exhortations , a work that eschews traditional
Inrithi values and espouses an ethos of unprincipled self-promotion, is born.

3211: Hatatian, infamous author of the Exhortations , dies at age 37. Opparitha, near
antique Cengemian moralist most famous for his On the Carnal, is born.

3256: Throseanis, late Ceneian dramatist, famed for his Triamis Imperator , a
dramatic account of the life of Triamis I, is born.

3299: Opparitha, near antique Cengemian moralist, dies at age 88.

3317: Throseanis, late Ceneian dramatist, dies at age 61. Sarothesser I, founder of
High Ainon, is born.

3351: Cenei is destroyed by the Scylvendi under Horiötha. The Heron Spear is lost in
the attack. The Sack of Cenei marks the end of both the Age of Cenei and Near
Antiquity. Batathent, a fortress-temple dating back to pre-classical Kyraneas, is
destroyed by the Scylvendi shortly after the fall of Cenei.

\textbf{Present Era (3352- )}

3371: Shriah Diagol, known for his cruel excesses, holds the Seat.

3372: Cenei General Maurelta surrenders to Sarothesser I at the Battle of Charajat.
This marks the traditional collapse of the Ceneian Empire. Sarothesser I
ascends the Assurkamp Throne as the first Ainoni King, at age 30. Following
the fall of the Ceneian Empire, Cengemis gains independence, in Nilnamesh,
the New Invishi period begins.

3374: Conriya is founded around Aöknyssus, the ancient capital of Shir.

3383: Shriah Diagol, known for his cruel excesses, is assassinated after 12 years.

3386: Writ of Psata-Antyu is issued by the high clergy of the Thousand Temples at the
Council of Antyu to limit the power of the Shriah. The Writ was motivated by
the cruel excesses of Shriah Diagol.

3402: Sarothesser I, first king of High Ainon, after ruling for 55 years, dies at age 85.

3411: Beginning of the Trimus Emperors' rule of Nansur (the traditional name for the
district surrounding Momemn). Under the Trimus Emperors, Nansur unified
the Kyranae Plains.

3470: Zerxei Triamarius I, first of the Zerxei Emperors of Nansur, is born.

3508: Lasting for 97 years, the Trimus Dynasty in Nansur ends when Trimus
Meniphas I is assassinated.

3511: At age 41, Triamarius I is acclaimed by the Imperial Army as the
first Zerxei Emperor of Nansur, beginning the Zerxei Dynasty.

3517: Triamarius I, first Zerxei Emperor of Nansur, dies at age 47 after ruling Nansur
for 6 years.

3539: Nansur conquers Shigek.

3569: Nansur conquers Enathpaneah.

3574: Nansur conquers the Sacred Lands (Xerash and Amoteu).

3588: Zerxei Triamarius III, last of the Zerxei Emperors of Nansur, is born.

3619: Zerxei Triamarius III is assassinated by his palace eunuchs at age 31. This marks
the end of 108 years of Zerxei rule over Nansur, and the beginning of the Surmante Emperors. Surmante Skilura II, a future Emperor of Nansur, is born.

3639: Surmantic Gates, the great northern gate of Carythusal, are built and financed
by Surmante Xatantius I to commemorate the ill-fated Treaty of Kutapileth, a
short-lived military pact between Nansur and High Ainon.

3644: Surmante Xatantius I, Emperor of Nansur is born.

3666: Pherokar I, One of Kian's earliest and fiercest Padirajahs, is born.

3668: The deranged antics Surmante Skilura II ``the Mad,'' lead to the Granary
Revolts. Skilura II dies at age 49, and Xatantius I takes the throne at age 24.

3669: Fane, Prophet of the Solitary God and founder of Fanimry is born.

\begin{enumerate}
\def\labelenumi{\alph{enumi}.}
\setcounter{enumi}{2}
\tightlist
\item
  3683: Galeoth proper begins when King Norwain I reputedly concluded twenty
  148years of campaigning and conquest by having his captive foes butchered en
  masse in the reception hall of Moraör, the great palace complex of the Galeoth
  Kings.
\end{enumerate}

3684: In Caraskand, Xatantius raises the fortress Insarum (later called the Citadel of
the Dog).

3688: Zarathinius, famed author of A Defence of the Arcane Arts, is born.

3693: Surmante Xatantius I dies at age 49, having ruled for 25 years. During his reign,
Xatantius I enlarged the Nansur Empire to its greatest extent. He subdued the
Norsirai tribes of the Cepalor as far north as the River Vindauga. For a time he
even managed to hold the far southern city of Invishi (though he failed to
entirely subdue the Nilnameshi countryside). Despite his military successes, his
continual wars exhausted both the Nansur people and the Imperial Treasury
and his practice of debasing the talent in order to finance the wars wrecked the
empire's economy. This inadvertently lay the groundwork for the disastrous
wars against the Kianene following his death.

3703: Fane, a Shrial Priest in the Nansur province of Eumarna, is declared a heretic
by the ecclesiastical courts of the Thousand Temples and is banished to certain
death in the Carathay Desert. According to Fanim tradition, rather than dying
in the desert, Fane went blind, experienced the series of revelations narrated in
the kipfa'aifan, the ``Witness of Fane'' in Kianni, and was granted miraculous
powers (the same powers attributed to the Cishaurim) he called the Water of
Indara. He spent the remainder of his life preaching to and consolidating the
desert tribes of the Kianene.

\begin{enumerate}
\def\labelenumi{\alph{enumi}.}
\setcounter{enumi}{2}
\tightlist
\item
  3704: The Kianene tribes begin to convert to Fanimry.
\end{enumerate}

3711: Hamishaza, renowned Ainoni dramatist and author of Tempiras the King, is born.

3716: Fan'oukarji I, the son of the Prophet Fane and the first Padirajah of Kian, is
born. Fane was 47 at the birth of his son, and had been living with the Kian for
13 years.

3722: Surmante Caphrianus I ``the Younger,'' Nansur emperor famed for his wily
diplomacy and far-reaching reforms of the Nansur legal code, is born. The Tydonni tribes overwhelm the Men of Cengemis at the Battle of Marswa.

1493724: After living with the Kian for 21 years, Fane managed to convert all the Kianene
tribes.

38th c.: Fanic missionaries would succeed in converting the Girgashi to Fanimry in
the thirty-eighth century.

3739: Meigeiri, administrative and spiritual capital of Ce Tydonn, is founded about
the Ceneian fortress of Meigara.

3741: King Haul-Namyelk finally succeeds in unifying the various Tydonni tribes
under his absolute authority, Ce Tydonn proper comes into existence.

3742: Cengemis is overrun by Tydonni tribes, ending its 370 years of independence.
Ce Tydonn is founded in the wake of Cengemis's collapse. Fane dies at age 73,
39 years after he was banished to the Carathay Desert.

3743: At age 27, Fan'oukarji I, first Padirajah of Kian, begins the White Jihad against
the Nansur Empire.

3745: Zarathinius, author of A Defence of the Arcane Arts, dies at age 57.

3752: Fan'oukarji I founds Nenciphon as the administrative capital of Kian, on the
banks of the River Sweki.

3759: Mongilea becomes original conquest of Fan'oukarji I, it latter be known as the
``Green Homeland'' of the Kianene.

3771: Fan'oukarji I dies at age 55, and after fighting for 28 years, the White Jihad dies
with him. In addition to founding Nenciphon and conquering Mongilea, he also made serious inroads into Eumarna.

3783: Hamishaza, renowned Ainoni dramatist, dies at age 72.

3785: Surmante Caphrianus I ``the Younger,'' Nansur emperor, dies at age 63.

\begin{enumerate}
\def\labelenumi{\alph{enumi}.}
\setcounter{enumi}{2}
\tightlist
\item
  3787: As a result of continually pressured by the Sranc tribes that largely ruled the
  great forests of the Dameöri Wilderness, the Thunyeri migrated down the length of the Wernma River. The Thunyeri begin to ply the Three Seas as pirates and raiders, for the next two hundred years.
\end{enumerate}

3796: By order of Ekyannus XIV, the Scholastic Wars begin. Made up of series of holy wars waged against the Schools, the Scholastic Wars saw the near-destruction of several Schools and the beginning of the Scarlet Spires' hegemony over High Ainon.

3801: During the height of the Scholastic Wars, Grandmaster Shinurta of the Scarlet Spires creates the Javreh slave-soldiers.{[}150{]} Kian captures Eumarna from Nansur during a Jihad.

3804: To defend themselves during the Scholastic Wars, the Mikka Council from Cironji, the Oaranat from Nilnamesh, and the (Cengemic) Nilitar Compact from Ce Tydonn join together to form the Mysunsai ``mercenary School.'' During the War under the terms of the infamous Psailian Concession, the Mysunsai assisted the Inrithi in their Ainoni campaigns.

3808: Sorainas, celebrated Nansur scriptural commentator, and author of The Book of Circles and Spirals, is born.

3817: House Morghund becomes the ruling dynasty of Atrithau.

3818: After 22 years, the Scholastic Wars come to an end. By this time, the School of
the Scarlet Spires, based in Carythusal, managed to destroy the army of
King Horziah III and assumed indirect control of High Ainon.

3821: Pherokar I, one of Kian's earliest and fiercest Padirajahs, dies at age 155.

3823: Nersei Onoyas II, King of Conriya who first forged the alliance between the
School of Mandate and House Nersei, is born.

3839: Caraskand, and its fortress Insarum (later called the Citadel of the Dog), are
captured by the Fanim. They rename the fortress Il'huda, ``the Bulwark'' in Kianni.

3842: Kian captures Enathpaneah in a Jihad.

3845: Kian captures both Xerash and Amoteu. The College of Marucee, a College of
the Thousand Temples is destroyed in the Sack of Shimeh.

3878: Nersei Onoyas II, King of Conriya who first forged the alliance between the
School of Mandate and House Nersei, dies at age 55.

3892: Habal ab Sarouk, first organizes the Coyauri, the famed elite heavy cavalry of
the Kianene Padirajah, as a response to the Nansur Kidruhil.

3895: Sorainas, celebrated Nansur scriptural commentator, and author of The Book
of Circles and Spirals, dies at age 87.

3905: Anwurat, a large Kianene fortress to the south of the River Sempis Delta is
built.

3921: The School of Mandate give the tower Attrempus to be held in trust by House
Nersei of Conriya.

3933: The Kian conquer both Shigek and Gedea during the Dagger Jihad of Fan'oukarji III. The College of Sareöt, a College of the Thousand Temples dedicated to the preservation of knowledge, was destroyed during the fall of Shigek. However, their library, the Sareötic Library, was spared by Fan'oukarji III, thinking it the will of the Solitary God. After the fall of Shigek, the Nansur built a number of small fortresses in the Gedean interior,
including Dayrut, Ebara and Kurrut.

3941: Following a coup brought about by the turmoil following the loss of Shigek to
the Kianene, the Surmante Emperors lose control of the Nansur after 322 years.
A former Exalt-General, Ikurei Sorius I reorganized both the Imperial Army and
the empire, becoming the first Ikurei emperor. These changes allowed him and
his descendants to defeat no fewer than three full-scale Fanim invasions.

3942: The entire line of King Nejata Medekki of Conriya is murdered during the
Aöknyssian Uprisings. House Nersei becomes the ruling House of Conriya.

3987: After three generations of Inrithi missionaries had largely converted the
Thunyeri from their traditional Kiünnat beliefs, the tribes elected their first
King, Hringa Hurrausch, and began adopting the institutions of their Three
Seas neighbours.

4000: By the end of the fourth millennium Kian was easily the pre-eminent military
and commercial power of the Three Seas, and a source of endless consternation
not only for the much-diminished Nansur Empire but for Inrithi Princes in
every nation. Haurut urs Mab, an Utemot memorialist when Cnaiür urs
Skiötha was a child, is born.

4009: Psailas II, Shriah of the Thousand Temples, is born.

4022: Ikurei Anphairas I, Emperor of Nansur, is born.

4036: Charamemas, famed Shrial commentator and author of The Ten Holies, is
born.

4038: Okyati urs Okkiür, cousin of Cnaiür urs Skiötha, is born. Skiötha urs Hannut,
father of Cnaiür urs Skiötha, and former Chieftain of the Utemot, is born.

4049: Sasheoka, Grandmaster of the Scarlet Spires, is born.

4054: Am-Amidai, a large Kianene fortress located in the heart of the Atsushan
Highlands is raised.

4062: Aethelarius VI, later King of Atrithau, is born.

\begin{enumerate}
\def\labelenumi{\alph{enumi}.}
\setcounter{enumi}{2}
\tightlist
\item
  4063: Drusas Achamian is born.
\end{enumerate}

4064: Sancla, Drusas Achamian's cellmate and lover during his adolescence in
Atyersus, is born.

4066: Ikurei Anphairas I, becomes Emperor of Nansur at age 44.

4067: Hasjinnet ab Skauras, eldest son of Skauras ab Nalajan, is born.

4072: Psailas II, becomes Shriah of the Thousand Temples at age 63. Psailas II
censured King Sareat II of Galeoth. As a result, fairly half of his client nobles
rebelled, and Sareat was forced to walk barefoot from Oswenta to Sumna in
contrition. Cutias Sarcellus, Knight-Commander of the Shrial Knights, is born.

4075: Nersei Tirummas, eldest brother of Nersei Proyas, and Crown Prince of
Conriya, is born.

4076: Birth of Anasûrimbor Kellhus.

4079: Skiötha urs Hannut, father of Cnaiür urs Skiötha, and former Chieftain of the
Utemot, dies at age 41.

4080: Okyati urs Okkiür, cousin of Cnaiür urs Skiötha, brings Anasûrimbor
Moënghus as a captive to the Utemot camp.

4081: Ikurei Anphairas I, Emperor of Nansur and grandfather of Ikurei Xerius III, is
assassinated by persons unknown. He had reigned for 15 years, and was 44.

4082: Okyati urs Okkiür, cousin of Cnaiür urs Skiötha, dies at age 44. Haurut urs
Mab, Utemot memorialist when Cnaiür was a child, dies at age 82.

4083: Sancla, Achamian's cellmate and lover during his adolescence in Atyersus, dies
at age 19.

4086: Psailas II, Shriah of the Thousand Temples, dies at age 77, having led the
Thousand Temples for 14 years.

4092: Conriya and Ce Tydonn fight the minor Battle of Maän.

4093: At age 57, Charamemas, famed Shrial commentator, replaces Achamian as
Proyas's tutor in exoterics.

4099: Cutias Sarcellus, Knight-Commander of the Shrial Knights, is murdered and
replaced by Consult skin-spies, at age 27. Shoddû Akirapita, Prince of Nilnamesh, is born.

4100: Nersei Tirummas, eldest brother of Nersei Proyas, and Crown Prince of
Conriya, dies at sea at age 25. Sasheoka, Grandmaster of the Scarlet Spires, is
assassinated by the Cishaurim for reasons unknown. Hanamanu Eleäzaras becomes the new Grandmaster.

4103: The Kianene host of Hasjinnet ab Skauras and the Scylvendi under Yursut urs
Muknai meet on the Jiünati Steppe at fight the Battle of Zirkirta. Kianene
cavalry proved no match for the Scylvendi, and Hasjinnet himself was slain.
However, the Kianene were quick in recovering, and most of the ill-fated
expedition survived. The first of the Galeoth Wars are fought between Galeoth
and the Nansur Empire in 4103--4104. In each case the Galeoth, under the
generalship of Coithus Saubon, enjoyed early successes, only to be subsequently
defeated in more decisive engagements.

4106: More Galeoth Wars are fought, last of which was the Battle of Procorus,
where Ikurei Conphas commanded the Imperial Army against Coithus Saubon.

4108: Charamemas, famed Shrial commentator, author of The Ten Holies and
Achamian's replacement as Proyas's tutor in exoterics, dies at age 72.

4109: Conriya and Ce Tydonn fight the Battle of Paremti. This is the first military
victory of Prince Nersei Proyas. Historically significant because Proyas had his
cousin, Nersei Calmemunis, whipped for impiety.

\textbf{The Coming of Kellhus}

4110: Vulgar Holy War. Battle of Kiyuth.

4111: Sudica, province of the Nansur Empire, is largely depopulated.

4111-4112: First Holy War and the emergence of Anasûrimbor Kellhus, first the
Prince of Nothing, then the Warrior-Prophet and finally the Aspect-Emperor of
the Three Seas.

4112: A Werigda tribe are interrogated by the Consult for information on the Dûn-
yain.

4112-4126: Unification Wars. The Kellian or New Empire accumulates from the
detritus of the nations that preceded it.

4113: The Year of the Child Grandees; Nenciphon falls, Kianene Empire dissolved.

4114: The Novum Arcanum is circulated throughout the Three Seas; Rash
Soptet (4088--- ) is hailed as ``Lord of the Sempis'' after quelling Fanim
uprisings. Schismatics denounce Maithanet; the War-between-Temples begins.
Kellhus issues the Rehabilitation of Sorcery, rescinding all Shrial and Tusk
condemnations of sorcery. Kellhus also issues the Manumission of the
Feminine, awarding equal social, legal and moral status to the women of the
Three Seas. Combined, the two also allow the legitimising of female sorcerers.
The Swayal Compact is founded shortly after this time, with Kellhus ordering
them to be trained in the Gnosis by the Mandate.

4115: Prince Shoddû Akirapita (4099---4123) routes the first Zaudunyani invasion of
Nilnamesh at the Battle of Pinropis.

4116: The death of King Eryeat, combined with the secret conversion of his eldest
surviving son, Coithus Narnol, delivers Galeoth to the Empire nearly intact.
King Hringa Vûkyelt expels Schismatics from Thunyerus.

4117: The first songs extolling the exploits of Sasal Charapatha against the
Nilnameshi Orthodox begin circulating throughout the Three Seas; First
Carythusali uprising; Earl Couras Nantilla is Whelmed, raises Cengemic
provincesin revolt against Meigeiri; the Tydonni Orthodox begin massacring
Ketyai villages and towns along the Eleterine Coast.

4118: Meigeiri falls; Anasûrimbor Kellhus orders the Orthodox of Numaineiri
blinded; Eselos Mursidides (4081---4132) conquers Cironj for the Zaudunyani
losing, miraculously, only one hundred and eighteen souls.

4119: The Koraphean Uprising; Hoga Hogrim (4093--- ) is declared Zaudunyani
Believer-King of Ce Tydonn; King Hringa Vûkyelt of Thunyerus declares
himself a Believer-King as well; the Mandate takes up residence in Kiz. The
Compendium , a work critical of the Aspect-Emperor written by the Holy Tutor
Drusas Achamian, is published.

4120: Anasûrimbor Kellhus declares Holy Bounty on Sranc scalps; Sack of Sarneveh;
Circulation of the Toll pamphlet, and subsequent Toll uprisings.

4121: Nurbanu Soter (4069--- ) declared King-Regent of High Ainon; Invishi falls after
the famed Throwing-of-the-Hulls. Kellhus spends four months studying with
155Heramari Iyokus, the Blind Necromancer and expert of the Daimos. At the end
of the tutelage, Kellhus emerged with two demonic heads -- the Decapitants --
tied to his hip. Kellhus would prove reluctant to explain their origin or purpose.

4122: Nilnameshi Orthodox crushed at the Battle of Ushgarwal. Anasûrimbor Kellhus
declares the Unification Wars concluded. The Shriah of the Thousand
Temples, Maithanet, proclaims him Holy Aspect-Emperor of the Three Seas.
The Ekkinû, a sorcerous arras or tapestry made up of shifting, unknown figures,
appears in Kellhus's possession. Its origin and purpose is unknown.

4123: Prince Shoddû Akirapita (4099---4123) is found drowned in a well in Girgash.
Only Fanayal ab Kascamandri remains of the Empire's notorious enemies.

4124: Reconstruction of Auvangshei begins. The School of Mandate is reconstituted
as the Imperial School of the New Empire, although it popularly remains
known by its former name.

4125: First of the Angnaya are sent to the Palace of Plumes in Zeüm. The Scalper
Purges take place following reports of widespread corruption in the reporting
and paying of the Holy Bounty.

\begin{enumerate}
\def\labelenumi{\alph{enumi}.}
\setcounter{enumi}{2}
\tightlist
\item
  4126: Approximate destruction of Ishuäl.
\end{enumerate}

4129: The Tower of Grojehald, an outpost of Sakarpus, falls to the Sranc.

Spring, 4132: The Great Ordeal begins. Achamian and Mimara ally with the Skin
Eaters and set out to find the Great Library of Sauglish, and the secret location
of Ishuäl.

Autumn, 4132: The Great Ordeal confronts the Unholy Consult in battle at
Golgotterath. Cet'ingira, Aurang, Nersei Proyas and many others are killed.
Anasûrimbor Kellhus reveals that he forged an alliance with Ajokli to defeat the
Consult, which he realised had been subverted by the Dûnyain. Betrayed by his
son Kelmomas, Kellhus was killed by a Chorae. Kelmomas was fed into the
Carapace, becoming the second incarnation of the No-God. The Great Ordeal
is destroyed utterly. Achamian, Mimara, Esmenet and a tiny handful escape the
conflagration. Cnaiür urs Skiötha is killed and his son Moënghus succeeds him
as King-of-Tribes.

The Second Apocalypse begins.

\textbf{References}

The Second Apocalypse by R. Scott Bakker

The Prince of Nothing
The Darkness That Comes Before (2003)
The Warrior-Prophet (2004)
The Thousandfold Thought (2005)

\textbf{The Aspect-Emperor}

The Judging Eye (2009)
The White-Luck Warrior (2011)
The Great Ordeal (2016)
The Unholy Consult (2017)

\textbf{Atrocity Tales (short stories)}

The Four Revelations of Cinial'jin (2011)\emph{
The False Sun (2012)}
The Knife of Many Hands (2015)
The Carathayan (2017)

\begin{itemize}
\tightlist
\item
  Published in The Unholy Consult
\end{itemize}

\textbf{Websites}

The Second Apocalypse trailer
R. Scott Bakker's website
Three Pound Brain (R. Scott Bakker's blog)
Prince of Nothing Wiki
Second Apocalypse Forum
157Second Apocalypse Facebook Group
R. Scott Bakker Reddit Group
Jason Deem's DeviantArt page
Jason Deem at ArtStation
The Wertzone (Adam Whitehead's blog)


\end{document}
